\documentclass{beamer}
\usepackage[english]{babel}
% Designelemente
\usetheme{Hannover}
\beamertemplatenavigationsymbolsempty

\newenvironment{code}{\footnotesize\tt \begin{tabbing}
\hskip12pt\=\hskip12pt\=\hskip12pt\=\hskip12pt\=\hskip5cm\=\hskip5cm\=\kill}
{\end{tabbing}}

\parskip6pt

\title[The SymbolicData Project]{The SymbolicData Project -- a Community Driven
  Project for Computer Algebra}

\author{Hans-Gert Gr\"abe}

\institute[]{Leipzig University, Germany\\
\texttt{http://bis.informatik.uni-leipzig.de/HansGertGraebe}}

\date{ACA-2016, Kassel, August 1, 2016}
\begin{document}
\begin{frame}
\titlepage
\end{frame}

\begin{frame}
\frametitle{Preliminary Remarks}

This talk (and all the session) is about Computer Algebra,
\begin{itemize}
\item but \emph{not} about algorithms,
\item not even about research at all,
\item but \textbf{about research infrastructure}.
\end{itemize}

Digital change -- the Big Challenge:
\begin{quote}
  Turn the \textbf{hyperlinked} Digital Universe into an \textbf{interlinked}
  one -- the Semantic Web.
\end{quote}

The \textbf{interlinked Digital Universe} can be viewed as a single
(distributed) global database.
\end{frame}

\begin{frame}
\frametitle{Preliminary Remarks}
What does it mean?

\emph{The pessimist's version:} We loose all control, no more privacy. All my
real world social relations and activities have to be exposed and become
visible in the web. Even today they are yet explored by Google, Facebook and
WhatsApp.

\emph{The optimist's version:} Let's control the loss. Use appropriate means of
the web to actively shape virtual visibility of your social relations and
activities. Since only \textbf{real world} social relations and activities
matter.

The latter requires individual engagement \emph{and} cooperative efforts.  
This talk is about cooperative efforts.

\end{frame}

\begin{frame}\frametitle{The SymbolicData Project}
The SymbolicData Project is on such a way since 1998, i.e., for a long time,
and tries to secure a part of a common research data infrastructure within
Computer Algebra.
\begin{itemize}
\item SymbolicData is a small voluteers' project (intentionally) without own
  funding.
\item We acknowledge support from different players within the CA community at
  different time, in particular from the board of the German Fachgruppe, and
  also from a small project within the „Saxonian E-Science Initiative“.
\end{itemize}
\end{frame}

\begin{frame}\frametitle{The SymbolicData Project}

The SymbolicData Project had different focuses over the time:
\begin{description}
\item[Version~1] (1999--2002): Secure uniform digital open access to research
  data about Polynomial Systems collected within the PoSSo (1992--1995) and
  FRISCO (1996--1999) EU funded projects. \medskip

  Develop concepts and tools for certified benchmarks on that data.
\item[Version~2] (2006--2013): Transform SymbolicData concepts and data to the
  evolving RDF based concepts, standards, protocols, and tools of the Semantic
  Web.
\end{description}
\end{frame}
\begin{frame}\frametitle{The SymbolicData Project}
\begin{description}
\item[Version~3] (since 2013): Provide „fingerprints“ of other collections of
  CA research data, in partucular polytopes within the \emph{polymake} project,
  examples from the \emph{normaliz} collection and the \emph{database of
    transitive groups} (Kl\"uners, Malle).\medskip

  First steps towards the RDF basics of a \emph{CA Social Network}, in
  particular to unify information processes reflected in the web portal of the
  German Fachgruppe.
\end{description}

And now lets go to \url{http://wiki.symbolicdata.org/Events.2016-08.Graebe}
\end{frame}
\end{document}
