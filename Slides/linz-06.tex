% H.-G. Graebe | Univ. Leipzig 
% $Date: 2006/03/13 15:10:02 $

\documentclass{slides}
\usepackage{url,amsmath,amssymb,fancybox}
\usepackage[a4paper,landscape]{geometry}
\usepackage[latin1]{inputenc} 

\parindent0cm
\parskip2pt
\newcommand{\section}[1]{\begin{center}\bf #1 \end{center}}

\newcommand{\Rahmen}[2]{
\setlength{\fboxsep}{12pt}\begin{center}
\shadowbox{\parbox{#1\textwidth}{\em #2}}\end{center}}

\title{\Huge The SymbolicData Project\\[4cm] \small Talk given at the Special
  Semester\\ on Groebner Bases Linz 2006 }

\author{Hans-Gert Gr\"abe,\\ Dept.\ Computer Science, Univ.~Leipzig,
Germany\\ \url{http://www.informatik.uni-leipzig.de/~graebe}}

\date{March 8, 2006}

\begin{document}

\begin{slide}
  \maketitle
\end{slide}

\begin{slide}
\section{Motivation}\small
(1) For different purposes algorithms and implementations are tested on
certified and reliable data.

(2) The development of tools and data for such tests is usually
``orthogonal'' to the main implementational efforts.

(3) In many cases tools and data could easily be reused - with slight
modifications - across similar projects.

\Rahmen{.8}{The SymbolicData Project is set out to coordinate such efforts
within the Computer Algebra Community.}

(4) Commonly collected certified and reliable data can also be used to compare
otherwise incomparable approaches, algorithms, and implementations. 

(5) Benchmark suites and Challenges for symbolic computations are not as well
established as in other areas of computer science. 
\end{slide}

\begin{slide}
\section{Main Goals}

\Rahmen{.9}{(1) Collect data of examples from various areas of Computer
Algebra
\begin{itemize}
  \item in a systematic and uniform way together with related background
  information;

  \item in a form that conveniently allows to extend, manipulate, and
  categorize the collected data;

  \item such that they can be extracted in a form readable by different
  Computer Algebra Software;
  
  \item such that interrelations of the collected data can be specified;
\end{itemize}
}
\end{slide}

\begin{slide}
\Rahmen{.9}{(2) Share best practice experience how to run test computations on
   these data, i.e.,
\begin{itemize}
  \item to prepare data for input to different Computer Algebra Software;

  \item to set up, start, time, interrupt, and monitor the computations;

  \item to collect, analyse, and evaluate output data from these computations; 
\end{itemize}

(3) Run a net of Web sites that present data from the collection and results
of test computations.}

\end{slide}
\begin{slide}
\section{Some History}\small

The SymbolicData project started at the ISSAC-98 special session on
Benchmarking organized by H. Kredel. Mainly driven by Olaf Bachmann (Singular
group, Kaiserslautern) and Hans-Gert Gr\"abe (Univ. Leipzig).

1999: joint forces with the symbolic computation groups of the University of
Paris VI (J. C. Faugere, D. Lazard), of Ecole Polytechnique (J. Marchand,
M. Giusti), and of the University of Saarbr�cken (W. Decker).  

1999: Incooperated into the benchmarking activities of the "Fachgruppe
Computeralgebra" (Chair at those times: G.-M. Greuel).

\end{slide}
\begin{slide}
\section{Some History (2)}\small
1999--2000: Main design decisions and implementations of the first prototype
(O. Bachmann, H.-G. Gr\"abe) during two visits in Leipzig and Kaiserslautern.
\begin{quote}
  Flat XML-like syntax for data, Meta information stored in the same format,
  managed with elaborated Perl tools, set up a CVS repository.

  Data from Polynomial System Solving and Geometry Theorem Proving, Test
  computations at UMS Medicis, with main focus on Polynomial System Solving.

  Prototype was presented at the Meeting of the Fachgruppe Computeralgebra,
  Kaiserslautern, February 2000.
\end{quote}

End 2000: O. Bachmann left the project (and science, and Kaisers\-lautern) for
a new job.

\end{slide}
\begin{slide}
\section{Some History (3)}\small

Main focus moved to Geometry Theorem Proving. 
\begin{quote}
  M. Witte (Leipzig) digitized a great part of the 512 geometry theorems [Chou
  88]

  Benchmark computations with the GeoProver package of H.-G. Gr\"abe.

  Talks at RWCA-02, ADG-02 and also in the CA-Rundbrief 
\end{quote}
Declining interest to really push the project during 2002-2005.
\begin{quote}
  Main focus 1 (2004/05): Move the format of data storage to a truly XML-based
  design and have the META information encoded as XSchema.

  Main focus 2 (2005/06): Incorporate new concepts of OWL ontology design. 
\end{quote}
Relaunch within the Special Semester on Gr\"obner Bases (Linz 2006).
\end{slide}
\begin{slide}
\section{Current State}

{\bf Data:} 350 Polynomial Systems, 297 Proof Schemes, 43 BIB items, 16
Testsets, 8 GAlgebra examples

{\bf Tools:} Shift to collecting scripts as best practice examples

{\bf Community Tools:} Mailing list with archive, CVS repository, domain
www.SymbolicData.org, running on behalf of the CA-Fachgruppe on a server at
the GI Bonn, release bundle of data for download from the Web site

{\bf Design:} All data translated to true XML, first experiments with OWL (to
be explained now in more detail)

\end{slide}
\begin{slide}
\section{Design of the SymbolicData Data Collection}

Organized as a {\bf Knowledge base} using the spirit of Web Ontology concepts
as proposed in the W3C Recommendation for the {\bf OWL Web Ontology Language}.

The data is divided into two parts, the XMLResources and the OWLResources.

{\bf XMLResources}: Smallest indivisible units of information; XMLType
described by a XSchema; stored locally to the SymbolicData distribution or
(huge items) globally in locations all over the world

\begin{verbatim}
  <XML XMLType="IntegerPolynomialSystem"
  url="sdxml:INTPS/ZeroDim.example_61.xml"/>
\end{verbatim}
\end{slide}
\begin{slide}
{\bf OWLResources}: Store information describing the XMLResources and also
relational information about them according to OWL design principles.

We use {\bf Protege} (a Java based OWL tool with a big community) to develop
and maintain a common ontology {\bf Ontology.owl}.

For the moment, OWL classes are translated to XSchema and OWL individuals are
stored as valid XML-files. 

Individuals are identified according to their class and id. 

\begin{verbatim}
<OWL xref="ZeroDim.example_7" class="INTPSAnnotation"/> 
\end{verbatim}

\end{slide}

\end{document}
