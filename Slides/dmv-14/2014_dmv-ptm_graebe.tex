%% Note that the length of the abstract should not exceed one page
\documentclass{dmv-ptm}

%% This is a place for LaTeX packages you need
\usepackage{url}

%% This is a place for your own commands

\newcommand{\SD}{{\sc Symbolic\-Data}}

%%
\title{The {\SD} Project -- towards a Computer Algebra Social Network}
\author{Hans-Gert Gr\"abe}
\institution{Universt\"at Leipzig}
\country{Germany}
\email{graebe@informatik.uni-leipzig.de}

%% the following command is optional: if you don't need leave %

%\coauthor{The talk is based on joint work with Gert-Martin Greuel }

%% Enter the session title - you can find it on the web page
%% http://ptm-dmv.wmi.amu.edu.pl/Symposia-accepted.html

\session{Information and Communication in Mathematics}

\begin{document}

\maketitle

\begin{abstract} %% put your abstract below

Information and Communication in Mathematics is not only a matter of
propagating facts and information but also a social interrelation between
humans.  We discuss the prospects to support such social interrelations
technically by modern semantic approaches within the Computer Algebra (CA)
community.

This project idea is part of the {\SD} Project \cite{ref-02} that aims at
two main goals: 
\begin{itemize}
\item to unify efforts to collect digital data for profiling, testing and
  benchmarking Computer Algebra Software from various CA subcommunities
  together with concepts, tools and experience for their management both
  globally and also for special profiling, testing and benchmarking purposes
  at a local site and
\item to promote a network of repositories of digital data and related
  information from different areas of Computer Algebra.
\end{itemize}
In the talk we show how far such an approach can be extended to disseminate
also other valuable information about, e.g., upcoming conferences, projects,
working groups or publications in a semantic aware way within such a
scientific community.

%% References are optional, follow the format
\begin{thebibliography}{22}
\bibitem{ref-01} H.-G. Gr\"abe, A. Nareike, S. Johanning, \emph{The
  SymbolicData Project -- Towards a Computer Algebra Social Network}, in
  \emph{Workshop and Work in Progress Papers at CICM 2014}, CEUR-WS.org
  vol. 1186 (2014), \url{http://ceur-ws.org/Vol-1186/#paper-21}
\bibitem{ref-02} The {\SD} Project. \url{http://symbolicdata.org} 
\end{thebibliography}

\end{abstract}
\end{document}

