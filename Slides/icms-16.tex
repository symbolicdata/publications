\documentclass{beamer}
\usepackage[english]{babel}
% Designelemente
\usetheme{Hannover}
\beamertemplatenavigationsymbolsempty

\newenvironment{code}{\small\tt \begin{tabbing}
\hskip12pt\=\hskip12pt\=\hskip12pt\=\hskip12pt\=\hskip5cm\=\hskip5cm\=\kill}
{\end{tabbing}}

\title[Semantic-aware Fingerprints]{Semantic-aware Fingerprints\\ of Symbolic
  Research Data} 

\author{Hans-Gert Gr\"abe}

\institute[]{Leipzig University, Germany\\
\texttt{http://bis.informatik.uni-leipzig.de/HansGertGraebe}}

\date{ICMS-2016, Berlin, 2016-07-13}
\begin{document}
\begin{frame}
\titlepage
\end{frame}

\begin{frame}
\frametitle{Introductory Remarks}
\begin{itemize}
\item \emph{Information Services for Mathematics} addresses a more complex
  target compared to \emph{Mathematical Software}.
\item Mathematical software is only part of a whole \emph{infrastructure for
  mathematical research} that nowadays goes much beyond the classically hawked
  \emph{paper and pencil} or \emph{chalk and blackboard}.
\item The themes \emph{software, services, models, and data} point to at least
  four dimensions to enhance the mathematical research infrastructure in the
  era of ubiquitous computing and increasingly important digital
  interconnectedness within the Digital Universe (DU).
\end{itemize}
\end{frame}

\section{Real World and Digital Universe}
\begin{frame}
\frametitle{Real World (RW) and Digital Universe (DU)}
\begin{itemize}
\item[(D1)] RW: Real world \emph{tasks} -- \emph{Resources} in the RDF
  terminology.

  A \textbf{task} is a real world \emph{process} with a \emph{goal} triggered
  by \emph{interested} people.
\item[(D2)] DU: Associated task related \emph{descriptions}.  

  Requires \emph{pointers} to RW task details, i.e., \textbf{digital
    identities} (DI) as URIs (unique resource identifiers) in the most
  formalized way required to be processed by computers.
\item [(D3a)] DU: Many descriptions relate to the same RW task. 

  \emph{Problem:} Match related DIs in communication. 
\item [(D3b)] DU: Many descriptions relate different RW task. 

  \emph{Problem:} Express such relations as relations between DIs in the DU. 
\end{itemize}
\end{frame}
\begin{frame}
\frametitle{Real World and Digital Universe}

RDF -- the \emph{Resource Description Framework} -- with its models, standards,
protocols and web architecture is nowadays an established standard to process
such problems on a technical level. Together with the \emph{Open Culture
  Paradigm} it is the basis for the ever growing \emph{Linked Open Data
  Cloud} (LOD). 

Links:
\begin{itemize}\itemsep0pt
\item \url{https://www.w3.org/RDF/}
\item \url{http://www.w3.org/standards/techs/rdf}
\item \url{http://lod-cloud.net/}
\end{itemize}

\begin{itemize}
\item[(D4)] RW: The digitally supported social communication processes about
  relations between DIs in the DU should have \emph{impact on the RW
    performance of coupled tasks} within a cooperative environment.
\end{itemize}
\end{frame}

\section{Information Services for Mathematics}
\begin{frame}
\frametitle{Information Services for Mathematics}
Information Services for Mathematics have to address at least four dimensions:
\medskip
\begin{itemize}
\item[(I1)] \emph{Research Data} -- papers, conference announcements, mailing
  lists, web sites {\ldots} (data streams).\vskip1em
  \begin{small}
    Although it seems to be completely within the DU, it is a good advice to
    differentiate between \emph{resources} (D1) -- as rather a part of the RW
    -- and \emph{resource descriptions} (D2). \vskip1em

    Resource descriptions are required to \emph{structure} resources for search
    and filter processes (D3b). RDF is best suited for such a task.
  \end{small}
\end{itemize}
\end{frame}
\begin{frame}
\frametitle{Information Services for Mathematics}
\begin{itemize}
\item[(I2)] \emph{Research Input Data} -- well curated publicly available data
  stocks with data relevant for a whole research community (data stores).
  \vskip1em
  \begin{small}
    This is a new phenomenon within the upcoming DU with great impact on
    research problems, research methods and research paradigms, in particular
    within the \emph{digital humanities}. \vskip1em

    Important for the coherence of research questions addressed by the
    community and thus for the formation of a specific research community
    itself around its central research problems.
  \end{small}
\end{itemize}
\end{frame}
\begin{frame}
\frametitle{Information Services for Mathematics}
\begin{itemize}
\item[(I3)] \emph{Mathematical Software} -- written to run computer
  simulations.  \vskip1em
  \begin{scriptsize}
    If software is not used in such a way it is of less academic interest.
    Moreover, computer simulations often require the interplay of several
    \emph{scientific packages} bundled within an \emph{application}, hence we
    propose to use \emph{computer simulation} as the broader notion.  \vskip1em

    Note that computer simulation (as a special kind of ``experiment'') relates
    to an epistemological dimension not covered by (D1--D3) above. \vskip1em

    The public availability of newly developed \emph{simulation methods,
      procedures and techniques} is relevant for the traceability of the
    proposed scientific approaches and increasingly accompanies classical forms
    of description of scientific advancement by academic papers. \vskip1em
  \end{scriptsize}
\end{itemize}
\end{frame}
\begin{frame}
\frametitle{Information Services for Mathematics}
\begin{itemize}
\item[(I4)] Public available \emph{output data} -- important for the
  independent reproduction of results and thus of essential importance for the
  process of academic quality assurance.  \vskip1em
  \begin{small}
    Output data is the starting point for new research questions and thus
    output data mutates to input data.\vskip1em

    In most of the cases such a mutation is mediated by a community-internal
    interpersonal transformation process that transforms the often large output
    data (or a whole bundle of such data) into (one or several) more compact
    input data adapted to the new research question(s).\vskip1em
  \end{small}
\end{itemize}
\end{frame}
\begin{frame}
\frametitle{Information Services for Mathematics}

Efforts to secure a research infrastructure for mathematical data at large at
the level (I2) or even (I4) are lost in the brushwood of everlasting (for at
least a decade) debates about reliable formal but semantically expressive
formats as MathML or OpenMath for data resulting from calculi, that are already
highly formalized -- at least at an informal level -- by the internal nature of
the research topics themselves.  \vskip1em

The situation reminds the Tower of Babel Project, since subcommunities are
digitally already well established, developed their own formalizations for
their own research data at (I2) level and apply such formalizations very
successful within their intracommunity communication processes.
\end{frame}

\begin{frame}\frametitle{The SymbolicData Project}
The SymbolicData Project 
\begin{itemize}
\item[(S1)] is an inter-community project with roots in the activities of
  different Computer Algebra Communities to develop concepts and tools for
  profiling, testing and benchmarking Computer Algebra Software (CAS) -- level
  (I2),
\item[(S2)] aims at interlinking these and other scientific activities between
  different subcommunities of the CA community using modern Semantic Web
  concepts -- tasks (D3a) and (D3b) and
\item[(S3)] during the last years concentrated efforts to set up the technical
  basis for a \emph{CA Social Network infrastructure} within the Linked Open
  Data Cloud -- level (I1).
\end{itemize}
\end{frame}

\section{Fingerprints}
\begin{frame}\frametitle{Research Data and Metadata}
  \begin{itemize}
  \item For management, search and filter functionality research data is
    usually enriched with \emph{metadata} that collect important relevant
    information of the individual data records in a compact manner.\vskip.5em

    We denote such metadata for an individual data record as its
    \emph{fingerprint}.
  \item Similar to a \emph{hash function} a fingerprint function computes a
    compact metadata record (\emph{resource description} in the RDF
    terminology) to each individual data record (\emph{resource} in the RDF
    terminology).
  \item As with a hash function one can use the fingerprints to distinguish
    different data records within the given collection and to match new records
    with given ones.
  \end{itemize}
\end{frame}

\begin{frame}\frametitle{Research Data and Metadata}
  \begin{itemize}
  \item There is an essential difference between (classical) hash functions and
    well designed fingerprints: fingerprint functions exploit not only the
    textual representation of the data record as meaningless syntactical
    character string but convey \emph{semantically important information} or
    even compute such information from the string representation.
  \item Fingerprints are in this sense \emph{semantic-aware} and can even be
    designed in such a way that they map ambiguities in the textual
    representation of records (e.g., polynomial systems given in different
    polynomial orders and even in different variable sets) to \emph{semantic
      invariants}.
  \item The design of appropriate fingerprint signatures is an important
    \emph{intracommunity} activity to structure its own research data
    collections.
  \end{itemize}
\end{frame}

\begin{frame}\frametitle{Research Data and Metadata}
  \begin{itemize}
  \item Such fingerprint signatures are also very useful for the
    \emph{intercommunity} usage of research data collections, since they allow
    to navigate within the (foreign) research data collection without
    presupposing the full knowledge of the ``general nonsense'' of the target
    research domain, i.e., the informal background knowledge required freely to
    navigate as scientist in that domain.
  \item Hence well designed fingerprint signatures are to be considered also as
    a first class service of a special research community to a wider audience
    to inspect their research data collections without using the
    community-internal tools to access the resources themselves.
  \end{itemize}
\end{frame}

\begin{frame}\frametitle{Example}
Fingerprints for ideals in polynomial rings:
\begin{code}
  <http://symbolicdata.org/Data/Ideal/Gerdt-93a> ...\+\\
sd:hasDegreeList "2,3,4" ;\\
sd:hasLengthsList "3,3,4" ;\\
sd:relatedPolynomialSystem\\\>\> 
<http://symbolicdata.org/Data/IntPS/Gerdt-93a> ;\\
a sd:Ideal .
\end{code}
Polynomial systems and ideals -- the semantic complexity of a seemingly easy
question.\vskip1em

Fingerprints are usually stored together with the resource itself (as for
polytopes in \emph{polymake}, contribution by A. Paffenholz, and for integer
programming examples in \emph{normaliz}, contribution by Tim R\"omer) or within
a database (transitive groups by Kl\"uners and Malle).   
\end{frame}

\begin{frame}\frametitle{Fingerprints and the LOD}
For navigational and filter tasks it is necessary to extract fingerprints into
a common database.  Best practice uses an RDF representation to integrate that
information into the Linked Open Data Cloud and to offer SPARQL querying the
metadata. \vskip1em

Example SPARQL query for polynomial systems:
\begin{code}\small
PREFIX sd: <http://symbolicdata.org/Data/Model\#>\\
select ?a\\
from <http://symbolicdata.org/Data/PolynomialSystems/>\\
where \{\\
?a a sd:Ideal .\\
?a sd:hasDegreeList "2,3,4" . \}
\end{code}
\end{frame}

\begin{frame}\frametitle{Fingerprints and the LOD}
During the last years the SymbolicData Project concentrated on collecting such
fingerprint information from different CA sources -- the research data are
maintained by the subcommunity, the fingerprints allow for easy navigation
within that data.\vskip1em

\emph{Different approach:} Direct integration of the resources themselves into
a general software system as, e.g., SageMath. Drawbacks: Restricted search
functionality, no direct integration into the Linked Open Data Cloud possible.

\end{frame}

\section{The SymbolicData Project}
\begin{frame}\frametitle{The SymbolicData Project}
The SymbolicData provides \bigskip

\textbf{Data and Fingerprints:}
\begin{itemize}
\item Polynomial Systems Solving
\item Geometry Theorem Proving
\item Free Algebras
\item G-Algebras
\end{itemize}\bigskip
\textbf{Fingerprints:}
\begin{itemize}
\item Test Sets from Integer Programming (T. R\"omer)
\item Fano Polytopes (A. Paffenholz)
\item Birkhoff Polytopes (A. Paffenholz)
\item Transitive Groups (J. Kl\"uners, G. Malle)
\end{itemize}
\end{frame}

\begin{frame}\frametitle{The SymbolicData Project}
\textbf{Tools:}\bigskip

SDEval Package (Albert Heinle)
\begin{itemize}\small
\item Aim: Set up, run, log, monitor standardized Computations on SD data
  series in a reliable way 
\item Technology: Python standalone on top of the OS
\item \url{http://wiki.symbolicdata.org/SDEval}
\end{itemize}
SDSage Package (Andreas Nareike)
\begin{itemize}\small
\item Aim: Call the new Polynomial Systems format from SageMath 
\item Technology: SageMath Python Package
\item \url{http://wiki.symbolicdata.org/PolynomialSystems.Sage}
\end{itemize}\vfill
Tools and data are designed to be used both on a local site for special testing
and profiling purposes to manage a central repository at
  \texttt{http://www.symbolicdata.org}
\end{frame}

\begin{frame}\frametitle{SymbolicData Infrastructure}
\begin{itemize}
\item Github organizational account \url{http://github.com/symbolicdata}
\item A project wiki at \url{http://symbolicdata.org}
\item A mailing list
\item Web access to the XML resources
\item A centrally operated Virtuoso based RDF data store for meta data
\item Organized along Linked Data Principles
\item Regular dumps of RDF data in Turtle format
\item A SPARQL endpoints to query the data
\item Advise for local installation of tools and data based on Virtuoso and a
  local Apache Web server
\end{itemize}
\end{frame}

\section{Towards a CA Social Network}
\begin{frame}\frametitle{Towards a CA Social Network}

During the last years we consolidated the following infrastructure:
\begin{itemize}
\item We collect, update, and serve relevant information about CA people,
  upcoming and past CA conferences through our central RDF store.
\item We set up local CASN nodes at
  \begin{itemize}
  \item \url{http://symbolicdata.org/rdf}
  \item \url{http://fachgruppe-computeralgebra.org/rdf}
  \end{itemize}
  with more information in Linked Open Data format as best practices.
\item We operate \url{http://symbolicdata.org/info} as example how to integrate
  such information into local web pages, see also
  \begin{center}\small
    \url{http://fachgruppe-computeralgebra.org/symbolicdata}
  \end{center}
\end{itemize}

\end{frame}

\section{Links}
\begin{frame}\frametitle{Links}
\begin{itemize}
\item \texttt{http://wiki.symbolicdata.org} -- the SD Wiki
\item \texttt{http://symbolicdata.org/XMLResources} -- the SD XML Resources
\item \texttt{http://symbolicdata.org/RDFData} -- the SD RDF Data Turtle Files
\item \texttt{http://symbolicdata.org/Data} -- the SD OntoWiki view on the
  RDF data, including the CASN data
\item \texttt{https://github.com/symbolicdata} -- the SD organizational github
  account with several git repos
\end{itemize}
\end{frame}

\end{document}


\section{RDF -- Basic Concepts}
\begin{frame}\frametitle{RDF and Linked Data Principles}
\begin{itemize}
\item RDF = Resource Description Framework
\begin{itemize}\small
\item Main idea: Store pieces of information in a unified way as triples,
  use standard tools to manage these data.
\end{itemize}
\item \emph{Resources:} URI, HTTP access
\begin{itemize}\small
\item URI = Unique Resource Identifier
\item Access to worldwide distributed data in a unified way
\end{itemize}
\item \emph{Resource Descriptions:} Deliver a valuable piece of information in
  structured RDF format, that can be combined with other pieces of information
  from other sources into new RDF sentences.
\item Run \emph{RDF Triple Stores} as part of a worldwide distributed data
  storage infrastructure
\item (Federated) Query Language SPARQL
\item Run \emph{SPARQL Endpoints} on RDF triple stores
\end{itemize}
\end{frame}
