\documentclass[a4paper,11pt]{article}
\usepackage{CAR}

\def\SD{\textsc{Symbolic\-Data}}
\def\eg{e.g.\ }
\def\ie{i.e.\ }

\begin{document}

\begin{center}
  {\Large\bf The {\SD} Project -- Maturing the Computer Algebra Social Network
  Perspective} \vskip1em

H.-G.\ Gr\"abe (Universit\"at Leipzig)\\[6pt]
\texttt{graebe@informatik.uni-leipzig.de} 
\end{center}
To appear in \emph{Computeralgebra Rundbrief} 59, October 2016.  

\begin{multicols}{2}
\noindent

%%%%%%%%%%%%%%%%%%%%%%%%%%%%%%%%%%%%%%%%%%%%%%%%%%%%%%%%%%%%%%%%%%%%%%%%%%%%%%%%


\Ueberschrift{Introduction}{intro}

In \cite{car-55,cicm-14,icms-16,aca-16} we described the roots and goals of
the {\SD} Project and also basic Linked Data and RDF principles in more
detail.  In this note we report about advances of the project during the last
year.  Our main efforts were directed to strengthen and consolidate the
Computer Algebra Social Network (CASN) part.

In March 2016 we released {\SD} version 3.1 with new data from different CA
subcommunities, a reorganized git repo structure, close integration of the
CASN part into the main project and a couple of examples based on the
\emph{EasyRDF} PHP library and the Bootstrap web framework to show how to set
up web presentations of the data in a very simple way.

The focus of the {\SD} Project moved from a project mainly centered around
activities of the German CA Fachgruppe to a broader international scope.  This
opening started with a presentation of the project in the Work in Progress
section of the 2014 CICM conference and was discussed with the SIGSAM chair at
the ACA 2015 conference in Kalamata, where {\emph{Hans-Gert Gr\"abe}} gave a
presentation and also a hands on session on {\SD}, see our publications and
presentations overview\footnote{
  \url{http://wiki.symbolicdata.org/Publications}} for more details.  The
contacts were deepened by \emph{Albert Heinle} and \emph{Victor Levandovskyy}
with a paper \cite{heinle-15} about the SDEval framework in the \emph{ACM
  Communications in Computer Algebra}.

In 2016 we had the opportunity to present and discuss various aspects of {\SD}
and more general perspectives of a digital research infrastructure for the CA
community at the session \emph{Information services for mathematics: software,
  services, models, and data} at ICMS-2016 in Berlin (organized by Wolfram
Sperber, FIZ Karlsruhe, and Michael Kohlhase, Jacobs University Bremen) and at
the session \emph{Information services for mathematical software, models, and
  research data} at ACA-2016 in Kassel (organized by Hans-Gert Gr\"abe,
University of Leipzig, Albert Heinle, University of Waterloo, and Wolfram
Sperber, FIZ Karlsruhe).  A more detailed report on these discussions will
appear in a forthcoming publication.

\Ueberschrift{Enlarging the {\SD} Database}{esd}

With {\SD} 3.1 we consolidated the integration of several data sources into
our main data and metadata collection that were available so far only in a
draft version.

With that integration the conceptual design of the {\SD} database changed from
a data store to a metadata store -- the new data collections provide (only)
metadata information about the core research data that is hosted and
maintained in a separate remote data store by a CA subcommunity. The CA
subcommunity provides expertise to maintain the research data in a
semantically correct way and provides metadata in ``raw form''.  The {\SD}
team collects that metadata, transforms it into RDF and prepares it for search
and filter processing.

Note that such a design change was enabled by the consistent transformation of
the metadata collected so far along RDF principles during preparation of {\SD}
version~3.  RDF requires a strong distinction between data and metadata
whereas the data structure design of CA research data usually handles metadata
as ``data extension'' and stores it together with the primary data in a common
file.

Our new conceptual approach along that lines supports the formation of an
\emph{interlinked distributed research data infrastructure} within the CA
community and between its severe subcommunities. We realized that in many
cases CA subcommunities (in particular subcommunities developing specialized
research CA software systems) have well established research data
infrastructures with no need to be duplicated, but interlinking these
infrastructures is yet a challenge.

In the following subsections we describe the advances at the ``data frontier''
in more detail.
 
\Ueberschriftu{Fingerprints for Polytopes and Groups}

In \cite{cicm-14} we announced draft versions of RDF based resource
descriptions (i.e., metadata or \emph{fingerprints}, see \cite{icms-16}) of
Fano and Birkhoff Polytopes collected by Andreas Paffenholz as part of the
\emph{polymake} data store and of Transitive Groups from the \emph{Database
  for Number Fields} collected and stored in a similar way by J\"urgen
Kl\"uners and Gunter Malle.

The draft versions were extracted from the primary data sources and
transformed into RDF based \emph{fingerprints} by \emph{Andreas Nareike} in
2013.  The metadata was provided as part of the primary data in different
formats within the respective remote collections. The derived metadata is now
integrated into the {\SD} main database. Each such metadata record contains a
link to the corresponding data record within the remotely maintained research
data store provided by the respective CA subcommunity.

As main advantage such separated metadata can be queried in a common, uniform
and well established way using the SPARQL query language -- another W3C web
standard with many tools and concepts developed for a performant management of
big data given in the RDF semantic web format and for integration with other
applications.

SPARQL plays a similar role for querying the worldwide distributed and
interlinked semantic web data store as SQL plays for querying local databases.
Since this is in the first plan a technical question we don't go into details
here but refer to our wiki\footnote{
  \url{http://wiki.symbolicdata.org/MoreQueries}} for more background and some
example queries.  Using that technology one can navigate within such data,
prestructure it for efficient search or identify a given example within the
database.

\Ueberschriftu{Transforming Test Sets into the Normaliz Format}

In a similar way the data on Integer Programming -- the {\SD} Test Sets
collection were transformed and enlarged.

The old {\SD} Test Sets collection was compiled by \emph{Raymond Hemmecke}
several years ago along the former {\SD} rules -- develop a data model, an
XML-binding for data storage to represent this special data type, and an RDF
ontology for metadata (fingerprints and maintenance information).

The former exclusive usage of XML-bindings for data representations was
inspired by the success and wide usage of XML as a unified way to represent
data in other application areas at that time, in particular influenced by the
upcoming MathML standard.  Meanwhile XML is much less prominent for exchange
practices of structured data, and within the redesign of {\SD} we decided to
accept and use data also in other formats.

Such a decision also was inspired by the observation that specialized CA
software comes with well a defined input data format, and within
subcommunities using a common software the data are stored and exchanged in
just that format.  Hence for such a subcommunity (as, e.g., the Polytope
subcommunity around \emph{polymake} or the Integer Programming subcommunity
around \emph{Normaliz}) there is no need to develop another standard for data
exchange -- such a standard would even hardly be accepted.  Note that the
situation is different within the Polynomial Systems Solving subcommunity
since there exist several major software systems with different input formats,
as \emph{Singular}, \emph{Macaulay2}, \emph{GB} or \emph{CoCoA} -- to name
only the most important ones.

For the new Test Sets collection we use and store data in the \emph{Normaliz}
exchange format and thus prepared the data in a similar way as for polytopes
and transitive groups. This work was done by \emph{Tim R\"omer} who
transformed also the ``legacy'' Test Sets into the new format.  All stuff from
the old format was cleaned up from the repos and the web pages were adjusted.

\Ueberschrift{Towards a Computer Algebra Social Network}{casn}

\Ueberschriftu{Motivation}

A powerful digital research infrastructure all parties want to have but are
rarely enough willing or able to invest in it.  It is a complex social
challenge to organize active goal-oriented cooperation in such an area outside
the scientific reputation process, and we learned over the years not only to
concentrate on the collection of scientific \emph{data} but also on structured
and semantically enriched information about the scientific and social
\emph{processes} to produce this data.

Several years ago the {\SD} Project extended its scope to analyze and support
the exchange of such information in a structured way.  Our vision is a
distributed and tool based network of semantic aware nodes corresponding to
the (small and big) nodes of the real CA research network.  Such a
\emph{Computer Algebra Social Network} (CASN) should be a semantically
enriched digital infrastructure for a social network of scientific research
and scientific researchers within the Computer Algebra community and its
severe subcommunities similar to other social networks as, e.g., Facebook.

Note that the starting point for such a CASN is in at least two ways different
from the Facebook starting point. First, there is no Marc Zuckerberg nor such
an amount of money to push the project. Second, there is already a digital
``CA memory'' -- a huge number of very loosely related web pages about
conferences, meetings, working groups, projects, private and public
repositories, private and public mailing lists etc. The CASN design has to
take such a diversity into account and develop a decentralized solution based
on modern semantic technologies that increases the awareness of the different
parts of that already existing ``CA network'' and supports the
\emph{exploration} of that network to get useful deep results in an easy
manner.

\Ueberschriftu{CASN Nodes}

For a proper CASN design it is essential to exploit the potential of concepts,
tools and standards of the fast growing distributed Linked Open Data (LOD)
Cloud\footnote{ \url{http://lod-cloud.net}}.  Pascal Hitzler emphasizes the
importance of such a coordinated conceptual approach to set up an
interoperably interlinked digital universe, since ``with the omnipresence and
availability of data from different times, locations, perspectives, topics,
cultures, resolutions, qualities, and so forth, \emph{exploration} becomes an
additional (4th) paradigm of science'' \cite{hitzler-13}.

As a first step towards a digital network within the CA community capable to
\emph{explore} social and scientific relations
\begin{itemize}
\item we operate the RDF based {\SD} main data store together with its SPARQL
  endpoint \cite{sdsparql} to query centrally maintained data and
\item propose to convert other nodes of the ``CA memory'' into CASN nodes that
  provide part of their data in structured RDF format.
\end{itemize}
RDF principles neither require such nodes to be uniformly structured nor
running on big web resources. LOD sources are self-explanatory by design and
its structure can be explored with appropriate RDF tools by interested third
parties at run time to prepare to fetch the information in a structured way.
Hence efforts to present and explore data within such a CASN network can be
shared in a wide scope between data providers and data consumers.

In a first version such a node can be even only a directory with valuable RDF
files publicly accessible in the web as provided by the CASN sample
node\footnote{ \url{http://symbolicdata.org/rdf/}} of the {\SD} Project. As
proof of concept in the subdirectory \texttt{Conferences} we provide
prototypically detailed information about five CA conferences using the
(meanwhile outdated) \emph{Semantic Web Conference Ontology}\footnote{
  \url{http://data.semanticweb.org/ns/swc/swc_2009-05-09.html}}.

The CASN node of the German CA Fachgruppe\footnote{
  \url{http://www.fachgruppe-computeralgebra.de/rdf/}} is designed along a
more advanced concept. During the revision of concepts towards {\SD} 3.1 we
consequently redesigned this data to form a proper CASN node with publicly
accessible but locally maintained RDF sources of (almost) all structured
information displayed on the web site of the German CA Fachgruppe.  This
information is explored by a special plugin and rendered in its Wordpress
based web presentation\footnote{
  \url{http://www.fachgruppe-computeralgebra.de/}}.  Hence one can explore
both the ``pure'' information in standard RDF notation to embed it into third
party web workflows as \emph{interlinked data} and in the ``old fashion'' as
\emph{hyperlinked text}.  Note that the technical realization is unpretentious
-- the RDF data is stored as plain files in RDF/XML format in the CASN node
and the plugin uses the \emph{EasyRDF} PHP library and the Wordpress Shortcode
mechanism for rendering.  No advanced technique as RDF store or SPARQL
endpoint has to be set up.  The code is mirrored as best practice example in
our \emph{maintenance} git repo.

\Ueberschriftu{CA Conferences}

As another service within the CASN we maintain a list of \emph{Upcoming
  Conferences}.  The data about conferences is extracted from several sources,
transformed into RDF format and delivered by our main SPARQL endpoint
\cite{sdsparql}.  This information is used by the German CA Fachgruppe at one
hand to present an online list of upcoming conferences and at the other to
generate the conference announcement section of the printed version of their
CA Rundbrief.  The RDF database contains more advanced information about
conferences as, e.g., submission deadlines or program committees.

We run this service in a draft version already for several years and compiled
from it a list of (at the moment 166) \emph{Past Conferences}. In summer 2016
this data was enhanced with additional data about past conferences supplied by
the SIGSAM web team and extended by a \emph{Conference Series} concept from
the SIGSAM collection.  The SIGSAM collection provides structured information
about such conference series (description and publication rules) in an
(almost) unstructured way that was transformed to structured RDF using
predicates \texttt{sd:description} and \texttt{sd:publicationRules}.  Not to
duplicate information without reason we use the standard predicate
\texttt{rdfs:seeAlso} to link with the corresponding part in the SIGSAM
conference series web page for additional information.

\Ueberschriftu{The {\SD} People Database}

The concept of URI (Unique Resource Identifier) as part of the RDF standard
provides a generic way to disambiguate people and artifacts.  More precisely,
each such URI considered as \emph{digital identity} is the entry point from
the real world to the digital universe, and any statement within the digital
universe can be followed and traced back using digital technology only up to
such (combinations of) URIs. URIs are bound to real world entities by more
complex socio-political and technical ``agreements''. To shape politically
such ``agreements'' is the real core of digital privacy.

The use of URIs provides an easy way to assign digital facts to special
digital identities and thus solve the \emph{disambiguation problem} -- a great
problem in the text oriented ``hyperlinked universe'' that required powerful
text mining so far.  One of the great challenges to academic content providers
within the transformation of their digital universes is \emph{author
  disambiguation}. Such disambiguation is required to, e.g., assign URIs of
publications to the correct author URIs.  Most of the academic content
providers come up with own solutions for their own universe, i.e., for the
provider's internal data collection that counts as its main ``capital''.
Interoperability between providers remains a great challenge since it requires
to interlink data sources that are very private from a business point of view.
Whereas this Gordian knot is hard to cut from a provider's position, a
comparatively small scientific community could solve that interoperability
challenge by a common effort -- develop its own People Database, i.e., its own
URI system for people and provide dictionaries to the part of the URI systems
of the different providers relevant to their academic scope.

This is the goal of the {\SD} People Database for the CA community.  As one of
the benefits of such a disambiguation one can track reputation and merits more
precisely querying the whole {\SD} database or even interlinking it with other
RDF based sources within the Linked Open Data Cloud. Moreover people within
the CASN can systematically provide and update information about their own
scientific activities.

Currently the {\SD} People Database contains more than 1200 entries, i.e.,
digital identities of scientists that are active in the area of Computer
Algebra.  These URIs were mainly extracted from program committee lists of
different conferences or (in a restricted scope) from lists of authors of
accepted papers.

As standard information we provide personal information as instance of
\texttt{foaf:Person} with (a subset of) keys \texttt{foaf:name},
\texttt{foaf:homepage} and \texttt{sd:affiliation} (a literal). Due to privacy
reasons we do not provide \texttt{foaf:mbox} (email) values.  This list is
steadily enlarged and used as URI reference for reports about different
activities (invited speakers, conference organizers, program committees etc.)
in other parts of our CASN database.

As proof of concept we aligned our URIs in a common task with the
``Zentralblatt'' with their author disambiguation system and produced more
than 300 \texttt{sd:hasZBMathAuthorID} matches. This work was done in 2014 on
an early version of the {\SD} People Database and can be queried from our RDF
store, too.  In a near future we plan to update that alignment with
``Zentralblatt''.  The concept can easily be extended to other content
providers (in particular to the ACM people database or the MathSciNet author
disambiguation system) that are interested in such a cooperation.

\Ueberschriftu{The CA Dissertations Project}

The CA Rundbrief of the German CA Fachgruppe maintains a section with reports
about dissertations in Computer Algebra finished in working groups within the
Fachgruppe.  We made the metadata available also in RDF within the CASN node
of the Fachgruppe and display it at their web site.  Within the discussions
with SIGSAM in summer 2016 we realized that there is a large data pool of
similar information collected by SIGSAM for years that could be integrated
into a common database of dissertations in Computer Algebra. For the moment we
moved the existing RDF data about dissertations to the {\SD} main data store
and aligned the presentation in the web site of the German CA Fachgruppe
accordingly.

\Ueberschriftu{The CA Systems Project}

In summer 2016 we also intensively discussed perspectives of the swMATH
project \cite{swmath}. In particular we considered ways to popularize it to a
larger audience (within the CA community) and discussed to what extend RDF
principles and LOD alignment could support such a popularization.  We agreed
that it would be helpful to represent a core part of the swMATH metadata in
RDF, provide URIs with a consistent naming scheme, and publish this data as
Linked Open Dataset to achieve better visibility within the semantic web
community.  Such a metadata extraction also makes the alignment with other
overviews on CA systems as, e.g., maintained by SIGSAM, much easier.

A first prototypical draft version of such an RDF based overview on \emph{CA
  systems} extracted from the swMATH database was compiled during our
discussions in summer 2016 and is available from our RDF store.  We also set up
a prototypical view on that data within the {\SD} info pages\footnote{
  \url{http://wiki.symbolicdata.org/info}}.

Additionally, we discussed whether the swMATH data model has to be redesigned
better to reflect subtleties as the relation between CA systems and CA
packages or different versions of the same system.  All these questions
require much deeper analysis.  Since RDF can be used in a consistent way to
express modeling aspects a Linked Open Dataset as just described could support
also such a discussion.

\Ueberschrift{Advances in the {\SD} Infrastructure}{ai}

In October 2015 we converted our main git repo\footnote{
  \url{https://github.com/symbolicdata}} to an organizational account. With
{\SD} version 3.1 we reorganized our git repo structure and set up several new
repos with different maintenance rules.
\begin{itemize}
\item \emph{data} -- the data repo with a single master branch mainly to backup
  recent versions of data, no versioning,
\item \emph{code} -- code directory with master and develop branches, under
  versioning,
\item \emph{maintenance} -- code chunks from different tasks and demos how to
  work with RDF based data, no versioning,
\item \emph{publications} -- as a backup store of the {\LaTeX} sources of {\SD}
  publications, only master branch, no versioning,
\item \emph{web} -- as an extended backup store of the {\SD} web site that
  provides useful code to learn how RDF based data can be presented in the
  web.
\end{itemize}

The main development is coordinated by the \emph{{\SD} Core Team} (Hans-Gert
Gr\"abe, Ralf Hemmecke, Albert Heinle) with direct access to the
organizational account.

The repos \emph{maintenance} and \emph{web} are intended to show best practice
code for using the RDF based data of the {\SD} project.  In particular, the
\emph{maintenance} repo contains a mirror of the Wordpress plugin code used by
the German CA Fachgruppe and the transformation code developed by Andreas
Nareike in 2013 for polytopes and groups databases.  To use the code you may
fork the repo, but there is almost no reason to pull code back. If you have a
valuable contribution please contact the Core Team to discuss how that
contribution can be added to the project.

The repo \emph{data} is mainly for backup purposes. If you plan to add
valuable data to the project please contact the Core Team to discuss how that
contribution can be added. We provide help to put the data in an appropriate
Linked Open Data format.

The repo \emph{publications} is used mainly for reference and not intended for
public additions. We provide {\LaTeX} sources of our papers and slides and
also information about the review processes of our work since reviews provide
many valuable suggestions for the further development of our project.  The
repo \emph{code} contains several coding subprojects concerning {\SD} tools
for various purposes. 

%%%%%%%%%%%%%%%%%%%%%%%%%%%%%%%%%%%%%%%%%%%%%%%%%%%%%%%%%%%%%%%%%%%%%%%%%%%%%%%%

% Literaturverzeichnis
\begin{thebibliography}{1}
\itemsep=0cm plus 0pt minus 0pt

\bibitem{car-55} Hans-Gert Gr\"abe, Simon Johanning, Andreas Nareike.
  \newblock The {\SD} Project -- from Data Store to Computer Algebra Social
  Network. \newblock \emph{Computeralgebra-Rundbrief}, 55:22--26, 2014.
\bibitem{cicm-14} Hans-Gert Gr\"abe, Simon Johanning, Andreas Nareike.
  \newblock The {\SD} Project -- Towards a Computer Algebra Social Network.
  \newblock \emph{Workshop and Work in Progress Papers at CICM 2014}.
  CEUR-WS.org, vol. 1186, 2014.
\bibitem{icms-16} Hans-Gert Gr\"abe. \newblock Semantic-aware Fingerprints of
  Symbolic Research Data. \newblock In Gert-Martin Greuel, Thorsten Koch, Peter
  Paule, Andrew Sommese (Eds.).  \emph{Mathematical Software -- ICMS 2016}.
  \newblock Volume 9725 of Lecture Notes in Computer Science, page 411--418,
  2016.  
\bibitem{aca-16} Hans-Gert Gr\"abe. \newblock The SymbolicData Project – a
  Community Driven Project for the CA Community.  \newblock Talk given at the
  ACA 2016 Session ``Information services for mathematical software, models,
  and research data.'' \newblock 
  \url{http://symbolicdata.org/Papers/aca-16.pdf}. [2016-09-11]
\bibitem{heinle-15} Albert Heinle, Viktor Levandovskyy.  \newblock The SDEval
  Benchmarking Toolkit.  \newblock \emph{ACM Communications in Computer
    Algebra}, 49.1:1--10, 2015.
\bibitem{hitzler-13} Pascal Hitzler, Krzysztof Janowicz.  \newblock Linked
  Data, Big Data, and the 4th Paradigm.  \newblock \emph{Semantic Web},
  4.3:233--235, 2013.
\bibitem{swmath} swMATH -- an Information Service for Mathematical Software.
  \newblock \url{http://swmath.org/about_contact}. [2016-09-16]
\bibitem{sdsparql} The {\SD} SPARQL Endpoint. \newblock 
   \url{http://symbolicdata.org:8890/sparql}. [2016-09-11]

\end{thebibliography}

%%%%%%%%%%%%%%%%%%%%%%%%%%%%%%%%%%%%%%%%%%%%%%%%%%%%%%%%%%%%%%%%%%%%%%%%%%%%%%%%

\end{multicols}

\end{document}
