\documentclass[a4paper,11pt]{article}
\usepackage{CAR}
%\usepackage{sigsam}
\usepackage[utf8]{inputenc}

\def\SD{\textsc{SymbolicData}}

\parindent0pt
\parskip3pt

\title{20 Years {\SD}}

\author{Hans-Gert Gr\"abe, Leipzig}
\date{Version of September 2, 2018}

\newcommand{\vorspann}{
\Aufsatz
% Title
{20 Years {\SD} Project}
% Short title for the table of contents
{20 Years {\SD} Project}
% Authors
{H.-G.\ Gr\"abe}
% Label of authors' names
{graebe}
% Names and adresses
{H.-G.\ Gr\"abe, Leipzig University}
% Images of authors
{graebe.jpg}
% E-Mails
{graebe@informatik.uni-leipzig.de}

}

\begin{document}
%\vorspann
\maketitle

\section{Introduction}

At the OSCAR
meeting\footnote{\url{https://www.mis.mpg.de/calendar/conferences/2017/oscar2017.html}.}
in Leipzig in December 2017 Bernd Sturmfels promoted a discussion ``What to do
with Big Old
Data?''\footnote{\url{https://www.mis.mpg.de/fileadmin/pdf/slides_oscar2017_3198.pdf}}.
He addressed ``a key problem in the development of an open source computer
algebra system -- the design of mechanisms and formats for dealing with big old
data'', i.e. ``the output of a mathematical computation that is much larger
than a few lines, and is intended for storage in a repository, or for further
processing by a different program'', and presented a list of 12 papers. In
these papers authors from different areas of computer mathematics describe
efforts, concepts and data stores to resolve for themselves or for a small
computer mathematics subcommunity such a problem.  As central goal each of
these projects tries to establish from the very scratch a research
infrastructure capable to present, access, inspect, exchange and maintain
structured research data in both a powerful and sustainable way.

A similar situation was discussed 20 yeas ago at the \emph{special session on
  Benchmarking}\footnote{\url{http://bwcloud-108-017.bwcloud.uni-mannheim.de/cafgbench.html}.}
at the 1998 ISSAC conference in Rostock.  Heinz Kredel presented in his famous
talk\footnote{\url{http://bwcloud-108-017.bwcloud.uni-mannheim.de/cabench/issac98/}.}
also 12 examples of web resources that were set up and maintained by different
people to keep track of the development concerning several of the challenges
discussed at that time (Wester's CAS test suite, von zur Gathen's polynomial
factorization challenge, Montgomery's number field sieve factorizations or the
SATLIB benchmark suite -- to name some of them). Most of the links provided by
Heinz Kredel are outdated for a long time, of course.  So, didn't the world
change during the past 20 years?

This paper presents a comprehensive survey of the most important motivations,
concepts, steps, efforts and practical achievements of the {\SD} Project to
contribute to the formation of a reliable and sustainable research
infrastructure in the area of Computer Algebra.  For a more detailed
explanation of different aspects of the motivations, concepts and achievements
of the {\SD} Project we refer to our publication list \cite{sdpub}.

We start with a short discussion about the importance of research data in
science in general, touch the distinction between core concerns and cross
cutting concerns in modern software engineering and discuss consequences of the
emerging role of semantic technologies for research infrastructures as a great
difference between the situation today and 20 years ago.

A cornerstone of such concepts is the increasing importance of metadata that
have to accompany research data to organize their maintainability,
searchability and exchangeability. Within {\SD} we developed the concept of
\emph{semantic-aware fingerprints} to emphasize that the definition of such
metadata is a process of social coordination between the ``suppliers'' and the
``customers'' of research data to agree upon a \emph{domain ontology} and not
merely a question of just technically presenting the data.

Our experience indicates that the importance to organize the formation of a
research infrastructure as a social process is underestimated.  Whereas in
small subcommunities of Computer Algebra research infrastructures usually grow
and develop in a ``natural way'', it requires great efforts to organize such a
social process on the level of inter-subcommunity communication within Computer
Algebra.  Such inter- and infradisciplinary engagement is little recognized as
beneficial for the scientific career of productive young researcher. We
developed the concept of a \emph{Computer Algebra Social Network} (CASN)
\cite{car-55} based on modern semantic technologies and maintained a
prototypical distributed implementation as showcase over several years.

As me, a great number of people involved with the development of research data
infrastructure during the last 20 years already retired or will retire during
the next years.  It is up to the next generation to take over the baton.
During the last years we spent efforts to consolidate and concentrate the {\SD}
activities on github. In particular we moved the wiki to github pages, stopped
to update the CASN data base and canceled our announcement mailing list as a
probably outdated way of communication.  It is the part of the next generation
to evaluate the {\SD} heritage and update the parts that are worth to be
continued.  

\section{Research Data in Science}

Research data play an important role within science in at least four
dimensions: 1) artifacts as problem sources (in particular digitized artifacts
in the humanities), 2) benchmark examples as well established challenges for
different problem classes, 3) raw output of research to be analyzed and
evaluated within the scientific community and 4) publications and other
consolidated scientific output as part of a common scientific \emph{social}
infrastructure.

Traditionally, the Computer Algebra community focuses on dimension 4, in
particular on algorithms, implementations and software.  The swMATH project
\cite{swmath} and several big national or EU funded projects as PoSSo
\cite{PoSSo}, FRISCO \cite{FRISCO}, the SPP 1489, OpenDreamKit \cite{odk} or
OSCAR \cite{oscar} aimed and aim at contributing to the formation of a common
Computer Algebra research infrastructure in that direction.

At the 1998 ISSAC conference in Rostock, at the end of the projects PoSSo and
FRISCO, dimension 2 and 3 started to play a more important role.  Gert-Martin
Greuel reported in his invited lecture \emph{Computeralgebra and Algebraic
  Geometry, Achievements and Perspectives} in particular about the progress in
the Gr\"obner business as one of the core algorithmic building blocks in
advanced polynomial systems solving and algebraic geometry.  In 1993 the FGLM
algorithm was published but there was rumor about new algorithmic ideas of a
young guy in Paris and in particular Greuel's Singular group was interested in
evaluating these ideas.  Unfortunately, after the end of the PoSSo project
there was no established reference for the famous \emph{PoSSo test suite} and
different printed versions with a number of misprints were in circulation.  It
started a disputation about the timings of examples, since it was neither clear
what version of the PoSSo examples was referenced nor what ``timing'' really
means (clock time, wall time, processor time etc.).  Heinz Kredel proposed to
start a \emph{Computer Algebra Benchmarking Initiative} and invited for a
\emph{special session on Benchmarking} that took place on August 15, 1998,
18:00 in room 110 in the main building of the Rostock University.  ``The
initiative will discuss, develop, define, collect all facets of this
challenging problem. It should analyse and develop test suites but also define
standard examples for the various topics of computer algebra where algorithm
and system developer can test their newly developed and improved
methods. Furthermore, all kinds of test examples should be collected and
consolidated. ''\footnote{\url{http://bwcloud-108-017.bwcloud.uni-mannheim.de/cafgbench.html}.}
Heinz Kredel presented a \emph{Computer Algebra Benchmarks Collection from July
  1998} and shortly explained the results of the \emph{CASBENCH Computer
  Algebra Benchmarks} activities, started in 1995 by the German Fachgruppe.
The CASBENCH
setup\footnote{\url{http://bwcloud-108-017.bwcloud.uni-mannheim.de/cabench/casbench.html}.}
was inspired by the Parkbench
activities\footnote{\url{http://www.netlib.org/parkbench/html/}.} on
\emph{Public International Benchmarks for Parallel Computers} (nowadays known
as the annual HPC challenge award
competition\footnote{\url{http://www.hpcchallenge.org/}.}) but got much less
support from the community. Unfortunately, even within the ISSAC Special
Session on Benchmarking the community could not agree upon a further roadmap or
even a commonly accepted process or dedicated resources to advance that
matter. Once more the diversity of Computer Algebra challenges prevented
awareness of common interests in promoting the development of a common research
infrastructure.

The 1998 special session on Benchmarking was the starting point of the {\SD}
Project. Olaf Bachmann and me developed and implemented with great support by
other members of the Singular team the basic concepts and a first version of a
benchmarking environment in the area of polynomial systems solving, called {\sc
  PolyData} and later on {\SD}.  We collected a considerable number of the
benchmark examples used at that time for testing polynomial systems solver (in
particular the PoSSo test suite), made it publicly and reliably available in a
digital exchange format and developed a standardized environment based on GNU
make and GNU time to run, time and monitor test computations on such examples
using different solvers. But the main conceptual goal of {\SD} was a
nontechnical one – to develop a research infrastructure that is independent of
(permanent) project funding but operates based on overheads of its users. This
approach was inspired by the rich experience of the Open Culture movement
“business models” to run infrastructures. It was an early attempt to emphasize
the advantage of an explicitly elaborated concept of a community-based solution
to the “tragedy of the commons” within the Computer Algebra community and to
apply such a concept to run a part of its research infrastructure.

Nowadays the awareness of the importance of digital research infrastructure
increased both in the scientific communities and also in research politics.
The development of research infrastructures coordinated by the European
Research Infrastructure Consortium (ERIC) plays a crucial role in the EU
funding program HORIZON 2020. \emph{OpenDreamKit} is such a project within
Computer Algebra.  Unfortunately it concentrates on (technical)
interoperability rather than the research data problems discussed above.
``OpenDreamKit is a project that brings together a range of projects and
associate software to create and strengthen virtual research environments. The
most widely used research environment is the Jupyter Notebook from which
computational research and data processing can be directed. The OpenDreamKit
project provides interfaces to well established research codes and tools so
that they can be used seamlessly and combined from within a Jupyter Notebook.''

A different approach is pursued by the \emph{Leibniz Network MMS} on modelling
and
simulation\footnote{\url{https://www.wias-berlin.de/research/Leibniz-MMS/index.jsp?lang=en}.}
with great focus on research data and strong cooperation with TIB
Hannover \footnote{\url{https://www.tib.eu/de/}.} and the upcoming community of 
Research Software
Engineers\footnote{\url{https://www.de-rse.org/de/index.html}.}. 

\section{Testing and Benchmarking in Modern Software Engineering}

Testing and benchmarking is a common task in software engineering. Modern
software development concepts for enterprise middleware provide architectural
and technical support (git workflows, virtualization tools) for agile
approaches as continuous integration, continuous development and continuous
deployment of modular software pieces in distributed environments.  All this
makes software development more complex and requires a good theoretical
understanding of the corresponding architectural concepts.  To maintain such
diverging goals within a single software modern software engineering
distinguishes between core and cross cutting concerns. \emph{Core concerns}
define the main goal of the implementation that is mission critical or requires
special domain specific knowledge, insight and experience. \emph{Cross cutting
  concerns} (logging, testing, profiling, data management, security concerns,
inter process communication etc.) are implemented using well established generic
approaches.  This eases both the maintenance of the software and the training
of the developers. Frameworks as Spring or EJB realize the concept of
\emph{context aware programming}.  The core concerns are implemented in a
runtime that is embedded into another generic runtime -- the \emph{context} --
that provides cross cutting objects to be injected as dependencies into the
core runtime.  Such an approach allows to concentrate combined long time
efforts on the development of the commonly used context environment and to
realize short term core business in a more efficient way.

Such approaches are nowadays common also within the big Computer Algebra
software projects SageMath, OpenDreamKit and OSCAR. The {\SD} benchmark
activities designed by Olaf Bachmann 20 years ago anticipated several such
concepts still unknown at that time.  This framework was enhanced and advanced
during the last years by Albert Heinle and Viktor Levandovskyy to the
\emph{SDEval\footnote{\url{https://symbolicdata.github.io/SDEval}.}  Testing
  and Benchmarking Environment}.  It is used in particular for testing and
benchmarking in the vicinity of the Singular group.  Andreas Nareike
implemented within a project funded by the Saxonian E-Science Initiative
\emph{SDSage\footnote{\url{https://symbolicdata.github.io/PolynomialSystems.Sage}.}}
-- a module for the SageMath \cite{sagemath} generic environment to access the
{\SD} database as injected dependency object.  We don't know to what extend
this embedding is used by the SageMath community. The documentation
refers\footnote{\url{http://doc.sagemath.org/html/en/reference/databases/sage/databases/symbolic_data.html}.}
only to an earlier implementation provided by Martin Albrecht.

\section{Semantic Technologies}

With the consolidation of concepts as Open Access, Open Data and the emerging
semantic web the general understanding of the importance of community-based
efforts to develop common research infrastructures matured.  This development
was accompanied with conceptual, technological and architectural
standardization processes that had also impact on the development of concepts
and data structures within the {\SD} Project.  In 2009 we started to refactor
the data along standard Semantic Web concepts based on the Resource Description
Framework (RDF).  With a new {\SD} version released in September 2013 we
completed the redesign of the data along RDF based semantic technologies, set
up a Virtuoso based RDF triple store and an SPARQL endpoint as Open Data
services along Linked Data
standards\footnote{\url{https://en.wikipedia.org/wiki/Linked_data}.}. The
importance of the yet heavily growing Linked Open Data
Cloud\footnote{\url{http://lod-cloud.net}.} is hardly to underestimate.

\section{Semantic Aware Fingerprints}

The main goal of {\SD} is on data -- structure, maintain and present research
data in a digitally and publicly available way. 20 years ago we started with
examples from polynomial systems challenges, with the PoSSo test suite and
other sources. To make such a collection searchable one has to define and
compile meta information about the different objects to cluster them or even
identify a single one. This is challenging in particular for polynomial systems
since the same example can be noted with different variable names and different
term orders.  Hence a pure string matching doesn't work.  As a first approach
we compiled invariants of polynomial normal forms and stored it together with
the basis itself in a single information object.  Such a strategy -- to combine
data and metadata in a single object -- is commonly used also nowadays, e.g.,
within the LOM standard -- the Learning Objects Metadata are tightly coupled
with the Learning Objects themselves.

For our use case such a concept turned out to be suboptimal since it led to an
explosion of data: polynomial systems can be interpreted in different ways,
e.g., keeping a part of the variables as parameters, as homogenized ideals,
bounding variables to special values etc.. Each such version had to be kept as
a new data object since the metadata changed even if the basis was the same or
could be easily (i.e., in polynomial time) generated from another example. In
later versions we used the universal property of the ring
$\mathbbm{Z}[x_1,\dots,x_n]$, decided to reduce the number of stored systems
and keep track of the way how derived systems are generated from basic ones.

RDF strongly supports such a distinction between data (\emph{resources} in the
RDF terminology) and meta information (\emph{resource descriptions}). Data is
represented by URI's that can point even to remote locations.  Hence RDF is
well suited to describe also distributed research data even if the data is
maintained by different stakeholders and only the metadata is federated in a
common RDF store for search and data analytics.

{\SD} operates such a central RDF store \cite{sdstore} and Andreas Nareike
enhanced our metadata during his e-science project funded in 2012--2013 with
metadata from two distinguished sources -- the polytopes database of Andreas
Paffenholz \cite{Paffenholz} and the transitive groups Database for Number
Fields of Gunter Malle and J\"urgen Kl\"uners \cite{MalleKlueners}.  A central
challenge was the definition of the metadata as a social process that requires
not only sufficient domain specific insight but also resilient agreements about
responsibilities to update and maintain the data and metadata.  The first part
of this challenge is reflected in our concept of \emph{semantic-aware
  fingerprints} that focuses on the usage condition of \emph{any} research
data (awareness of the semantics) and our special use case for the meta
information -- search. The needle in the haystack, the fingerprint in the
police file or the puzzle piece in the stack: To find it a clear domain model
and a clear search strategy are required, and both are not independent one from
the other.  For more details we refer to \cite{icms-16}.

After a certain consolidation process on March 1, 2016, version 3.1 of the
{\SD} tools and data was released. The new release contained
\begin{itemize}
\item new resource descriptions (``fingerprints'') of remotely available data
  on transitive groups (\emph{Database for Number Fields} of Gunter Malle and
  J\"urgen Kl\"uners \cite{MalleKlueners}) and polytopes (databases of Andreas
  Paffenholz \cite{Paffenholz} within the \emph{polymake} project
  \cite{polymake}),
\item a recompiled and extended version of test sets from integer programming
  -- work by Tim R\"omer (\emph{normaliz} group \cite{normaliz}), 
\item an extended version of the \emph{SDEval benchmarking environment} -- work
  by Albert Heinle \cite{heinle-15} -- and
\item a partial integration ({\SD} People database, databases of upcoming and
  past conferences) of data from the CASN -- the Computer Algebra Social
  Network subproject.
\end{itemize}
Furthermore, our github
account\footnote{\url{https://github.com/symbolicdata}.} was transformed into
an organizational account and the git repo structure was redesigned better to
reflect the special life-cycle requirements of the different parts of our
activities.

\section{Research Infrastructure as a Social Project}

So far we mainly discussed technical questions of structuring data, defining
and compiling metadata and designing tools and workflows for local testing and
benchmarking activities.  But benchmarking -- as \emph{any} process of
scientific evaluation -- is primarily a \emph{social} process.  In other areas
of science there are well established benchmark competitions for different
algorithmic problem classes with clearly defined rules and places.

In 2012 we organized a workshop on
benchmarking\footnote{\url{https://symbolicdata.github.io/Events.2012-12}.}
with people from communities close-by Computer Algebra.  Satya Samal presented
the PoCaB Project -- Platform of Chemical and Biological Analysis Using
Computer Algebra Methods -- and explained in detail structural approaches
within the PoCaB Databases and how data are generated within the PoCaB
framework.  PoCaB is interlinked with different communities within Computer
Algebra (the polynomial systems solving and the polymake communities) and also
beyond. It heavily exploits biological databases (BioModel Database, KEGG
Database) that come with their own language SMBL and experiences how to express
semantic aspects in a computer readable way. This example showed very clearly
that communities are not interested in advice from outside how to reinvent
wheels properly running for a long time within the community but acknowledge
support and advice how to organise intercommunity communication properly and
more smoothly in a world of evolving Linked Open Data standards.

Johannes Waldmann gave a talk about Benchmarks and Competitions in Theoretical
Computer Science presenting best practices of three TCS Communities:
Termination, SAT and SMT. For Termination he explained TPBD -- the Termination
Problems Data Base -- and their way of benchmarking: They regularly organize
Termination Competitions on previously agreed data from different problem
categories in a similar way as the “Formula I” car race is organized: Upload
tools to a single dedicated server that runs all tools on all problems and
collects the results in aggregated form on a web page. Usually such a
competition runs accompanying the annual large conference in the field.
Similarly structured competitions take place in other areas of science, e.g.,
in High Performance Computing\footnote{\url{http://www.hpcchallenge.org},
  discontinued after 2014.} or in the SAT Solver
community\footnote{\url{http://www.satcompetition.org/}.}.

The 1998 Special Session on Benchmarking stated that such contests with clear
rules by certain reasons are lacking in the area of Computer Algebra. This did
not change during the last 20 years.  Evaluating the reason for such a
longstanding deficit we observed that socially mounted benchmarking cultures
live in certain Computer Algebra subcommunities but are rarely communicated
beyond their scope. So what's about communication between Computer Algebra
subcommunities in general? RDF concepts are well suited not only to describe
collections of benchmark data but also to support communication on other
scientific activities and achievements between different subcommunities.
Properly organized metadata generated by different stakeholders can easily be
collected not only in a central store but also in a well organized distributed
environment as a ``scientific Facebook'' -- we called such a concept
\emph{Computer Algebra Social Network} (CASN) \cite{cicm-14} -- that could be
implemented as a network of \emph{CASN nodes} as part of a social research
infrastructure within the Linked Open Data Cloud.

Since 2012 we tried to identify problem settings of common interest,
implemented building blocks of such a network, tried to get showcases socially
running and promoted our CASN idea.  We report shortly about three of these
showcases and refer to our
wiki\footnote{\url{https://symbolicdata.github.io/CASN}.} for more details.

\paragraph{Conferences in Computer Algebra.}
Reporting about upcoming conferences seems to be a common need in many Computer
Algebra subcommunities and could be a first class service of a CASN. The German
Fachgruppe set up such a service for a long time in printed form within their
Rundbrief.  Upcoming conferences are listed independently on the websites of
both SIGSAM and the German Fachgruppe.  We maintained for several years such
information about upcoming and (archiving the entries) past conferences in a
structured RDF format that can be used to extract the different web and printed
views from a single commonly maintained source.  Defining such an exchange
format the entries can even be produced by the subcommunities and the boards
have merely to collect the information.  We terminated that service due to
limited staff capacity. A presentation of our past conferences collection can
be found at our {\SD} demonstration site \cite{sdinfo}.

RDF is well suited to combine such conference announcements with more detailed
information about the conferences (tracks and sessions, papers and authors
etc.) that has to be compiled, e.g., for the web presentation of the
conference.  In many cases such information is already stored in a structured
way and the web site of the conference is generated from that source.  In
particular, Serge Autexier as the publicity chair invented such a model for the
CICM conferences\footnote{\url{https://www.cicm-conference.org/cicm.php}.} and
compiled all information of each of the 12 conferences in publicly available
XML files, thus arriving at level 3 of the 5 stars
scale\footnote{\url{https://5stardata.info/de/}.} for Open Data of Tim
Berners-Lee.

As a showcase we transformed four of these presentations into RDF and stored it
in our CASN node\footnote{\url{http://symbolicdata.org/rdf}.}. This is level 4
of the 5 stars scale since the data is available as RDF but not operated within
a RDF store and thus not directly accessible for SPARQL query exploration.
This could be part of an upcoming conference reporting structure within an
emerging CASN. 

\paragraph{The {\SD} People Database.}
Conference announcements are a first class resource of information about people
actively working in Computer Algebra.  We attached to our conference records
information about organizers, invited speakers, program committees etc.  We
have more than 1.000 entries in our database and joint forces\footnote{See
  \url{https://symbolicdata.github.io/Events.2014-07} for details.} with
Wolfram Sperber and Uwe Sch\"oneberg (Zentralblatt) to solve the problem of
identification of those people in the Zentralblatt and partly also in the
MathReviews.  A presentation of this database can be found at our {\SD}
demonstration site \cite{sdinfo}.

Such a People Database maintains a set of established URI's and thus is a
central building block to get activities within Computer Algebra recognized
within the Linked Open Data world.  It allows to embed the ``stories'' told
within Computer Algebra and its subcommunities into a bigger world, to join
forces with the author disambiguation projects of ``big players''
(Zentralblatt, Math Reviews, ACM digital library, Springer, Elsevier, ORCID,
ResearchGate, VIAF, GND) and thus to increase the visibility of Computer
Algebra research in the emerging digital world.

\paragraph{Computer Algebra Software.}
Another central problem within benchmarking Computer Algebra software is
software disambiguation. {\SD} started 20 years ago to maintain a consolidated
list of Computer Algebra software. With the maturing swMATH project
\cite{swmath} we stopped in 2012 such activities and compiled together with
Wolfram Sperber and Hagen Chrapary (Zentralblatt) a translation list between
our URI's and those of swMATH. In the last years there was much discussion
(What is a software, what a package? How to deal with libraries or different
versions of the same software?)  but little practical progress to prepare that
collection for the Linked Open Data world of the 21st century. Being a first
class reference of mathematical software swMATH achieves only 2 of the 5 stars
of Tim Berners-Lee since it has no open interface to the data itself.

As a showcase we compiled in a common effort with Wolfram Sperber a
consolidated RDF based version of Computer Algebra software (that is only a
\emph{part} of swMATH, since swMATH addresses mathematical software in general)
combining URI's and descriptions from swMATH, the SIGSAM list of Computer
Algebra
software\footnote{\url{https://www.sigsam.org/Resources/Software.html}.} and
also the (very outdated)
overview\footnote{\url{http://www.fachgruppe-computeralgebra.de/systeme/}.} on
the website of the German Fachgruppe.  We used an undocumented feature of
swMATH to compile also links to the abstracts of 10 papers related to that
software.  The data is available also from our RDF store hence earns all 5
stars of Tim Berners-Lee.  A presentation of this database can be found at our
{\SD} demonstration site \cite{sdinfo}.

\section{{\SD} as Non-Project}

A project is usually defined by a \emph{goal}, \emph{attached resources}
(money, human resources) and a \emph{time span} (as basis for planning, work
packages, milestones etc.).  20 years ago {\SD} grew up from the relicts
remaining after the end of two such projects -- PoSSo and FRISCO -- and was
designed from the very beginning as \emph{non-project} -- it was driven by
casual volunteers, bringing in their own resources (time, web space), it was
partly supported by different community structures (the Singular group, UMS
Medicis, the German Fachgruppe) and it had never a defined project end but
survived several ``dry periods'' almost without activities.

Such a situation is typical for research infrastructures and it is hard to
allocate resources for such non-projects in a time of increasing importance of
project-oriented research funding.  The problems and workarounds are described
on the pages of the OEIS Foundation as the goals of another old (since 1964)
research infrastructure non-project -- \emph{The Online Encyclopedia of Integer
  Sequences} -- in the following way: ``1) own the intellectual property, 2)
maintain the infrastructure as a service that is freely accessible by the
general public, 3) act so as to maintain its own existence indefinitely, 4)
collect and distribute funds in order to carry out the first three
goals.''\footnote{\url{http://oeisf.org/\#GOALS}.}

During the last years the {\SD} team spent efforts on goal 3 to prepare for
another ``dry period'' since we didn't succeed with goal 4\footnote{For details
  we refer to \url{https://symbolicdata.github.io/New.html}.}.  We concentrated
the {\SD} data and wiki at our github account and terminated several of our
ongoing activities (updating the record of upcoming conferences, advancing the
alignment with swMATH or the dissertations
project\footnote{\url{https://symbolicdata.github.io/Dissertations}.})

One of the core semantic web facilities of {\SD} is the domain
\texttt{http://symbolicdata.org} as a prefix of the {\SD} ontologies.  By the
RDF best practices it is of great importance to own that domain and to set up
and operate an RDF store under that web address.  The domain
\texttt{http://symbolicdata.org} is owned by the German Fachgruppe and
currently operated on a server at Leipzig University.  Unfortunately, the Board
of the German Fachgruppe doesn't understand well enough the importance to keep
running such an arrangement ``indefinitely'' (private communication with Gregor
Kemper).

\section{What Else?}

We acknowledge the strong support from the Board of the German Fachgruppe over
many years who sponsors the domain \texttt{symbolicdata.org} since 2005 and was
the power partner in our experiments towards a CASN. During the last years
(2012--2017) we presented {\SD} at several international conferences and
submitted 6 papers for publication (2 accepted, 4 rejected\footnote{For details
  we refer to \url{https://symbolicdata.github.io/Publications.html}.}), not
counting our contributions to the Rundbrief of the German Fachgruppe.

Stephen Watt asked in the discussion to my presentation in the \emph{Work in
  Progress} session at CICM 2014 ``How will you sustainably attract resources
for your project?'' In my response I shortly explained our non-project
philosophy, the role of casual volunteers and ended with the famous answer of
Linus Torvalds on a similar question posed by Andrew Tannenbaum: ``I won't.''
But time certainly changed, nowadays there is a big competition between
projects resting on such ``casual volunteers'' and one has to spent much time
in advertising the own projects.

We did so and tried to align {\SD} not only with swMATH but also with other big
community projects as OpenDreamKit (Michael Kohlhase), OSCAR (Wolfram Decker),
SIGSAM (Ilias Kotsireas, Matthew English) or people showing interest in ``big
old data'' (Bernd Sturmfels). Such advertisement could only be done with very
restricted resources since the single volunteer actively developing {\SD} at
the moment is a specialist on semantic technologies but far away from core
Computer Algebra for many years.  The results were disappointing.  Even the
German Fachgruppe stopped with the relaunch of their website its direct
cooperation with {\SD} and moved the pages\footnote{See the overview at
  \url{http://www.fachgruppe-computeralgebra.de/symbolicdata/}.} with input
from the CASN node of the German
Fachgruppe\footnote{\url{http://www.fachgruppe-computeralgebra.de/rdf/}.} into
the background.

A great number of people (Gert-Martin Greuel, Gerhard Pfister, Winfried Neun,
Wolfram Sperber, Hannes Schöne\-mann, me) involved in one or another way with
{\SD} already retired or will retire during the next years.  Other people (Olaf
Bachmann, Ralf Hemmecke, Andreas Nareike, Viktor Levandovsyy, Albert Heinle)
timely involved with {\SD} left Computer Algebra or are inactive with the
project at the moment.

In this paper we described the main achievements and conceptual points of the
{\SD} project so far.  It is up to the next generation to take over the baton,
to evaluate the {\SD} heritage and update the parts that are worth to be
continued. If any.

\begin{thebibliography}{xxx}\raggedright
\bibitem{normaliz} Bruns, W., Ichim, B., R\"omer. T., Sieg, R., S\"oger, C.:
  Normaliz. Algorithms for Rational Cones and Affine Monoids.
  \url{https://www.normaliz.uni-osnabrueck.de}. [2018-09-02]
\bibitem{FRISCO} FRISCO -- A Framework for Integrated Symbolic/Numeric
  Computation. (1996--1999).
  \url{https://cordis.europa.eu/project/rcn/31471_de.html}.  [2018-09-02]
\bibitem{polymake} Gawrilow, E., Joswig, M.: Polymake: a Framework for
  Analyzing Convex Polytopes. In: Kalai, G., Ziegler, G.M. (eds.), Polytopes --
  Combinatorics and Computation (Oberwolfach, 1997), pp. 43--73, DMV Sem., 29,
  Birkh\"auser, Basel (2000). 
\bibitem{cicm-14} Gr\"abe, H.-G., Johanning, S., Nareike, A.: The {\SD} Project
  -- Towards a Computer Algebra Social Network. In: Workshop and Work in
  Progress Papers at CICM 2014, CEUR-WS.org, vol. 1186 (2014).
\bibitem{car-55} Hans-Gert Gr\"abe, Simon Johanning, Andreas Nareike.
  \newblock The {\SD} Project -- from Data Store to Computer Algebra Social
  Network. \newblock \emph{Computeralgebra-Rundbrief}, 55:22--26, 2014.
\bibitem{icms-16} Hans-Gert Gr\"abe. \newblock Semantic-aware Fingerprints of
  Symbolic Research Data. \newblock In Gert-Martin Greuel, Thorsten Koch, Peter
  Paule, Andrew Sommese (Eds.).  \emph{Mathematical Software -- ICMS 2016}.
  \newblock Volume 9725 of Lecture Notes in Computer Science, page 411--418,
  2016.  \newblock DOI 10.1007/978-3-319-42432-3.
\bibitem{heinle-15} Heinle, A., Levandovskyy, V.: The SDEval Benchmarking
  Toolkit. ACM Communications in Computer Algebra, vol. 49.1, pp. 1--10 (2015).
\bibitem{MalleKlueners} Kl\"uners, J., Malle, G.: A Database for Number Fields.
  \url{http://galoisdb.math.uni-paderborn.de/}. [2018-09-02]
\bibitem{odk} OpenDreamKit: Open Digital Research Environment Toolkit for the
  Advancement of Mathematics. \url{http://opendreamkit.org/}. [2018-09-01]
\bibitem{oscar} OSCAR: The OSCAR project.
  \url{https://oscar.computeralgebra.de/}.  [2018-09-01]
\bibitem{Paffenholz} Paffenholz, A.: Polytope Database.
  \url{http://www.mathematik.tu-darmstadt.de/~paffenholz/data/}.  [2018-09-02] 
\bibitem{PoSSo} The PoSSo Project. Polynomial Systems Solving -- ESPRIT III BRA
  6846.  (1992--1995).
  \url{https://cordis.europa.eu/project/rcn/9106_en.html}.  [2018-09-02]
\bibitem{sagemath} The SageMath Project.  \url{http://www.sagemath.org/}.
  [2016-03-16]
\bibitem{swmath} swMATH -- an Information Service for Mathematical Software.
  \newblock \url{http://swmath.org}. [2018-09-02]
\bibitem{sdinfo} The {\SD} Demonstration site.
  \url{http://symbolicdata.org/info}.  [2018-09-02]
\bibitem{sdstore} The {\SD} RDF Data Store.
  \url{http://symbolicdata.org/Data}.  [2018-09-02]
\bibitem{sdpub} The {\SD} Publications.
  \url{http://symbolicdata.github.io/Publications}.  [2018-09-02]
\end{thebibliography}

\end{document}
