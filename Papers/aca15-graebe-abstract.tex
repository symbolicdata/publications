%    Latex Template for ACA 2015
%%%%%%%%%%%%%%%%%%%%%%%%%%%%%%%%%%%%%%%%
%    DO NOT CHANGE THE SETTINGS BELOW
%%%%%%%%%%%%%%%%%%%%%%%%%%%%%%%%%%%%%%%%
\documentclass[11pt]{article}
\usepackage{epsfig}
\usepackage{graphicx}
\usepackage[latin1]{inputenc}
\usepackage[T1]{fontenc}
\usepackage{mathptmx}
\usepackage{url}



\begin{document}



%%%%%%%%%%%%%%%%%%%%%%%%
%%%%%%%%%%%%%%%%%%%%%%%%%%%%%%%%%%%%%%%%
%    DO NOT CHANGE THE SETTINGS ABOVE
%%%%%%%%%%%%%%%%%%%%%%%%%%%%%%%%%%%%%%%%
%
\vspace{-0.8cm}
\begin{flushleft}
\Large \textbf{\noindent
%%% PLEASE INSERT THE TITLE OF YOUR TALK HERE %%%%%%%
SymbolicData, Computer Algebra and Web 2.0}
%%%%%%%%%%%%%%%%%%%%%%%%%%%%%%%%%%%%%%%%%%%%%%%%%%%%%%%%
\\
%%%%%%%%%%%%%%%%%%%%%%%%%%%%%%
\vspace{0.5cm}
\normalsize
 %%%%%%             NAMES OF AUTHORS         %%%%%%
 %%%%%% NAME OF SPEAKER SHOULD BE UNDERLINED %%%%%%%%%
\normalsize{
 %%%%%% FOR EXAMPLE %%%%%
\underline{H.-G. Gr\"abe}$^1$, A. Heinle$^2$, S. Johanning$^3$
 %%%%%%%%%%%%%%%%%%%%%%%%
} \\
\vspace{5mm}
\textit{\footnotesize
 %%%%%% AFFILIATION OF AUTHORS %%%%%%
$^1$ Leipzig University, Germany,
graebe@informatik.uni-leipzig.de\\
$^2$ University of Waterloo, Canada,
aheinle@uwaterloo.ca\\
$^3$ Leipzig University, Germany,
simonjohanning@googlemail.com\\
 %%%%%%%%%%%%%%%%%%%%%%%%%%%%%%%%%%%%%
}
\end{flushleft}
\def\SD{\textsc{Symbolic\-Data}}
%The extended abstract length is limited to 5  pages

\section{Introduction}

What is Computer Algebra (CA)? Twenty years ago more than 200 leading edge
computer algebraists in a worldwide joint effort compiled a description of the
CA landscape \cite{CAHB} and defined the target of CA in the following way:
\begin{quote}\small
  ``Computer Algebra is a subject of science devoted to methods for solving
  mathematically formulated problems by symbolic algorithms, and to
  implementation of these algorithms in software and hardware. It is based on
  the exact finite representation of finite or infinite mathematical objects
  and structures, and allows for symbolic and abstract manipulation by a
  computer. Structural mathematical knowledge is used during the design as well
  as for verification and complexity analysis of the respective algorithms.
  Therefore computer algebra can be effectively employed for answering
  questions from various areas of computer science and mathematics, as well as
  natural sciences and engineering, provided they can be expressed in a
  mathematical model.'' \cite[p. 2]{CAHB}
\end{quote}
Johannes Grabmeier, at that time head of the German CA Fachgruppe,
developed an even broader view of a subject \emph{Computer Mathematics} as a
symbiosis of computer technology and mathematics at large as the true core of
``Scientific Computing'' \cite{Grabmeier95}. Such a
\emph{technology\footnote{Technology considered in the broad sense of
    \emph{applicable processual knowledge} in the ancient greek meaning, see
    e.g., \cite{TechnologyWikipedia}, not in the narrow meaning of \emph{tool,
      tool making, tool using} as, e.g., in the German VDI-3780 norm
    \cite{VDI-Richtlinie}.}  of quantitative methods} is a corner stone of any
science, that ``can be considered as developed''\footnote{A claim attributed to
  Karl Marx by Paul Lafargue. David Hilbert supports such a view: ``Everything
  that can be an object of scientific thinking as soon as it matures to
  formation of theories is in the bondage of the axiomatic method and thus
  indirectly of mathematics.''  \cite{Hilbert} }.  Figure 1 displays these
concepts, the difference between deductive and numerical mathematics and the
position of such a \emph{Computer Mathematics} between mathematics and computer
science.

\begin{table}[b]
\begin{center}
\newcommand{\abox}[1]{\framebox(35,15){\parbox{3cm}{\centering\large #1}}}
\newcommand{\bbox}[1]{\makebox(35,25){\parbox{3cm}{\centering\small #1}}}
\setlength{\unitlength}{.8mm}
\begin{picture}(140,140)
\put(40,125){\framebox(65,10){\centering \Large Mathematics}}

\put(10,100){\abox{Numerics}}
\put(55,100){\abox{Symbolic Computing}}
\put(100,100){\abox{Discrete Mathematics}}

\put(10,70){\bbox{Image of an uncountable world in the finite memory
    of a computer}}
\put(55,70){\bbox{Semantically infinite, descriptionally finite mathematical
    structures}}
\put(100,70){\bbox{Structurally finite mathematical structures}}

\put(10,50){\dashbox(35,15){\parbox{3cm}{\centering\large
      Approximative\\ Mathematics}}}
\put(55,50){\dashbox(80,15){\parbox{6.5cm}{\centering \large deduktive
      or\\ ``exact'' Mathematics}}}

\put(27,43){\LARGE$\Downarrow$}
\put(72,43){\LARGE$\Downarrow$}
\put(117,43){\LARGE$\Downarrow$}

\put(10,30){\framebox(125,10){\Large Computer-Mathematics}}

\put(45,23){\LARGE$\Uparrow$}
\put(72,23){\LARGE$\Uparrow$}
\put(99,23){\LARGE$\Uparrow$}

\put(40,0){\framebox(65,20){\parbox{6cm}{\centering \Large Computer
    Science\\ (Informatics)}}} 

\thicklines
\put(55,124){\vector(-3,-1){25}}\put(72,124){\vector(0,-1){9}}
\put(90,124){\vector(3,-1){25}}

\end{picture}\\[10pt]
\textbf{Figure 1: The Genesis of Computermathematics}
\end{center}
\end{table}

Twenty years later a practical incarnation of such a vision is any of the
mature General Purpose Computer Algebra Systems, in particular the one that
claims to be ``the world's definitive system for modern technical computing''
\cite{Mma}.

What remained from such an integrated view -- a \emph{CA Tower of Babel} --
twenty years later? What is the relation between the sections of ACA 2015, that
address specialized and over the years even more and more specialized topics?
What are the consequences of a division of CA into more and more sub- and
subsubcommunities? Has \textsc{the Lord} confused their language\footnote{But
  the Lord came down to see the city and the tower which the sons of men had
  built.  And the Lord said, ``Indeed the people are one and they all have one
  language, and this is what they begin to do; now nothing that they propose to
  do will be withheld from them. Come, let Us go down and there confuse their
  language, that they may not understand one another's speech.'' (Genesis 11)}
once more?

\section{The {\SD} Project}
\subsection{The Roots}
% source: car-54.tex -- The {\SD} Project at Large

{\SD} hab been part of CA infrastructural efforts for more than 15 years.  It
grew up from the Special Session on Benchmarking at the 1998 ISSAC conference,
and started to build a reliable and sustainably available reference of
Polynomial Systems data, to extend and update it, to collect meta information
about the records, and also to develop tools to manage the data and to set up
and run testing and benchmarking computations on the data. The main design
decisions and implementations of the first prototype were realized by Olaf
Bachmann and Hans-Gert Gr\"abe in 1999 and 2000. We collected data from
\emph{Polynomial Systems Solving} and \emph{Geometry Theorem Proving}, set up a
CVS repository, and started test computations with the main focus on Polynomial
Systems Solving.

In 2005 the Web site \url{http://www.symbolicdata.org} sponsored by the German
CA Fachgruppe went online and a second phase of active development started.
Data was supplied by the CoCoA group (F.~Cioffi), the Singular group
(M.~Dengel, M.~Brickenstein, S.~Steidel, M.~Wenk), V.~Levandovskyy (non
commutative polynomial systems, G-Algebras) and Raymond Hemmecke (Test sets
from Integer Programming). During the Special Semester on Groebner Bases in
March 2006 we discussed aspects to extend the project, in particular with Bruno
Buchberger, Alexander Zapletal, and Viktor Levandovskyy.

\subsection{Version 3 -- {\SD} Goes Semantic}
% source: car-54.tex -- The {\SD} Project at Large

A third phase started in 2009 when the project joined forces with the Agile
Knowledge Engineering and Semantic Web (AKSW) Group at Leipzig University
\cite{AKSW} in order to strongly refactor the data along standard Semantic Web
concepts based on the Resource Description Framework (RDF) \cite{RDF}.  In
2012--2014 these efforts were supported by a 12 month grant for Andreas Nareike
and another 5 month grant for Simon Johanning within the \emph{Saxonian
  E-Science Initiative} \cite{E-Science-Sachsen}.

Within this scope we completed a redesign of the data along the rules of Linked
Data and semantic, RDF-based technology, distinguishing more consequently
between data (\emph{resources} in the RDF terminology) and meta data
(\emph{knowledge bases} in the RDF terminology).  The new {\SD} data and tools
were released as version~3.0 in September 2013.  Version 3.1 is to be released
in the mid of 2015.

Resources (examples for testing, profiling and benchmarking software and
algorithms from different CA areas) are publicly available (mainly) in XML
markup, meta data in RDF notation both from a public git repo, hosted at
\texttt{github.org}, and from an OntoWiki based RDF data store at
\url{http://symbolicdata.org/Data}.  Moreover, we offer a SPARQL endpoint to
explore the data by standard Linked Data methods.

The website operates on a standard installation using an Apache web server to
deliver the data, a Virtuoso RDF data store as data backend, a SPARQL endpoint
and (optionally) OntoWiki to explore, display and edit the data.  This standard
installation can easily be rolled out on a local site (tested under Linux
Debian and recent Ubuntu versions; a more detailed description can be found in
our wiki \cite{sdwiki}) to support local testing, profiling and benchmarking.

The distribution offers also tools for integration with a local computational
environment as, e.g., provided by Sagemath \cite{Sagemath} -- the Python based
\emph{SDEval package} \cite{sdeval} by Albert Heinle offers a JUnit-like
framework to set up, run, log, monitor and interrupt testing and benchmarking
computations, and the \emph{sdsage package} \cite{sdsage} by Andreas Nareike
provides a showcase for {\SD} integration with the Sagemath computational
environment.

Currently the {\SD} data collection contains resources from the areas of
Polynomial Systems Solving (390 records, 633 configurations), Free Algebras (83
records), G-Algebras (8 records), GeoProofSchemes (297 records) and Test Sets
from Integer Programming (49 records).

Note that such a concept is not restricted to resources centrally managed at
\texttt{symbolicdata.org}, but can easily be extended to other data stores on
the web that are operated by different CA subcommunities and offer a minimum of
Linked Data connectivity.  There are draft versions of resource descriptions
about Fano Polytopes (8630 records) and Birkhoff Polytopes (5399 records) from
the \texttt{polymake} project hosted by Andreas Paffenholz and about Transitive
Groups (3605 records) from the Database for Number Fields of J{\"u}rgen
Kl{\"u}ners and Gunter Malle that point to external resources.  For Test Sets
we joined forces with the Normaliz group --- Tim R{\"o}mer provided some new
benchmarks and refactored the old examples to fit with the \texttt{normaliz}
syntax.

\subsection{The Vision: Towards a Computer Algebra Social Network}
% source: car-54.tex -- The {\SD} Project at Large

I come back to the \emph{CA Tower of Babel} image unfolded in the introduction.
If we shift the focus of any project -- and we did so for {\SD} with version
3.0 -- from the \emph{data} to the \emph{people and their intentions}
perspective -- \emph{why} they collect and manipulate such data -- new
organizational perspectives for common efforts are revealed.  The business of
any CA project is a techno-social one and we think in the spirit of
sociomateriality \cite{Sociomateriality}, it is time to use the \emph{technical
  means} of a semantically enriched Web 2.0 to also strengthen the
\emph{social} part of our cooperation and to contribute to the efforts to build
up an interconnected \emph{E-Science World}.

During the last years such efforts matured within the \emph{Science at Large}.
Services such as MathSciNet, arXiv.org, or EasyChair.org have been established
and their usefulness is widely acknowledged. There are plenty of new
activities, in particular by the \emph{national libraries and organizations}
\cite{VIAF}, by the \emph{Zentralblatt Mathematik} \cite{postagging}, or by the
IMU, that advances the vision of a \emph{21\textsuperscript{st} Century Global
  Library for Mathematics Research} (GDML) \cite{GDML}.

It is a great challenge to smaller scientific communities to adopt such
developments for their own scientific communication processes and to join
forces with other scientific communities to get own requirements publicly
recognised.  A first step in such a direction could be a more detailed
description of ongoing scientific processes using standard RDF terminology.

With version 3 {\SD} started to address the technical aspects of such
cooperational needs in more detail, developed a vision of a \emph{Computer
  Algebra Social Network} (CASN) \cite{casn}, and started to realize it.

\section{About our Talk}

In this abstract we tried to explain the background of our project and in
particular the data services available to the public. We operate these services
on a stable basis, but the current focus of the project is on the CASN concepts
and the difficult (social) task to get them running. In our presentation at ACA
2015 we will concentrate on a report about the current state of our efforts
towards such a CASN. To prepare for our talk we invite interested people to
look at our project wiki \cite{sdwiki} and in particular at the survey paper
\cite{casn}.

We expect fruitful discussions and hope to convince more people that such a
project is not only a ``nice to have'' but also deserves own \emph{practical}
efforts.  The train is ready for departure. Don't miss it!

%%%%%%%%%%%%%%%%%%%%%%%%%%%%%%%%%%%%%%%


%%%%%%%%%%%%%%%%%%%%%%%%%%%%%%%%%%%%%%%%
%% BIBLIOGRAPHY
%%%%%%%%%%%%%%%%%%%%%%%%%%%%%%%%%%%%%%%%
\begin{thebibliography}{00}
\addcontentsline{toc}{chapter}{References}
\setlength{\itemsep}{-1mm}
\footnotesize\raggedright

\bibitem{AKSW} The Agile Knowledge Engineering and Semantic Web Group at
  Leipzig University, \url{http://aksw.org/About.html} [2014-02-19].

\bibitem{GDML} \textit{Developing a 21st Century Global Library for Mathematics
  Research.  Report of the Committee on Planning a Global Library of the
  Mathematical Sciences}, The National Academies Press (2014).

\bibitem{E-Science-Sachsen} The eScience Research Network Saxony,
  \url{http://www.escience-sachsen.de} [2014-02-19].

\bibitem{CAHB} J. Grabmeier, E. Kaltofen, V. Weispfenning (eds.),
  \textit{Computer Algebra Handbook. Foundations -- Applications -- Systems},
  Springer Verlag, Berlin (2003).

\bibitem{Grabmeier95} J. Grabmeier, \textit{Computeralgebra -- eine S\"aule des
  Wissenschaftlichen Rechnens},  it + ti, \textbf{6:5}, p. 20 (1995).

\bibitem{casn} H.-G.\ Gr\"abe, S.\ Johanning, A.\ Nareike, \textit{The {\SD}
  Project -- from Data Store to Computer Algebra Social Network},
  Computeralgebra Rundbrief, vol. \textbf{55}, pp. 22--26 (2014), see also
  \url{http://symbolicdata.org/Papers/car-55.pdf} [2015-04-12].

\bibitem{sdeval} A. Heinle, V. Levandovskyy, \textit{The SDEval Benchmarking
  Toolkit}, Communications in Computer Algebra, vol. \textbf{49.1},
  pp. 1--10 (2015), see also \url{http://wiki.symbolicdata.org/SDEval}
  [2015-04-12].

\bibitem{Hilbert} Citation from \url{http://de.wikipedia.org/wiki/Wissen}
  [2015-04-11], attributed to Hermann Weyl, \textit{Philosophie der Mathematik
    und Naturwissenschaft}, R. Oldenbourg, Munich (1976).  

\bibitem{Mma} \url{http://www.wolfram.com/mathematica/} [2015-04-11]

\bibitem{sdsage} A. Nareike, \textit{The {\SD} sdsage package},  
  \url{http://wiki.symbolicdata.org/PolynomialSystems.Sage} [2015-04-12].

\bibitem{postagging} U. Sch\"oneberg, W. Sperber, \textit{POS Tagging and its
  Applications for Mathematics}, in \textit{Intelligent Computer Mathematics},
  LNCS vol. 8543, pp. 213--223 (2014).

\bibitem{Sociomateriality} Sociomateriality, see 
\url{http://en.wikipedia.org/wiki/Wanda_Orlikowski} [2015-04-12].

\bibitem{Sagemath} Sage -- a free open-source mathematics software system,
  \newblock \url{http://www.sagemath.org} [2014-02-19].

\bibitem{sdwiki} The {\SD} Project Wiki, \url{http://wiki.symbolicdata.org}
  [2015-04-12].

\bibitem{RDF} J. Tauberer, \textit{Quick Intro to RDF},
  \url{http://www.rdfabout.com/quickintro.xpd} [2014-02-20].

\bibitem{TechnologyWikipedia} Technology,
  \url{http://en.wikipedia.org/wiki/Technology} [2015-04-11].

\bibitem{VDI-Richtlinie} DIN Deutsches Institut f{\"u}r Normung e.V.,
  \textit{VDI 3780 -- Technology Assessment Concepts and Foundations},
  VDI-Richtlinie, ed. 2000-09.

\bibitem{VIAF} VIAF: Virtual International Authority File.
  \url{https://viaf.org/} [2015-04-12].
\end{thebibliography}
\end{document}

\bibitem{KR} B. Keller and I. Reiten, \textit{Cluster-titled algebras are
  Gorenstein and stably Calabi-Yau}, Adv. Math. \textbf{211}, 1, pp. 123-151
  (2007).
\bibitem{WK} S. Washito and K. Karabashi, \textit{Use This Template}, in
  \textit{Proceedings of the 5th International Conference on Photon
    Interaction} (ICPI-2012), San Francisco, USA, ed. A. Chen, pp. 22-24
  (2012).
\bibitem{ABS} A.B. Smith, \textit{Describing the style for the AMMCS-CAIMS-2015
  book of abstracts}, in C.D. Science (ed.), General Book of Wisdom,
  2\textsuperscript{nd} ed., University of Queensland, pp. 23-42 (2011).
