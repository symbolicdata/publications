%==================================================
% ICMS LaTeX Template
%==================================================

%===== DO NOT MODIFY ==============================
\documentclass[runningheads,a4paper]{llncs}
\usepackage{amssymb}
\setcounter{tocdepth}{3}
\usepackage{graphicx}
\usepackage{url}
\newcommand{\keywords}[1]{\par\addvspace\baselineskip
\noindent\keywordname\enspace\ignorespaces#1}
\newcommand{\SD}{{\sc Symbo\-lic\-Data}}
\begin{document}

\mainmatter
%================================================


%==== FILL IN ====================================
\title{Semantic-aware Fingerprints\\ of Symbolic Research Data}  % Full title
\titlerunning{Semantic-aware Fingerprints of Symbolic Research Data} 
% Short title
\author{Hans-Gert Gr\"abe}
\authorrunning{Gr\"abe}
\institute{
Univ. Leipzig, Germany\\
\email{graebe@informatik.uni-leipzig.de},\\ 
\texttt{http://bis.informatik.uni-leipzig.de/HansGertGraebe}
}
\maketitle

\begin{abstract}
One of the goals of the {\SD} Project is to set up a navigational structure on
the research data associated with the project. In 2009 we started to refactor
the data and metadata along standard semantic web concepts based on the
Resource Description Framework (RDF) thus opening the door to the Linked Open
Data world.

One of the main metadata concepts used for navigational purposes is that of
\emph{semantic-aware fingerprints} as semantically sound invariants of the
given data.  We applied this principle, first used to navigate within
polynomial systems data, to the data sets on polytopes and on transitive groups
newly integrated with {\SD} version~3, and also within the recompiled version
of test sets from integer programming.  

The RDF based representation of fingerprints allows for a unified navi\-gation
and even cross navigation within such data using the SPARQL query mechanism as
a generic web service, a clear advantage compared to metadata management
traditionally in use within the domain of computer algebra.

In this paper we discuss merely the conceptual background of our
\emph{fingerprinting approach} and refer to the {\SD} wiki for more details and
examples how to use that service.

\keywords{semantic technology, RDF, computer algebra, metadata management,
  SPARQL query mechanism}
\end{abstract}

%------------------------------------------------------------
\section{Introduction}

The section ``Information Services for Mathematics'' addresses a more complex
target compared to the title ``Mathematical Software'' of this conference at
large since mathematical software can be considered as part of a whole
infrastructure for mathematical research. Nowadays such an infrastructure goes
much beyond the classically hawked ``paper and pencil'' or ``chalk and
blackboard'' claimed to be sufficient -- together with access to the work of
colleagues within an, nowadays also not self-evident, information and
communication infrastructure -- to pursue advanced mathematical research.  The
themes ``software, services, models, and data'' point to at least four
dimensions to enhance the mathematical research infrastructure in the era of
ubiquitous computing and increasingly important digital interconnectedness.

This paper addresses the dimension of \emph{research data} in more detail, in
particular aspects of public availability of reliable and well curated
\emph{research input data} that is important for the coherence of research
questions addressed by communities and thus for the formation of specific
research communities themselves. 

We discuss relevant questions in the specific context of intra- and
intercommunity communication within the specific research domain of
\emph{symbolic and algebraic computations} (CA) coarsely defined by the MSC
2010 classification number 68W30.  We analyze the situation of public
availability of research data in that area on the background of almost 20 years
of experience with research data management in that domain within the {\SD}
Project \cite{sdwiki}.  We address the special challenges to small scientific
communities as the CA community compared to larger ones as the whole
mathematical community, that nevertheless splits into a number of CA
subcommunities. These CA subcommunities are organized around special research
topics and in many cases already managed to organize and consolidate their own
intracommunity research infrastructures. We discuss lessons to be learned from
these activities and hurdles and obstructions to generalize such experience to
an intercommunity level within the CA domain.

In section~2 we develop a more detailed view on the interplay between (digital)
research data and research infrastructures and discuss the situation of the
mathematical digital research infrastructure compared to other sciences. 

In section~3 we give a short report about {\SD} activities in the CA domain
during the years. In particular we emphasize the importance of a redesign of
the {\SD} basics during the last years towards standard semantic web concepts
and the implementation of an RDF based infrastructure to manage descriptions
(``fingerprints'') of research data collections of different CA subcommunities
and thus to open them for the Linked Open Data world. 

Sections~4 and 5 are devoted to a more detailed explanation of the notion of
\emph{fingerprints} of research data and our conceptual background of data and
metadata management.  Further we discuss the advantages of an RDF based
approach to metadata management compared to approaches traditionally in use
within the CA domain.

%------------------------------------------------------------
\section{Research Data and Digital Research Infrastructures}

Digital change and the accelerated development of a (seemingly) universally
interconnected digital universe lead to an essential reshaping of many areas of
life.  Also the world of scientific research is affected by these mainly
technologically triggered social changes. The public availability and easy
accessibility of very detailed descriptions and information about research
processes leads to a strong increase of transparency and provides a basis for
completely new cooperation forms whose importance for the future hardly can
underestimated.

Such a development started to change research methods already in the
\emph{computer age} since the 1960th complementing established forms of
intermediation of scientific results by journal papers and preprints with
computer simulations and scientific software\footnote{\emph{Scientific
    software} is written to run \emph{computer simulations} -- we use this
  notion in an appropriate broad meaning -- and if software is not used in such
  a way it is of less academic interest. Moreover, computer simulations often
  require the interplay of several \emph{scientific packages} bundled within an
  \emph{application}, hence computer simulation is the broader notion and we
  use it throughout this paper instead of scientific software.} as an
essentially new form of scientific knowledge production.  Within the upcoming
\emph{networking age} a number of questions of scientific knowledge production
have to be addressed anew. Three themes related to simulation are of particular
importance: (1) the input data, (2) the simulation procedures (scientific
software) and (3) the output data.

Not only with the digital universe the free availability of data for public
research from each of these three thematic areas plays an important but
scientific-sociologically different role:
\begin{enumerate}
\item [(1)] The public availability of reliable and well curated \emph{input
  data} is important for the coherence of research questions addressed by the
  community and thus for the formation of a specific research community itself
  around its central research problems.
\item [(2)] The public availability of newly developed \emph{simulation
  methods, procedures and techniques} is relevant for the traceability of the
  proposed scientific approaches and increasingly accompanies classical forms
  of description of scientific advancement by academic papers.
\item [(3)] The public availability of \emph{output data} is important for the
  independent reproduction of results and thus of essential importance for the
  process of academic quality assurance.
\end{enumerate}
It is in the nature of the scientific process that output data is the starting
point for new research questions and thus output data mutates to input data. In
most of the cases such a mutation happens not immediately but is mediated by a
community-internal interpersonal transformation process that transforms the
often large output data (or a whole bundle of such data) into (one or several)
more compact input data adapted to the new research question(s).

Within the digital change the different scientific communities are faced with
the challenge to adapt their research and communication infrastructure to these
new socio-technical opportunities. Of central importance -- beside a culture of
public access -- is the allocation of resources for such a mainly non-academic
business to restructure this highly technical research infrastructure of the
community and keep it running. After many years of community-driven grassroot
activities of academic self-organization (e.g., ArXiv) this topic begins to
move into the focus of research and political administrations at different
levels and is reflected in different calls and rules at German wide (e.g.,
\cite{dfg-fd}) or EU level (e.g., \cite{h2020,esfri}).

Other scientific communities (e.g., with programs as TextGrid, DARIAH,
CLARIN-PLUS) act very successful to acquire EU funding to upgrade their
research data infrastructure mainly at the theme (1) level -- in particular to
set up a sustainable environment for text corpora (e.g., ``Deutsches
Textarchiv'') as the central research data form within Digital Humanities.  The
mathematical community is much less successful within the EU Research
Infrastructures Program \cite{h2020} (but see the OpenDreamKit Project
\cite{odk}) and concentrates with projects as swMath \cite{swmath}, sagemath
\cite{sagemath} and also this conference on the theme (2) level of sustainably
available scientific software.  Note that the application of the OpenDreamKit
Project was successful also due to the fact that it does \emph{not} address
mathematical software as such but \emph{successful cooperate practices} using
mathematical software.

Efforts to secure a research infrastructure for mathematical data at the theme
(1) or even theme (3) levels are lost in the brushwood of everlasting (for at
least a decade) debates about reliable formal but semantically expressive
formats as MathML or OpenMath for data resulting from calculi, that are already
highly formalized -- at least at an informal level -- by the internal nature of
the research topics themselves.  The situation reminds the Tower of Babel
Project, since subcommunities are digitally already well established, developed
their own formalizations for their own research data at theme (1) level and
apply such formalizations very successful within their intracommunity
communication processes.

%------------------------------------------------------------
\section{The {\SD} Project}

The {\SD} Project is a small project initiated at the end of the 1990th as an
intracommunity project in the area of \emph{Polynomial Systems Solving} to
secure a research data infrastructure at the theme (1) level built up within
the EU funded PoSSo \cite{PoSSo} and FRISCO \cite{FRISCO} projects. It grew up
from the Special Session on Benchmarking at the 1998 ISSAC conference in a
situation where the research infrastructure built up within these projects --
the Polynomial Systems Database -- was going to break down.  After the end of
the projects' fundings there was neither a commonly accepted process nor
dedicated resources to keep the data in a reliable, concise, sustainably and
digitally accessible way. Even within the ISSAC Special Session on Benchmarking
the community could not agree upon a further roadmap to advance that matter.

The {\SD} Project was set up by a small number of volunteers not involved
within the EU funded projects, but strongly interested in the public
availability of this research data as reference that can be used as input data
(1) for certified benchmark activities on specialized mathematical software
that was written to run simulations (2) in a special domain of Algebraic
Geometry. At those times almost 20 years ago most of the nowadays well
established concepts and standards for storage and representation of research
data did not yet exist -- even the first version of XML as a generic markup
standard had to be accepted by the W3C. It was Olaf Bachmann and me who
developed during 1999--2002 with strong support by the Singular group concepts,
tools and data structures for a structured representation and storage of this
data and prepared about 500 instances from \emph{Polynomial Systems Solving}
and \emph{Geometry Theorem Proving} to be available within this research
infrastructure, see \cite{Bachmann2000}.

The main conceptional goal was a nontechnical one -- to develop a research
infrastructure that is independent of (permanent) project funding but operates
based on overheads of its users. This approach was inspired by the rich
experience of the Open Culture movement ``business models'' to run
infrastructures. It was an early attempt to emphasize the advantage of an
explicitly elaborated concept of a community-based solution to the ``tragedy
of the commons'' \cite{hardin} within the CA community and to apply such a
concept to run a part of its research infrastructure.  

Even 15 years later it remains difficult to keep the {\SD} Project running on
such a base, and for many years we concentrate our efforts to secure the
sustainable public digital availability of the research input data within our
collections and to develop appropriate concepts and tools to manage, search and
filter this data.  In 2009 we started to refactor the data along standard
semantic web concepts based on the Resource Description Framework (RDF).  With
{\SD} version~3 released in September 2013 we completed a redesign of the data
along RDF based semantic technologies, set up a Virtuoso based RDF triple store
and an SPARQL endpoint as Open Data services along Linked Data standards, and
started both conceptual and practical work towards a semantic-aware Computer
Algebra Social Network \cite{cicm-14}.

Since then we continued that development.  On March 1, 2016, version 3.1 of
the {\SD} tools and data was released. The new release contains
\begin{itemize}
\item new resource descriptions (``fingerprints'') of remotely available data
  on transitive groups (\emph{Database for Number Fields} of Gunter Malle and
  J\"urgen Kl\"uners \cite{MalleKlueners}) and polytopes (databases of Andreas
  Paffenholz \cite{Paffenholz} within the \emph{polymake} project
  \cite{polymake}),
\item a recompiled and extended version of test sets from integer programming
  -- work by Tim R\"omer (\emph{normaliz} group \cite{normaliz}) --, 
\item an extended version of the \emph{SDEval benchmarking environment} -- work
  by Albert Heinle \cite{heinle-15} -- and
\item a partial integration ({\SD} People database, databases of upcoming and
  past conferences) of data from the Computer Algebra Social Network
  subproject.
\end{itemize}
Moreover, the github account \url{https://github.com/symbolicdata} was
transformed into an organizational account and the git repo structure was
redesigned better to reflect the special life-cycle requirements of the
different parts and activities within {\SD}. We provide the following repos
\begin{itemize}
\item \emph{data} -- the data repo with a single master branch mainly to backup
  recent versions of the data,
\item \emph{code} -- the code directory with master and develop branches,
\item \emph{maintenance} -- code chunks from different tasks and demos as best
  practice examples how to work with RDF based data,
\item \emph{publications} -- a backup store of the {\LaTeX} sources of {\SD}
  publications, 
\item \emph{web} -- an extended backup store of the {\SD} web site that
  provides useful code to learn how RDF based data can be presented.
\end{itemize}
The main development is coordinated within the \emph{{\SD} Core Team}
(Hans-Gert Gr\"abe, Ralf Hemmecke, Albert Heinle) with direct access to the
organizational account.  We refer to the {\SD} Wiki \cite{sdwiki} for more
details about the project's organization and the new release.

%-------------------------------------------
\section{Research Data and Metadata}

From the internal perspective of a research community a special aspect of every
research data collection is the design of management, search and filter
functionality.  For this purpose data is usually enriched with metadata that
collect important relevant information of the individual data records in a
compact manner.  We denote such metadata for an individual data record as its
\emph{fingerprint}.

Similar to a hash function a fingerprint function computes a compact metadata
record (\emph{resource description} in the RDF terminology) to each individual
data record (\emph{resource} in the RDF terminology).  As with a hash function
one can use the fingerprints to (almost) distinguish different data records
within the given collection and to match new records with given ones. But there
is an essential difference between (classical) hash functions and well designed
fingerprints: fingerprint functions exploit not only the textual representation
of the data record as meaningless syntactical character string but convey
semantically important information or even compute such information from the
string representation.  Fingerprints are in this sense \emph{semantic-aware}
and can even be designed in such a way that they map ambiguities in the textual
representation of records (e.g., polynomial systems given in different
polynomial orders and even in different variable sets) to \emph{semantic
  invariants}.

The design of appropriate fingerprint signatures is an important
\emph{intracommunity} activity to structure its own research data collections.
Such fingerprint signatures are also very useful for the \emph{intercommunity}
usage of research data collections, since they allow to navigate within the
(foreign) research data collection without presupposing the full knowledge of
the ``general nonsense'' of the target research domain, i.e., the informal
background knowledge required freely to navigate as scientist in that domain.
Hence well designed fingerprint signatures are to be considered also as a first
class service of a special research community to a wider audience to inspect
their research data collections without using the community-internal tools to
access the resources themselves.

%-------------------------------------------
\section{Working with Semantic-aware Fingerprints}

Usually the research data collections (resources in the RDF terminology) of a
certain community are stored in a specially designed community-internal format,
often as plain text (e.g., the Normaliz Collection \cite{normaliz}), in a
special XML notation (e.g., the Polymake Collection \cite{polymake}) or as SQL
database (e.g., the Database for Number Fields \cite{MalleKlueners}).  Such
formats usually employ special formal semantics agreed within the community as
an effective way to store domain specific input and output data and used by
commonly developed tools with appropriate parsing functionality.

Usually such formats are extended to store research metadata, i.e.,
fingerprints or \emph{resource descriptions} in the RDF terminology, together
with the research data. This has one benefit and two drawbacks:
\begin{itemize}
\item \emph{Benefit:} A fingerprint can be computed immediately by the commonly
  used tools or with their slight extension, and can be stored with the
  resource itself.
\item \emph{First Drawback:} Metadata unfold its full expressiveness only if
  one can search and navigate within it. A storage together with the resource
  itself implies high extraction costs for metadata navigation and access to
  the research data collection. 
\item \emph{Second Drawback:} The very different formats prevent an easy
  combination of metadata from different communities and even from different
  sources.
\end{itemize}
The first drawback can be addressed if the metadata are extracted into a
data\-base -- either a central one or delivered with the tools for local use --
and the commonly used intracommunity tools provide search and navigational
functionality within that metadata representation. Such an approach based on a
web interface was realized for the \emph{Database for Number Fields}
\cite{MalleKlueners} and a tool integration based on a Mongo-DB for the
\emph{Polymake Database} \cite{polymake}.  But such a solution has two further
drawbacks:
\begin{itemize}
\item \emph{Drawback 1a:} The search and navigational functionality is not or
  only in a restricted way adapted for machine-readable interaction and thus
  cannot be integrated into more comprehensive search and navigational
  processes.
\item \emph{Drawback 1b:} The search and navigational functionality can't be
  adapted by the user for its own needs.
\end{itemize}
A general solution that avoids these drawbacks proposes to extract the metadata
information from the resource data and to transform it into RDF.  RDF -- the
Resource Description Framework -- is the conceptual basis of Linked Open Data
as a worldwide distributed database that can be globally queried and
navi\-gated using the SPARQL query language in a similar unified way as SQL
allows to navigate in local relational databases.
\medskip

We applied this approach, first used within the {\SD} Project to navigate
within polynomial systems data, to the data sets on polytopes and on transitive
groups newly integrated with {\SD} version~3, and also within the recompiled
version of test sets from integer programming.  We store these fingerprints in
our RDF data store \cite{sdstore} thus allowing for a unified navigation and
even cross navigation within such data using the SPARQL query mechanism as a
generic Web service provided by our SPARQL endpoint \cite{sdsparql}.  

We refer to the {\SD} wiki \cite{sdwiki} for detailed information and examples
how to use that service.

%------------------------------------------------------------
\begin{thebibliography}{4}\raggedright
\bibitem{Bachmann2000} Bachmann, O., Gr\"abe, H.-G. : The SymbolicData Project
  -- Towards an Electronic Repository of Tools and Data for Benchmarks of
  Computer Algebra Software. Reports on Computer Algebra 27 (2000), Centre for
  Computer Algebra, University of Kaiserslautern.
\bibitem{normaliz} Bruns, W., Ichim, B., R\"omer. T., Sieg, R., S\"oger, C.:
  Normaliz. Algorithms for Rational Cones and Affine Monoids.
  \url{https://www.normaliz.uni-osnabrueck.de}. [2016-03-08]
\bibitem{dfg-fd} DFG verabschiedet Leitlinien zum Umgang mit Forschungsdaten.
  DFG-Magazin ``Information f\"ur die Wissenschaft'' Nr. 66 (2015).
  \url{http://www.dfg.de/foerderung/info_wissenschaft/2015/info_wissenschaft_15_66/index.html}.
      [2016-05-07]
\bibitem{esfri} Strategy Report on Research Infrastructures.  Roadmap 2016.
  Published by the European Strategy Forum for Research Infrastructures
  (ESFRI), Br\"ussel (2016).  \url{http://www.esfri.eu/roadmap-2016}.
  [2016-03-16]
\bibitem{FRISCO} FRISCO -- A Framework for Integrated Symbolic/Numeric
  Computation. (1996--1999).  \url{http://www.nag.co.uk/projects/FRISCO.html}.
  [2016-02-19]
\bibitem{polymake} Gawrilow, E., Joswig, M.: Polymake: a Framework for
  Analyzing Convex Polytopes. In: Kalai, G., Ziegler, G.M. (eds.), Polytopes --
  Combinatorics and Computation (Oberwolfach, 1997), pp. 43--73, DMV Sem., 29,
  Birkh\"auser, Basel (2000). 
\bibitem{cicm-14} Gr\"abe, H.-G., Johanning, S., Nareike, A.: The {\SD} Project
  -- Towards a Computer Algebra Social Network. In: Workshop and Work in
  Progress Papers at CICM 2014, CEUR-WS.org, vol. 1186 (2014).
\bibitem{hardin} Hardin, G.: The Tragedy of the Commons. Science 162 (3859),
  pp. 1243--1248 (1968). \url{doi:10.1126/science.162.3859.1243}. 
\bibitem{heinle-15} Heinle, A., Levandovskyy, V.: The SDEval Benchmarking
  Toolkit. ACM Communications in Computer Algebra, vol. 49.1, pp. 1--10 (2015).
\bibitem{MalleKlueners} Kl\"uners, J., Malle, G.: A Database for Number Fields.
  \url{http://galoisdb.math.uni-paderborn.de/}. [2016-03-08]
\bibitem{odk} OpenDreamKit: Open Digital Research Environment Toolkit for the
  Advancement of Mathematics. \url{http://opendreamkit.org/},
  \url{http://cordis.europa.eu/project/rcn/198334_en.html}. [2016-03-16]
\bibitem{Paffenholz} Paffenholz, A.: Polytope Database.
  \url{http://www.mathematik.tu-darmstadt.de/~paffenholz/data/}.  [2016-03-08] 
\bibitem{PoSSo} The PoSSo Project. Polynomial Systems Solving -- ESPRIT III BRA
  6846.  (1992--1995).
\bibitem{h2020} Research Infrastructures, including e-Infrastructures.
  \url{http://ec.europa.eu/programmes/horizon2020/en/h2020-section/research-infrastructures-including-e-infrastructures}. [2016-03-16]
\bibitem{sagemath} The SageMath Project.  \url{http://www.sagemath.org/}.
  [2016-03-16]
\bibitem{sdstore} The {\SD} RDF Data Store.
  \url{http://symbolicdata.org/Data}.  [2016-03-15]
\bibitem{sdsparql} The {\SD} SPARQL Endpoint.
  \url{http://symbolicdata.org:8890/sparql}. [2016-02-19]
\bibitem{sdwiki} The {\SD} Project Wiki.
   \url{http://wiki.symbolicdata.org}. [2016-03-13]
\bibitem{swmath} swMATH -- a new Information Service for Mathematical Software.
  \url{http://www.swmath.org/}. [2016-03-07]
\end{thebibliography}
\end{document}
