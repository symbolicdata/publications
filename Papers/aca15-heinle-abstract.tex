% Created 2015-05-11 Mo 10:26
\documentclass[11pt]{article}
\usepackage[utf8]{inputenc}
\usepackage[T1]{fontenc}
\usepackage{graphicx}
\usepackage{longtable}
\usepackage{float}
\usepackage{wrapfig}
\usepackage{soul}
\usepackage{amssymb}
\usepackage{hyperref}


\title{\textsc{SDEval} -- A Benchmarking Toolkit for Computer Algebra}
\author{Hans-Gert Graebe, Albert Heinle, Viktor Levandovskyy}
\date{2015-05-08}

\begin{document}

\maketitle

\setcounter{tocdepth}{3}

\section{Abstract}
\label{sec-1}

Since more than a decade, the computer algebra community 
collects interesting examples for certain algorithmic problems in several
databases (\cite{grabe2006symbolicdata,grabe2013symbolicdata,polyDB_paffenholz}). These examples appeared either in
the literature, or were taken from practical applications.

A fair and reproducible comparison of different computer
algebra systems solving these problems on given inputs is --
unfortunately -- not yet common practice.
With \textsc{SDEval} \cite{HeinleLev2015}, we aim at changing this. \textsc{SDEval} provides tools to translate
examples from Symbolic Data into executable code for different computer algebra
systems, and brings mechanisms to run and monitor
their computations. Furthermore, it makes it easy for other researchers to
run these computations themselves and verify them.

\textsc{SDEval} is designed to be flexible enough and easy to extend to fit the
needs of the various distinct areas of computer algebra. We intend to
present the main principles of \textsc{SDEval} on our poster, as well its
applicability across different communities. We also show an overview
of different use-cases of the tools in \textsc{SDEval}.

\section{Resources}
\label{sec-2}


\textsc{SDEval} is free software and is part of the Symbolic-Data
distribution. One can obtain it from
\begin{center}
\href{https://github.com/symbolicdata/symbolicdata}{https://github.com/symbolicdata/symbolicdata}
\end{center}
The latest developments can be found in this fork:
\begin{center}
\href{https://github.com/ioah86/symbolicdata}{https://github.com/ioah86/symbolicdata}
\end{center}

General information on the Symbolic Data project and \textsc{SDEval}, as well as the
latest developments are stated on the following website:
\begin{center}
\href{http://www.symbolicdata.org}{http://www.symbolicdata.org}
\end{center}
There is also a video tutorial on how to use \textsc{SDEval}. One can find it on
\begin{center}
\href{https://www.youtube.com/watch?v=CctmrfisZso}{https://www.youtube.com/watch?v=CctmrfisZso}
\end{center}

\bibliographystyle{apalike}
%%\bibliography{aca_15_poster_abstract}
\begin{thebibliography}{}

\bibitem[Gr{\"a}be, 2009]{grabe2006symbolicdata}
Gr{\"a}be, H.-G. (2009).
\newblock The {\sc Symbolic Data} Project.
\newblock Technical report (2000-2009).

\bibitem[Gr{\"a}be et~al., 2013]{grabe2013symbolicdata}
Gr{\"a}be, H.-G., Nareike, A., and Johanning, S.
\newblock The {\sc Symbolic Data} Project--Towards a Computer Algebra Social
  Network.
\newblock {\em Workshop and Work in Progress Papers at CICM 2014 in
CEUR-WS.org} vol. 1186 (2014)

\bibitem[Heinle and Levandovskyy, 2015]{HeinleLev2015}
Heinle, A. and Levandovskyy, V. (2015).
\newblock The \textsc{SDEval} Benchmarking Toolkit.
\newblock {\em ACM Commun. Comput. Algebra}, 49(1):1--9.
\newblock DOI \url{http://doi.acm.org/10.1145/2768577.2768578}

\bibitem[Paffenholz, 2015]{polyDB_paffenholz}
Paffenholz, A. (2015).
\newblock Polytope Database.
\newblock URL \url{http://www.mathematik.tu-darmstadt.de/\~paffenholz/data.html}
\end{thebibliography}

\end{document}
