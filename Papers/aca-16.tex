%    Latex Template for ACA 2016
%%%%%%%%%%%%%%%%%%%%%%%%%%%%%%%%%%%%%%%%
%    DO NOT CHANGE THE SETTINGS BELOW
%%%%%%%%%%%%%%%%%%%%%%%%%%%%%%%%%%%%%%%%
\documentclass[11pt]{article}
\usepackage{epsfig}
\usepackage{graphicx}
\usepackage[latin1]{inputenc}
\usepackage[T1]{fontenc}
\usepackage{mathptmx}
\usepackage{url}

\begin{document}

%%%%%%%%%%%%%%%%%%%%%%%%
%%%%%%%%%%%%%%%%%%%%%%%%%%%%%%%%%%%%%%%%
%    DO NOT CHANGE THE SETTINGS ABOVE
%%%%%%%%%%%%%%%%%%%%%%%%%%%%%%%%%%%%%%%%
%
\vspace{-0.8cm}
\begin{flushleft}
\Large \textbf{\noindent
%%% PLEASE INSERT THE TITLE OF YOUR TALK HERE %%%%%%%
The \textsc{SymbolicData} Project -- a Community Driven Project for the CA
Community}
%%%%%%%%%%%%%%%%%%%%%%%%%%%%%%%%%%%%%%%%%%%%%%%%%%%%%%%%
\\
%%%%%%%%%%%%%%%%%%%%%%%%%%%%%%
\vspace{0.5cm}
\normalsize
 %%%%%%             NAMES OF AUTHORS         %%%%%%
 %%%%%% NAME OF SPEAKER SHOULD BE UNDERLINED %%%%%%%%%
\normalsize{
 %%%%%% FOR EXAMPLE %%%%%
 \underline{H.-G. Gr\"abe}
 %%%%%%%%%%%%%%%%%%%%%%%%
} \\
\vspace{5mm}
\textit{\footnotesize
 %%%%%% AFFILIATION OF AUTHORS %%%%%%
Leipzig University, Germany, graebe@informatik.uni-leipzig.de\\
 %%%%%%%%%%%%%%%%%%%%%%%%%%%%%%%%%%%%%
}
\end{flushleft}
\newcommand{\SD}{{\sc Symbo\-lic\-Data}}

\section{Introduction}

A central phenomenon of the emerging digital age is the increasing importance
of a sustainably and reliably available \emph{digital interconnection
  infrastructure} for many areas of every day life. This distinguishes the
digital age from the computer age that focused on penetration of every day
life with \emph{compute power} rather than interconnectedness.  With
``ubiquitous computing'' such a penetration with compute power reached a high
level of saturation, but is in no way at its end as is demonstrated by the
development of modern sensor and actor systems as ``cyber-physical systems''
and its applications within ``industry 4.0''. 

During the last years the sensibility to the importance of investments also
into a modern digital research infrastructure remarkably increased. It was
continuously discussed on the ``big'' stage of research politics between
different stakeholders, see e.g., \cite{h2020,esfri}.  The disposition to
invest into the development of an appropriate digital infrastructure heavily
depends on the visibility of the demand, whereas the demand develops with the
productiveness of the available infrastructure -- a typical chicken-and-egg
problem, that can only be addressed in the socio-technical context of a
problem-aware community.  Such a community should have a good understanding of
the importance of the advancement of its own research infrastructure and the
ability to set up a socio-communicative process to coordinate the development
of its own demands \emph{and} activities in the desired direction.

Digital infrastructures are not only well suited to exchange research data and
make it publicly available, but also proved valuable as technical basis of
``social networks'' to promote such socio-communicative coordination processes.
Nowadays in many cases different channels and means are used for these
purposes, but it is due time to combine conceptually and also in practice both
aspects of a research infrastructure.

With the advancement of the {\SD} Project towards a Computer Algebra Social
Network (CASN) we pursued such a concept in a specific context for several
years.  We started to investigate questions of intra- and intercommunity
communication in correlation with practical aspects of the community driven
development of a decentrally organized, distributed semantic-aware digital
research infrastructure within the specific research domain of \emph{symbolic
  and algebraic computations} (CA) coarsely defined by the MSC 2010
classification code 68W13 -- a medium sized scientific community, that splits
into a number of subcommunities.  These CA subcommunities are organized around
special research topics and in many cases already managed to organize and
consolidate their own intracommunity digital research infrastructures.

In our talk we address relevant questions, observations, and experience of our
endeavor to develop and provide technical means to support the emergence of a
digital research infrastructure on the intercommunity level.  We discuss
lessons to be learned from these activities and hurdles and obstructions to
generalize intracommunity experience to an intercommunity level within the CA
domain.  

We propose to deploy a special RDF-based architecture of \emph{CASN nodes}
operated by different CA subcommunities and CA groups along the rules of the
Linked Open Data Cloud \cite{lod}.  To ensure interoperability, this should be
accompanied by a strong social intercommunity communication process to develop
a common \emph{data architecture} of data models and its ontological standards
of representation based on well established semantic web concepts and using
standard semantic web technology.

\section{The {\SD} Project as Community Project}
The allocation of resources for a sustainably available research infrastructure
seems to be a great challenge in particular to smaller scientific communities.
The {\SD} Project witnesses the peaks and troughs of such efforts. It grew up
from the Special Session on Benchmarking at the 1998 ISSAC conference in a
situation where the research infrastructure built up within the PoSSo
\cite{PoSSo} and FRISCO \cite{FRISCO} projects -- the Polynomial Systems
Database -- was going to break down. After the end of the projects' fundings
there was neither a commonly accepted process nor dedicated resources to keep
the data in a reliable, concise, sustainably and digitally accessible way. Even
within the ISSAC Special Session on Benchmarking the community could not agree
upon a further roadmap to advance that matter.

At those times almost 20 years ago most of the nowadays well established
concepts and standards for storage and representation of research data did not
yet exist -- even the first version of XML as a generic markup standard had to
be accepted by the W3C. It was Olaf Bachmann and me who developed during
1999--2002 with strong support by the Singular group concepts, tools and data
structures for a structured representation and storage of this data and
prepared about 500 instances from \emph{Polynomial Systems Solving} and
\emph{Geometry Theorem Proving} to be available within this research
infrastructure, see \cite{Bachmann2000}.

The main conceptual goal was a nontechnical one -- to develop a research
infrastructure that is independent of (permanent) project funding but operates
based on overheads of its users. This approach was inspired by the rich
experience of the Open Culture movement ``business models'' to run
infrastructures.  During the last ten years with Open Access, Open Data and the
emerging semantic web the general understanding of the importance of such
community-based efforts to develop common research infrastructures matured.
This development was accompanied with conceptual, technological and
architectural standardization processes that had also impact on the development
of concepts and data structures within the {\SD} Project.

In 2009 we started to refactor the data along standard Semantic Web concepts
based on the Resource Description Framework (RDF).  With {\SD} version~3
released in September 2013 we completed a redesign of the data along RDF based
semantic technologies, set up a Virtuoso based RDF triple store and an SPARQL
endpoint as Open Data services along Linked Data standards \cite{lod}, and
started both conceptual and practical work towards a semantic-aware Computer
Algebra Social Network \cite{cicm-14}.

In March 2016 version 3.1 of the {\SD} tools and data was released. On the
level of research tools and data the new release contains new resource
descriptions (``fingerprints'' in the notion of \cite{cicm-14}) of remotely
available data on transitive groups (\emph{Database for Number Fields} of
Gunter Malle and J\"urgen Kl\"uners \cite{MalleKlueners}) and polytopes
(databases of Andreas Paffenholz \cite{Paffenholz} within the \emph{polymake}
project \cite{polymake}), a recompiled and extended version of test sets from
integer programming -- work by Tim R\"omer (\emph{normaliz} group
\cite{normaliz}) -- and an extended version of the \emph{SDEval benchmarking
  environment} -- work by Albert Heinle \cite{heinle-15}.

The main development is coordinated within the \emph{{\SD} Core Team}
(Hans-Gert Gr\"abe, Ralf Hemmecke, Albert Heinle) with direct access to our
public github account \url{https://github.com/symbolicdata}.  We refer to the
{\SD} Wiki \cite{sdwiki} for more details about the project and the new
release.

\section{The CA Community and its Subcommunities}

During the last years the {\SD} Project adjusted its focus to address more
general technical and social aspects of a semantically enriched research
infrastructure within the domain of Computer Algebra based on RDF for
representation of intercommunity and relational information.  Such a change of
the focus had its impact on several earlier design decisions of the data store
itself. 

Enlarging the database of {\SD} we gained the following experience:
\begin{itemize}
\item The CA community consists of several subcommunities with own concepts,
  notational conventions, semantic-aware tools and established communication
  structures.  

  There is no need to duplicate such structures but to support the
  subcommunities to enrich semantically these communication processes.
\item We provide structural metadata (``fingerprints'' in the notion of
  \cite{cicm-14}) of the different data sets at our central RDF store but not
  necessarily duplicate the data itself.

  Thus we rely on sustainably available research infrastructures of CA
  subcommunities and restrict our activities to a central search and filter
  service on the metadata level to find and identify data. This service is
  based on a generic semantic web concept, the SPARQL query language, and can
  be operated via our SPARQL endpoint.
\item RDF is a useful and meanwhile well established standard for metadata and
  relational information, but there is no need and one cannot expect from CA
  subcommunities to give up established notational conventions in favor of
  RDF or XML markup for their primary sources. 
\end{itemize}

\section{About the CASN Architecture}

The CASN subproject tries to embed aspects of the maintenance of the {\SD}
data store into a more general process of formation of a semantically enriched
social network of academic communication within the CA community in the sense
of a (social) ``web of people''.

A first roadmap towards such a CASN and our experimental setting was described
in \cite{cicm-14} and developed further during the last years.  We try not to
``reinvent the wheel'' but to address step by step the already existing ``CA
memory'' -- a huge number of very loosely related web pages about conferences,
meetings, working groups, projects, private and public repositories, private
and public mailing lists etc. Hence the main focus towards CASN is to develop a
framework based on modern semantic technologies for a decentralized network
that increases the awareness of the different parts of that already existing
``CA network''.

We realized that this network itself is an ``overlay network'' that connects a
greater number of research networks of individuals around special topics with
own lightweight research infrastructures.  It is an interesting challenge for
semantic concepts to support the requirements of intercommunity communication
to exchange semantic content on different levels and different levels of
detail.

As a coarse architectural concept to establish such a network we propose
\begin{itemize}
\item to operate a central RDF store with SPARQL endpoint providing the full
  bandwidth of Linked Open Data services and
\item to convert nodes of the ``CA memory'' into CASN nodes providing part of
  their data in structured RDF format for easy access and exchange.
\end{itemize}
{\SD} version 3.1 is a first step in that direction since several data from the
formerly separate CASN RDF store are now integrated within the {\SD} main RDF
store and the experimental setting of the semantic support of the website of
the German Fachgruppe \cite{cafg} was reorganized as a first CASN node.

%%%%%%%%%%%%%%%%%%%%%%%%%%%%%%%%%%%%%%%%
%% BIBLIOGRAPHY
%%%%%%%%%%%%%%%%%%%%%%%%%%%%%%%%%%%%%%%%
\raggedright
\begin{thebibliography}{00}
\addcontentsline{toc}{chapter}{References}
\setlength{\itemsep}{-1mm}
\footnotesize
\bibitem{Bachmann2000} O. Bachmann, H.-G. Gr\"abe: The SymbolicData Project --
  Towards an Electronic Repository of Tools and Data for Benchmarks of Computer
  Algebra Software. Reports on Computer Algebra 27 (2000), Centre for Computer
  Algebra, University of Kaiserslautern.
\bibitem{normaliz} W. Bruns, B. Ichim, T. R\"omer, R. Sieg, C. S\"oger:
  Normaliz.  Algorithms for Rational Cones and Affine Monoids.
  \url{https://www.normaliz.uni-osnabrueck.de}. [2016-03-08]
\bibitem{FRISCO} FRISCO -- A Framework for Integrated Symbolic/Numeric
  Computation. (1996--1999).  \url{http://www.nag.co.uk/projects/FRISCO.html}.
  [2016-02-19]
\bibitem{polymake} E. Gawrilow, M. Joswig: Polymake: a Framework for Analyzing
  Convex Polytopes. In: G. Kalai, G.M. Ziegler (eds.), Polytopes --
  Combinatorics and Computation (Oberwolfach, 1997), pp. 43--73, DMV Sem., 29,
  Birkh\"auser, Basel (2000).
\bibitem{cicm-14} H.-G. Gr\"abe, S. Johanning, A. Nareike: The {\SD} Project --
  Towards a Computer Algebra Social Network. In: Workshop and Work in Progress
  Papers at CICM 2014, CEUR-WS.org, vol. 1186 (2014).
\bibitem{heinle-15} A. Heinle, V. Levandovskyy: The SDEval Benchmarking
  Toolkit. ACM Communications in Computer Algebra, vol. 49.1, pp. 1--10 (2015).
\bibitem{MalleKlueners} J. Kl\"uners, G. Malle: A Database for Number Fields.
  \url{http://galoisdb.math.uni-paderborn.de/}. [2016-03-08]
\bibitem{lod} The Linked Open Data Cloud.  \url{http://lod-cloud.net/}.
  [2016-05-20]
\bibitem{Paffenholz} A. Paffenholz: Polytope Database.
  \url{http://www.mathematik.tu-darmstadt.de/~paffenholz/data/}.  [2016-03-08]
\bibitem{PoSSo} The PoSSo Project. Polynomial Systems Solving -- ESPRIT III BRA
  6846.  (1992--1995).
  \url{http://research.cs.ncl.ac.uk/cabernet/www.laas.research.ec.org/esp-syn/text/6846.html}. 
      [2016-03-16]
\bibitem{h2020} Research Infrastructures, including e-Infrastructures.
  \url{http://ec.europa.eu/programmes/horizon2020/en/h2020-section/research-infrastructures-including-e-infrastructures}. [2016-03-16]
\bibitem{esfri} Strategy Report on Research Infrastructures.  Roadmap 2016.
  Published by the European Strategy Forum for Research Infrastructures
  (ESFRI), Br\"ussel (2016).  \url{http://www.esfri.eu/roadmap-2016}.
  [2016-03-16]
\bibitem{sdwiki} The {\SD} Project.  \url{http://wiki.symbolicdata.org}.
  [2016-05-20]
\bibitem{cafg} Website of the German Fachgruppe Computeralgebra.   
  \url{http://www.fachgruppe-computeralgebra.de/}. [2016-03-06]
\end{thebibliography}

\end{document}

Use this format (without changing the settings) to prepare your abstract. The
abstracts are used for evaluation purposes. Therefore, it is important that you
describe your intended presentation concisely and clearly, including the
objectives, status, methodology, results, and significance of your study.

The extended abstract length is limited to 5  pages.  Equations are to be centered horizontally, with equation numbers aligned with the right margin.  Paragraph text is full-justified.  Do not explicitly begin by writing the word ``abstract''.  Equations should be properly punctuated.  For instance, as
\begin{equation}
y=3x^2\  \  \  \mbox{and}\  \  \  x>0, \label{eq1}
\end{equation}
we see that
\begin{equation}
z=\sqrt{y}=\sqrt{3}x. \label{eq2}
\end{equation}

%\begin{eqnarray}
%\lim_{x \to +\infty} {\rm Abstract}(x) = 1\ {\rm page}.
%\label{eq1}
%\end{eqnarray}


Your abstract should be prepared in \LaTeX\ according to the template provided
and then \textbf{converted to PDF for submission.}   Abstracts not complying
with this template may not be included in the book of abstracts of the ACA
2015. Abstract submission should be done  via the session organizers, they will
confirm your submission. If your abstract is accepted, you will be invited to
present your talk.

The book of abstracts will be available to the participants prior to the
conference and will provide the audience with more information about the papers
being presented at the ACA 2015 Workshop. At least one author must be
registered for your talk to be included in the program of the conference.
