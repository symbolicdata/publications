\documentclass[a4paper,11pt]{article}
\usepackage{CAR}

\begin{document}
%\setcounter{footnote}{0}
%\setcounter{figure}{0}

%%%%%%%%%%%%%%%%%%%%%%%%%%%%%%%%%%%%%%%%%%%%%%%%%%%%%%%%%%%%%%%%%%%%%%%%%%%%%%%%

%\Abschnitt
% Name der Rubrik
%{Rubrik}
% Name der Rubrik (Inhaltsverzeichnis)
%{Rubrik}
% Label fuer die Rubrik
%{rubrik}

%\vspace{3mm}

%%%%%%%%%%%%%%%%%%%%%%%%%%%%%%%%%%%%%%%%%%%%%%%%%%%%%%%%%%%%%%%%%%%%%%%%%%%%%%%%%
%
%\Aufsatz
% Titel des Aufsatzes
%{Some steps to improve software information}
% Titel des Aufsatzes (Inhaltsverzeichnis)
%{Titel}
% Autor
%{N.\ Name}
% Abkuerzung des Namens als Label
%{autor}
% Autor mit Adresse
%{N.\ Name\\(Universit"at ........)}
% Dateiname eines Bildes
%{nn}
% E-Mail
%{@email.de}
%
%\iffalse

%%%%%%%%%%%%%%%%%%%%%%%%%%%%%%
% Falls es drei Autoren gibt:
%%%%%%%%%%%%%%%%%%%%%%%%%%%%%%

\Aufsatz
% Titel des Aufsatzes (hier)
{Some steps to improve software information}
% Titel des Aufsatzes (Inhaltsverzeichnis)
{Some steps to improve software information}
% Autoren
{Albert Heinle,  Wolfram Koepf, Wolfram Sperber}
% Abkuerzung der Namen als Label
{autor}
% Autoren mit Adressen
 {Albert Heinle, University of Waterloo \\
  Wolfram Koepf, Universit\"at Kassel \\
  Wolfram Sperber, FIZ Karlsruhe}
% Dateinamen der Bilder
{rubrik_themenanwendungen/Sperber-Final/acaheinle.jpg,rubrik_themenanwendungen/Sperber-Final/acakoepf.jpg,rubrik_themenanwendungen/Sperber-Final/acasperber.png}
% E-Mails
{aheinle@uwaterloo.ca,koepf@mathematik.uni-kassel.de,wolfram@zbmath.org}


%\fi

\vspace{3mm}

\begin{multicols}{2}
\noindent

%%%%%%%%%%%%%%%%%%%%%%%%%%%%%%%%%%%%%%%%%%%%%%%%%%%%%%%%%%%%%%%%%%%%%%%%%%%%%%%%

% Ueberschriften fuer Abschnitte im Aufsatz:
\Ueberschrift
% Ueberschrift
{Introduction}
% Label
{sec_1}
Computer Algebra Systems (CASs) and their use in research are a critical part of modern mathematics and more\-over for a broad class of applications. More generally, scientific software has established itself as an autonomous kind of scientific research. But the existing scientific infrastructure is focussed on articles and books and does not support the information about software optimally. Also, methods for the evaluation and quality control of scientific software, in particular computer algebra software, must be developed and the development of scientific software should be given the same credit and reputation as it is given to other research results.

Citations play an essential role for identifying resources, as they help to track and weight the development and the realization of ideas and theories. Furthermore, the evaluation of research results, improvement of visibility of the cited resources and proper credits to authors are the origin for developing information services. The first section describes recommendations for a better citation practice for software.

Scientific software development and software information are widely distributed which complicates the integration of scientific software into existing infrastructures in an adequate way. In section 2 we give a brief overview about existing information resources, their \mbox{role}, and some problems in scientific software information services in mathematics. At the top, portal and catalogues are a first contact point to find software and information about it. There exist a lot of portals, also for symbolic computation, but manual maintenance and updating the information is expensive.  The swMATH service---for a brief description of this system see section 4---is an attempt to create a comprehensive information service  on mathematical software in a (nearly) automatic way. Therefore the close relationships between software and publications are essentially used.  But the context of mathematical software is broader, it covers also algorithms, data, languages, people, communities, institutions, etc.  The idea of a social network for the symbolic computation community is introduced in section 5. It is based on semantic web technologies allowing to maintain the information in distributed resources. It is one of the aims of this paper to sensitize the symbolic computation community to the questions and problems of a suitable scientific infrastructure for this mathematical subject. We hope that it will lead to an intensive discussion on these addressed problems within the community.

\Ueberschrift
% Ueberschrift
{Software Citations}
% Label
{sec_2}
Publications---as the classical products of scientific research---use citations for embedding the content of a publication into its scientific context. Citations of publications are---at least in principle---standardized. But this is not at all true for software citations up until now. This results from the character of software. Software is dynamic and has a life cycle, has often different versions, releases, or bug fixes, etc., which prompt questions of archiving, reproducibility, and sustainability.  Software is written in a formal language and is accompanied by documentation, manuals, metadata files, etc. The line between software and other types of research, especially algorithms, languages, environments and services is fuzzy. Software is dependent on hardware, the operating system and other software; licenses and usability conditions for software are varying. A metadata standard for software is missing. Mathematical software is closely linked to models and mathematical objects, theories and algorithms.

With respect to the use of computer algebra systems and packages within these systems the missing citation standard leads to the absurd situation that in most cases citation is just ``turned off'', hence the author of the software is not cited at all. Users of such systems who get results that they use in their papers often are not aware of whom they should cite and how they should do this properly.

Recently, the increasing importance of software expresses itself in more software citations in scientific publications. But these citations often provide not enough information about the software used.  Fig.~1 gives an example for the citation of the ``Singular'' software in a publication.

% Abbildungen
\begin{figurehere}
 \centering
 \includegraphics[width=1.0\columnwidth]{rubrik_themenanwendungen/Sperber-Final/aca1.jpg}
 \caption{An example of a typical software citation.\label{abb_1}}
\end{figurehere}

Such a citation practice for software is more or less typical not only in mathematics. Howison and Bullard \cite{Howison&Bullard2015} analyzed nearly 300 software references in biology:

% Abbildungen
\begin{figurehere}
 \centering
 \includegraphics[width=0.9\columnwidth]{rubrik_themenanwendungen/Sperber-Final/aca2.jpg}
 \caption{Software citing in biology.\label{abb_2}}
\end{figurehere}

A lot of initiatives, being run by e.g., software companies, publishers, or repositories have discussed and developed proprietary recommendations for software citations. Mike Jackson  has given in his blog \cite{Jackson} a detailed state of the art analysis and pitfalls of software citation and recommendations for a better citation practice. Currently, the FORCE 11 Software Citation Working group \cite{SoftwareCitationWG}, an international initiative of more than 50 information experts from different scientific areas, has discussed basic concepts for software citation. As one result the group has published the Software Citation Principles (SCPs) \cite{SoftwareCitationPrinciples}. They emphasize that software is a legitimate product of research and therefore must be citable.  The six principles address the importance of software within research which should manifest  that all relevant software will be cited, that software citations should facilitate giving credit and attribution to the developers and contributors of software, include methods for a  unique identification, refer to persistent information about software, should facilitate access to the software, and provide accurate information about software (e.g., the version used). The SCPs define a general frame for software citations and moreover formulate principles for  maintaining of information about software.

The SCPs do not discuss the realization of the principles.  This is planned to be subject for a follow-up working group.\par

Exact information about software---together with the data used---is a necessary condition to evaluate and reproduce scientific results which were achieved by using mathematical software. Therefore, for the citation format, the following recommendation is given: ``We recommend that all text citation styles support the following: a) a label indicating that this is software, e.g., [software], potentially with more information such as [Software: Source Code], [Software: Executable], or [Software: Container], and b) support for version information, e.g., Version 1.8.7'' \cite{SoftwareCitationPrinciples}.  Each researcher who uses a software for research and publishes her or his results (e.g., in form of a paper, software or data file) is recommended to cite software according to this recommendation.

The mathematical community uses the {\TeX} format for publishing. References are encoded in the (outdated) Bib{\TeX} format or, more up to date, in the  Bib{\LaTeX} format. Actually, neither Bib{\TeX} nor Bib{\LaTeX}  support a type ``software''. Up to now, the Bib{\LaTeX} standard contains no special document type for software. Software must be typed as ``misc''. A pragmatic recommendation is that it should be added to the title if a citation refers to a software, together with detailed information about the software instance. This could be done in the following form: title [Software:special type (Source Code, Package, Executable, Library, Other)] [Version or release or URL and/or date of the update and/or date of the download]. This would be a first step to a better software citation practice and provides the required information for the human user.\\
A more rigorous and Semantic Web compatible solution would be an extension of Bib{\LaTeX} standard. Bib{\LaTeX} together with the backend software Biber provides the opportunity to define new document types, e.g., software and the corresponding fields.  A prototype for a {\TeX} implementation is under development within the framework of the swMATH activities. It is planned to provide a template for  {\TeX} encoding of software citations.

The Bib{\LaTeX} encoding allows also a simple transformation to other formats, e.g., JSON, which can be used for a machine-based semantic processing of software citations.

Comment: Also software which cites another software should contain the corresponding notations. This could be done by separate citation files which are  encoded in the same form as for papers, e.g., in Bib\LaTeX.

The SCPs recommend that ``the software itself should be cited on the same base as any other research product'', and should have a unique and persistent identifier, preferably a DOI. This does not mean that the DOI is assigned to the software code. Outdated software is often removed from the Web. Instead, ``the software identifier should resolve to a persistent landing page that contain metadata and a link to the software itself, rather than directly to the source code files, repository, or executable'', \cite{SoftwareCitationPrinciples}.  The problem of persistent identifiers and landing pages, especially of a DOI, is connected with additional efforts. Up to now, the existing landing pages, e.g.,  portals and software directories, provide only metadata about families of software which is offered under the same name, not about versions. But it seems to make sense that---similar to publications---persistent identifiers should be provided and maintained by special information services which integrate the information about software in a subject and make it available.
The SCPs make clear that a citation standard would be very helpful for better software information but it is only a building block in a better digital information infrastructure for software and scientific information in general. That is why we continue with a brief description about resources which are relevant for mathematical software information.

% Ueberschriften fuer Abschnitte im Aufsatz:
\Ueberschrift
% Ueberschrift
{The landscape of mathematical software information in the Web}
% Label
{sec_3}
The landscape of Web resources of mathematical software information is heterogeneous, widely distributed, and has different layers.  Here is an incomplete list of the relevant resources:
\begin{itemize}
\item{\textit{Individual websites of a software}}\\
This is in some sense the basic layer of the software information infrastructure. Websites exist for many though not all software
  packages (from our experiences in the swMATH project we estimate that nearly two thirds of mathematical software packages provide information on own websites). Typically, these websites contain a lot of detailed information about a software, documentation, manuals, tutorials, software code (if the software is free), the programming language used, contact information, usability and licences, hard- and software requirements, publication lists, etc.
\item{\textit{Repositories}}\\
Software repositories as ``The Comprehensive R Archive Network (CRAN)'' \cite{CRAN} or ``The Comprehensive Perl Archive Network (CPAN)'' \cite{CPAN} provide and maintain  metadata plus the source code of software collections. CRAN is a repository for statistical software written in the R language and presents standardized meta information, the version history,  and links to the source code for nearly 10,000 packages.
\item{\textit{Portals, directories and information services}}\\
Portals or directories of software provide lists of software, metadata, and links. ``Fachgruppe Computeralgebra'' \cite{FAG} or ``SIGSAM'' \cite{SIGSAM} offers structured webpages for computer
algebra systems. These services, like the Symbolic Data project \cite{SD}, are not limited to information about software but also on conferences and workshops, researchers, and data. We will discuss this in more detail below.

An informative list of computer algebra systems can also be found in  Wikipedia \cite{WikipediaCAS}.

% Abbildungen
\begin{figurehere}
  \centering
  \includegraphics[width=0.9\columnwidth]{rubrik_themenanwendungen/Sperber-Final/aca3.jpg}
  \caption{A snippet of the Wikipedia (I): list of  computer algebra systems.\label{abb_3}}
\end{figurehere}

\begin{figurehere}
  \centering
  \includegraphics[width=0.9\columnwidth]{rubrik_themenanwendungen/Sperber-Final/aca4.jpg}
  \caption{A snippet of Wikipedia (II): functionalities of computer algebra systems \label{abb_4}}
\end{figurehere}

The lists mentioned here are manually maintained and updated, have different structure also for metadata and are weakly coordinated, see also the remarks about the Computer Algebra Social Network (CASN) below.
\item{\textit{Further resources}}\\
\begin{itemize}
\item{Services}\\
Software can be available in different forms, e.g., as a service (cloud computing, Class Group Database \cite{CCDb,MKD}).
\item{Journals specialized in mathematical software}\\
  Also the software journals are mentioned here because they play a pioneering role for the quality control and evaluation of software. There exists a number of journals specialized in mathematical software, e.g., the Journal of Software for Algebra and Geometry \cite{JAG}, where the peer reviewing also includes mathematical software.
\item{Conferences (including proceedings) on CASs}
\end{itemize}
\end{itemize}
Moreover, mathematical software must be considered in its context which is given by mathematical theories, algorithms, programming languages, applications, data, e.g., benchmarks and data formats, and also the developers and user communities of software. Context analysis is also an important method to build up and develop powerful machine-based information services for mathematical software which will be demonstrated by the swMATH concept in the next section.

% Ueberschriften fuer Abschnitte im Aufsatz:
\Ueberschrift
% Ueberschrift
{swMATH}
% Label
{sec_4}
% Ueberschrift fuer Unterabschnitte im Aufsatz:
\Ueberschriftu
% Ueberschrift
{The publication-based-approach}
An important feature of the swMATH \cite{swMATH} concept, the publication-based approach, has its origin in the analysis of context information. Instead of analyzing  mathematical publications for software citations and information about software, the database zbMATH \cite{zbMATH} is used.  Especially, the zbMATH database contains the following data of publications which are of relevance for the analysis: title, keywords, reviews or abstracts, reference lists, and classification codes.
The data analysis and knowledge generation of the swMATH approach has several steps:
\begin{enumerate}
\item{\textit{Identification of software references within the zbMATH data}}\\
The title, the review or abstract, and the reference lists of publications are searched for indicators for software references by heuristic means. Such indicators can be artificial names in combination with characteristic words as software, module, package, etc.  Of course, the methods used are very simple but work surprisingly well. The acceptance of a software citation standard corresponding to the recommendations given above would make the heuristic methods obsolete and permit a secure and complete identification of software.
\item{\textit{Extraction of information about software}}\\
Reviews and abstracts of a mathematical publication contain most of all content information, especially a description of the problems investigated, the used methods, and results. \\ For software
references it is useful to distinguish between two classes of publications: The publications which describe a software (labeled as  ``standard publications'') and publications which use a software for solving a problem (labeled as ``user publications''). Both types provide different information about a software and are processed in different ways. Some information directly enters into swMATH, e.g., keywords or MSC codes.
\item{\textit{Aggregation and ranking of information}}\\
Currently, swMATH has nearly 16,000 entries on software packages and other mathematical research data which contain more than 215,000 software citations in more than 130,000 publications. In other words, there is often a great number of publications citing a software. This allows to weight the information by the corresponding number of the keyword frequencies which is done in the keyword cloud, to create an ``acceptance profile'' of the software (citation graph) by the number of annual publications citing a software.  It's also possible to give some information about  related software based on the MSC classification codes. The number of publications citing a software could also be used as a measure for credit to the developers. Further features are possible, e.g. the definition of an application profile of the software. All this can be done automatically by heuristic means.
\end{enumerate}
% Ueberschrift fuer Unterabschnitte im Aufsatz:
\Ueberschriftu
% Ueberschrift
{The Web-based approach}
The publication-based  method is a powerful tool but has limitations. Publications do not cover technical details about the implementation, the programming language, or the required hard- and software environment of a software. Typically, this information is given in manuals and documentations. Also other context information, e.g., test data and benchmarks or programming languages, are important for reproducing the results of a publication. As said above, this kind of information can often  be found on other resources on the Web.

Therefore swMATH tries to enrich the information about software by adding information from the Web. At first, swMATH tries to find the website of a software and links it if the search was successful. We have started to develop methods for analyzing the websites, see \cite{TPDL}. For this purpose the Internet Archive \cite{IA} is used which provides also the data from a lot of websites of mathematical software from the past.
Also the information of some repositories is integrated in swMATH.
swMATH shows that the analysis of different resources is a promising way to run and maintain useful and efficient information services for mathematical software.\par

% Ueberschrift fuer Unterabschnitte im Aufsatz:
\Ueberschriftu
% Ueberschrift
{swMATH in a nutshell}
Currently swMATH provides the following information:
\begin{enumerate}
\item a list of mathematical software packages (and other related mathematical research data), as complete as possible
\item a persistent identifier (a five digit number)  for each software,
\item metadata, especially about its content (description, keywords, MSC codes),
\item links  to the websites of the software (if existing).
\item a list of publications citing a software
\item a list of similar software
\item an acceptance profile for the software
\item links from the publications to the corresponding versions (currently only in the test version)
\item links to the Internet archive
\item links to manuals, documentation, source code, etc.
\end{enumerate}

Moreover, swMATH provides a simple and an extended search functionality for software. In the case of computer algebra systems, swMATH lists more than 100 entries, cutouts of this list and the swMATH page for the software ``Singular'' are shown below.\\

% Abbildungen
\begin{figurehere}
  \centering
  \includegraphics[width=0.9\columnwidth]{rubrik_themenanwendungen/Sperber-Final/aca5.jpg}
  \caption{A snippet of the swMATH list for computer algebra systems.\label{abb_5}}
\end{figurehere}

% Abbildungen
\begin{figurehere}
  \centering
  \includegraphics[width=0.9\columnwidth]{rubrik_themenanwendungen/Sperber-Final/aca6.jpg}
  \caption{A snippet of the swMATH webpage for the software ``Singular''.\label{abb_6}}
\end{figurehere}

\hspace{3cm}

% Ueberschriften fuer Abschnitte im Aufsatz:
\Ueberschrift
% Ueberschrift
{The Computer Algebra Social Network}
% Label
{sec_5}
Software information is an important part of the scientific digital information and communication infrastructure. The scientific digital infrastructure is  widely distributed and must be able to process and link heterogeneous resources, e.g., information about researchers,  publications, software, conferences and workshop, etc. and data formats  in a semantic way.
This requires an active goal-oriented conceptual and technical cooperation between different players. All relevant data must be digitized, semantically enriched and encoded in a machine-understandable way.
The idea of the Computer Algebra Social Network (CASN), for an overview on CASN see H.-G. Gr\"abe \cite{CASN},   is an advancement and continuation of the Symbolic Data concept. Established Web technologies, especially RDF, should be used for the semantic annotation of resources. Each resource, e.g., the servers of the German CA Fachgruppe or SIGSAM, can become a node in this network. The RDF files of the metadata which are created by the provider of a node guarantee that the information can be automatically linked and is accessible via CASN. The swMATH service is integrated in CASN.

% Ueberschriften fuer Abschnitte im Aufsatz:
\Ueberschrift
% Ueberschrift
{Summary and outlook -- what we can and should do}
% Label
{sec_6}

A powerful and sustainable infrastructure for mathematical software information is in the interest of  developers and  users.  It is also important for the positioning and the role of this research subject in the sciences and in society.  The infrastructure must be oriented towards the interests  of the developers and users. The mathematical community  should actively take part and influence the development.
Specifically, the information about software is inconsistent, is distributed, not standardized, and not machine-processable, which hampers the combination of heterogeneous resources of mathematical software and related information. A better citation practice, standardization, enrichment of the semantic information, coordination, and a better integration of software as desired in the Open DreamKit project \cite{OpenDreamKit}  opens new perspectives for this research subject.
We need a broad dialogue and a communication forum which brings the developers, the user communities, information experts, and service providers together for a  discussion of all aspects. The symbolic computation community could play a pioneering role to establish a sophisticated infrastructure for a mathematical subject which covers all relevant resources. These include
\begin{itemize}
\item{definition of the overall goals and principles of an infrastructure for mathematical software}\\
\item{standardization}\\
for the authors: The authors should cite the software corresponding to the recommendations of the SCPs, especially marking up the type and give information about the versions, releases, etc,\\
for the developers and service providers: There should be developed a standardized metadata scheme for mathematical software (analyzing the different facets of software information),\\
for service providers: The information should be provided in a machine-understandable way which supports a semantically sensible combination of information,\\
for service providers: Development of intuitive tools to support the standardized description of citations
\item{a better linking of the information resources of software}
for service providers: This requires a cooperation between the providers of the different services for software and its context and the development of user interfaces.
\end{itemize}

\section*{Acknowledgement}
We are grateful to the SIGSAM to help us reach a
wider audience by additionally publishing this article in the
``Communications in Computer Algebra''.

% Literaturverzeichnis
\bibliographystyle{plain}
\begin{thebibliography}{19}
\itemsep=0cm plus 0pt minus 0pt

% Makro fuer einen Eintrag im Literaturverzeichnis

\bibitem
%Label
{CCDb}
%Autoren
Boy, Maximilian:
%Titel
\newblock {A Database for Class Groups of Number Fields, homepage}.\\
%Zeitschrift
\newblock \url{http://www.mathematik.uni-kl.de/~numberfieldtables/}

\bibitem
%Label
{FAG}
%Autoren
Fachgruppe Computeralgebra:
%Titel
\newblock {Homepage}.\\
%Zeitschrift
\newblock \url{http://www.fachgruppe-computeralgebra.de/systeme/}

\bibitem
%Label
{swMATH}
%Autoren
FIZ Karlsruhe, ZIB Berlin:
%Titel
\newblock {swMATH Homepage}.
%Zeitschrift
\newblock \url{http://www.swMATH.org}

\bibitem
%Label
{zbMATH}
%Autoren
FIZ Karlsruhe:
%Titel
\newblock {zbMATH Homepage}.\\
%Zeitschrift
\newblock \url{http://www.zbMATH.org}

\bibitem
%Label
{Conferences}
%Autoren
Gr\"abe, Hans-Gert:
%Titel
\newblock {RDF File of CA Conferences}.\\
%Zeitschrift
\newblock \url{http://symbolicdata.org/rdf/Conferences/}


\bibitem
%Label
{SD}
%Autoren
Gr\"abe, Hans-Gert:
%Titel
\newblock {The SymbolicData Project}.
%Zeitschrift
\newblock \url{http://symbolicdata.org}

\bibitem
%Label
{CASN}
%Autoren
Gr\"abe, Hans-Gert:
%Titel
\newblock {The SymbolicData Project -- Maturing the Computer Algebra Social Network Perspective}.
%Zeitschrift
\newblock {{\em Computeralgebra Rundbrief 59, p. 17-21},\\
\url{http://symbolicdata.org/Papers/aca-16-paper.pdf}}

\bibitem
%Label
{TPDL}
%Autoren
Holzmann, Helge; Sperber, Wolfram; Runnwerth, Mila:
%Titel
\newblock {Archiving Software Surrogates on the Web for Future Reference}.
%Zeitschrift
\newblock {\em In: Norbert Fuhr, L\'aszlo Kov\'acs, Thomas Risse, Wolfgang Neidl, Research and Advanced Technology for Digital Libraries: 20th International Conferences on Theory and Practice of Digital Libraries, TPDL 2016, Hannover 2016, LNCS 9819, p. 215 - 226,}

\bibitem
% Label
{Howison&Bullard2015}
% Autoren
Howison, J.; Bullard, J.:
% Titel
\newblock{Software in the scientific literature: Problems with seeing, finding, and using software mentionend in the biology literature}.
% Zeitschrift
\newblock {\em Journal of the Association for Information Science and Technology, 2015,}\\
\url{http://dx.doi.org/10.1002/asi.23538}

\bibitem
%Label
{IA}
%Autoren
Internet Archive:
%Internet Archive
%Titel
\newblock {Homepage}.\\
%Zeitschrift
\newblock \url{https://archive.org/web/}


\bibitem
% Label
{Jackson}
% Autoren
Jackson, Mike:
% Titel
\newblock {How to cite and describe software?}\\
% Zeitschrift
\newblock \url{https://www.software.ac.uk/how-cite-and-describe-software}

\bibitem
%Label
{JAG}
%Autoren
Journal of Software for Algebra and Geometry Editorial Board:
%Titel
\newblock {Homepage}.\\
%Zeitschrift
\newblock \url{http://msp.org/jsag/about/cover/cover.html}

\bibitem
%Label
{MKD}
%Autoren
Kl\"uners, J; Malle, G:
%Titel
\newblock {A Database for Number Fields, homepage}.\\
%Zeitschrift
\newblock \url{http://galoisdb.math.uni-paderborn.de/}

\bibitem
%Label
{CPAN}
%Autoren
Perl Foundation:
%Titel
\newblock {The Comprehensive Perl Archive Network}.
%Zeitschrift
\newblock \url{http://www.cpan.org/}


\bibitem
%Label
{CRAN}
%Autoren
R Foundation:
%Titel
\newblock {The Comprehensive R Archive Network}.
%Zeitschrift
\newblock \url{https://cran.r-project.org/}

\bibitem
% Label
{SoftwareCitationPrinciples}
%%Autoren
Smith, Arfon M.; Katz, Daniel S.; Niemeyer, Kyle E.; FORCE11 Software Citation Working Group:
%% Titel
\newblock {Software citation principles}.\\
%% Zeitschrift
\newblock{{\em Peer J. Computer Science 2:e86, 2016},\\
 \url{https://doi.org/10.7717/peerj-cs.86}}

\bibitem
%Label
{SIGSAM}
%Autoren
Special Interest Group on Symbolic and Algebraic Manipulation of the ACM:
%Titel
\newblock {Homepage}.
%Zeitschrift
\newblock \url{http://www.sigsam.org/Resources/Software.html}


\bibitem
% Label
{SoftwareCitationWG}
%Autoren
Software Citation Working Group:
% Titel
\newblock {Homepage},
% Zeitschrift
\newblock \url{https://www.force11.org/group/software-citation-working-group}.

\bibitem
%Label
{OpenDreamKit}
%Autoren
Thi\'ery, Nicolas M.:
%Titel
\newblock {OpenDreamKit: Open Digital Research Environment Toolkit for the Advancement of Mathematics}.
%Zeitschrift
\newblock {{\em Computeralgebra Rundbrief 57, p. 17-18,}\\
\url{http://www.fachgruppe-computeralgebra.de/data/CA-Rundbrief/car57.pdf}}


\bibitem
%Label
{WikipediaCAS}
%Autoren
Wikipedia Community:
%Titel
\newblock {The Wikipedia Web page of CASs}.\\
%Zeitschrift
\newblock \url{https://en.wikipedia.org/wiki/Listofcomputeralgebrasystems}



\end{thebibliography}

%%%%%%%%%%%%%%%%%%%%%%%%%%%%%%%%%%%%%%%%%%%%%%%%%%%%%%%%%%%%%%%%%%%%%%%%%%%%%%%%

\end{multicols}
%\end{document}
