\documentclass{article}
\usepackage{a4wide,german,url}
\usepackage[utf8]{inputenc}

\newcommand{\SD}{{\sc Symbolic\-Data}}
\parindent0pt
\parskip4pt

\begin{document}

\section*{\centering Beitrag zum Teil „Berichte von Konferenzen“}

\subsection*{Workshop on SymbolicData Design}

Leipzig, 27. -- 28. August 2013

\url{http://symbolicdata.org/wiki/Events.2013-08}

The workshop was designed as final milestone of the E-Science Benchmarking
Project promoted for 12 month within the \emph{E-Science Saxony Framework}.
Unfortunately, the event was completely ignored by the Computer Algebra
Communities, so that we had no opportunity to present the results of the
project to a larger audience.  Instead we had intense discussions with people
from the \emph{swmath} project (\url{http://www.swmath.org}, a project of the
\emph{Zentralblatt Mathematik} towards an information service for mathematical
software) about trends in Semantic Web Technologies that are suitable to
support future common efforts towards a semantic aware IT infrastructure for
Computer Algebra.

In a first talk \emph{Hans-Gert Gräbe} presented the state of the SymbolicData
project.  Note that at the end of September 2013 version 3 of SymbolicData was
released, thus finishing a major redesign of SymbolicData, that marks a
milestone across the implementation of semantic techniques within Computer
Algebra.  We strongly use RDF and Linked Data principles in the organisation
of the data. These principles are also reflected in the presentation of the
data at \texttt{symbolicdata.org}. All resources are delivered via
\texttt{http/rdf+xml} and a Sparql endpoint allows for navigation in the
metadata. This can be installed also on a localhost and thus can be integrated
into a local benchmarking or profiling infrastructure (best using python as
scripting language and a web server at localhost). A more detailed description
of the new release is available from the SymbolicData web pages and will be
given also in the next issue of the \emph{Computeralgebra Rundbrief}.

\emph{Andreas Nareike} presented in a second talk his prototypical integration
of the Polynomial Systems subproject with sagemath and SymbolicData as a sage
package \emph{sdsage} that smoothly integrates both the global SD network
infrastructure and a local installation into the sagemath process. One can
load data and metadata transparently into sage objects and process them as
mathematical objects in the usual way within sage. 

\emph{Ulf Schöneberg} gave a talk about effort at the ZBMath to discover and
understand mathematical formulas in Zentralblatt mathematical reviews, mixing
classical colocation approaches with semantic enriched opportunities of latex
mark-up. This research is part of larger efforts within, e.g., the OpenMath
activities.

We discussed in great detail the potential interplay between
\begin{itemize}
\item the efforts at ZBMath to organize access to data in well established RDF
  based formats,
\item the SymbolicData intercommunity efforts and experience with Linked Data
  standards, Sparql endpoints, Virtuoso and Ontowiki based local
  installations,
\item ongoing efforts of the DNB and other libraries (SLUB Dresden, UB
  Leipzig) to reshape their catalogue data towards Linked Data standards and
  get them interoperating within the GND project,
\item perspectives to join forces with these library projects to strengthen
  the IT infrastructure for Computer Algebra Communities.
\end{itemize}
\begin{flushright}
    Hans-Gert Gr"abe (Leipzig)\footnote{to be published in
      „Computeralgebra-Rundbrief“ 53, Oktober 2013}
  \end{flushright}
\end{document}
