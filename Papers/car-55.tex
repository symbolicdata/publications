\documentclass[a4paper,11pt]{article}
\usepackage{CAR}
\hyphenation{name-space}

\def\Ueberschrift#1#2{\section{#1}}
\def\SD{\textsc{SymbolicData}}
\def\eg{e.g.\ }
\def\ie{i.e.\ }

\newenvironment{code}{\par\small\tt\footnotesize \begin{tabbing}
\hskip12pt\=\hskip12pt\=\hskip12pt\=\hskip12pt\=\hskip5cm\=\hskip5cm\=\kill}
{\end{tabbing}\normalsize}

\newcommand{\xsdprefix}{\char94\char94} % ^^
\def\dq{\symbol{34}}

\begin{document}

\begin{center}
  {\Large\bf The {\SD} Project -- from Data Store\\ to Computer Algebra Social
    Network} \vskip1em

H.-G.\ Gr\"abe, S.\ Johanning, A.\ Nareike (Universit\"at Leipzig)\\[6pt]
\texttt{(graebe$\mid$nareike)@informatik.uni-leipzig.de,
  simonjohanning@googlemail.com} 
\end{center}
Appeared in \emph{Computeralgebra Rundbrief} 55, October 2014, pp. 22--26.  

\begin{multicols}{2}
\noindent

%%%%%%%%%%%%%%%%%%%%%%%%%%%%%%%%%%%%%%%%%%%%%%%%%%%%%%%%%%%%%%%%%%%%%%%%%%%%%%%%

% Ueberschriften fuer Abschnitte im Aufsatz:
\Ueberschrift{Introduction}{intro}

A powerful digital research infrastructure becomes increasingly
important in today's networked and interlinked world.  This includes
digital support for dissemination of new papers, the refereeing
process, conference submissions, and scientific communication within
communities.  Services such as MathSciNet, arXiv.org, EasyChair.org,
or bibsonomy.org have been established and their usefulness is
acknowledged by the larger scientific community. For mathematics in
the large the vision of a \emph{21st Century Global Library for
  Mathematics Research} (GDML) \cite{GDML} matured during the last years. 

Smaller academic communities, \eg the Computer Algebra (CA) community,
are challenged to organize their \emph{intracommunity} communication
infrastructure in a similar way.  Infrastructural efforts are rarely
acknowledged by the reputational processes of science, however, and
are hence left to the casual engagement of volunteers unless led by
leading-edge scientists of the community.

Open Source culture offers plenty of experience of how to substitute 
centrally organized projects by decentralized networked structures and 
indeed the new focus of {\SD} version 3 is that of an intercommunity 
project, providing not only reliable access to data for testing and 
benchmarking purposes but also technical support for interlinking 
between different CA subcommunities.

In this paper we explain the technological basics of an RDF based
semantic web of mathematics and discuss social interlinking to the
\emph{vision in large} of an emerging GDML.

\Ueberschrift{RDF Basics}{RDFBasics}

We refer to \cite{cicm-14} for an overview of the goal, aims, concepts,
infrastructure and history of the {\SD} project, in particular for a more
detailed motivation to use RDF concepts and Linked Data principles.  The use
of such principles and notations for the metadata of our data store is one of
the core advances with version~3 of {\SD}. In this section we give a brief
overview of such concepts.
 
RDF is a data model which stores information as 
\emph{triples}
\begin{center}
s \; p \; o \; .
\end{center}
which are considered a \emph{sentence} with subject $s$, predicate $p$ and
object $o$.  Subjects and predicates are URIs (Uniform Resource Identifiers)
while objects (or `values') can be a URI or a literal (a string in quotes).
There are shortcut notations of RDF, \eg RDF-XML, JSON, or Turtle, and plenty
of tools and parsers for the different formats.

As an example we consider the resource at
\begin{code}
  http://symbolicdata.org/XMLResources/\\\> IntPS/Czapor-86c.xml
\end{code}
that represents the XML record about the polynomial system
\begin{align*}
 & x^2 + y\,z\,a + x\,d + g\\
 & y^2 + x\,z\,b + y\,e + h\\
 & z^2 + x\,y\,c + z\,f + k
\end{align*}
known as the \emph{Czapor-86c} example.  Such a polynomial system can be
interpreted differently as polynomial ideal in different polynomial rings (see
below).

We are going to store properties of the interpretation of that polynomial
system as ideal
$$
  I\subset \mathbb{Q}[x,y,z,a,b,c,d,e,f,g,h,k]
$$
(an \emph{ideal configuration}, associated with that record) as new RDF
subject.  First, a new URI has to be assigned to such a new subject
(using a sound naming scheme, not discussed here):
\begin{code}
<http://symbolicdata.org/Data/Ideal/\\\> Czapor-86c.Flat>
\end{code}
or \texttt{sdideal:Czapor-86c.Flat} for short.

Now we can assign metainformation to that subject.  Some of these 
properties can be extracted directly from the XML resource, others have 
to be calculated.  Here is the record in Turtle notation:
\begin{code}
  sdideal:Czapor-86c.Flat a sd:Ideal ;\\\+ 
  rdfs:comment {\dq}Flat variant of Czapor-86c{\dq} ;\\
  sd:hasDegreeList {\dq}3,3,3{\dq} ;\\
  sd:hasLengthsList {\dq}4,4,4{\dq} ;\\
  sd:relatedPolynomialSystem\\\>  sdpol:Czapor-86c ;\\ 
  sd:hasVariables\\\>  {\dq}x,y,z,a,b,c,d,e,f,g,h,k{\dq} .
\end{code}
This record contains 8 triples in Turtle shortcut notation.  Since all triples
share the same subject (the first line), Turtle compacts the notation by
separating property--value pairs to the same subject with a semicolon.  The
\texttt{sd:}, \texttt{sdp:}, \texttt{sdideal:}, \texttt{sdpol:} and
\texttt{sdxml:} prefixes are abbreviations for name-space
prefixes\footnote{Note that for more clarity we use a sloppy notation here that
  is not fully compliant with the RDF notational standards. }.

The configuration above refers to the Polynomial System
\texttt{sdpol:Czapor-86c}, described by the RDF record
\begin{code}
  sdpol:Czapor-86c\+\\  
  a sd:IntegerPolynomialSystem ;\\
  sd:createdAt {\dq}1999-03-26{\dq} ;\\
  sd:createdBy sdp:Graebe\_HG ;\\
  sd:relatedXMLResource\\\>  sdxml:Czapor-86c.xml . 
\end{code}
In other words, the \emph{Czapor-86c.Flat} configuration is derived 
from the `true' \emph{Czapor-86c} example, the ideal
\begin{gather*}
  I'\subset S'=\mathbb{Q}(a,b,c,d,e,f,g,h,k)[x,y,z]
\end{gather*}
generated by the same polynomials in a different polynomial ring $S'$.
Geometrically the latter represents a complete intersection of three
(generic affine) quadrics over the field of rational functions in the
given parameters.  There is also a record on that configuration:
\begin{code}
  sdideal:Czapor-86c a sd:Ideal ;\+\\
  sd:createdAt {\dq}1999-03-26{\dq} ;\\
  sd:createdBy sdp:Graebe\_HG ;\\
  sd:hasDegreeList {\dq}2,2,2{\dq} ;\\
  sd:hasLengthsList {\dq}4,4,4{\dq} ;\\
  sd:hasDegree {\dq}8{\dq}{\xsdprefix}xsd:integer ;\\
  sd:hasDimension {\dq}0{\dq}{\xsdprefix}xsd:integer ;\\
  sd:hasParameters {\dq}a,b,c,d,e,f,g,h,k{\dq} ;\\
  sd:parameterize sdideal:Czapor-86c.Flat ;\\ 
  sd:hasVariables {\dq}x,y,z{\dq} .
\end{code}
\emph{Czapor-86c} is derived from the \emph{Czapor-86c.Flat}
configuration by parameterization with respect to the given parameters
(see below for details).  Note that the degree lists of
\emph{Czapor-86c.Flat} and \emph{Czapor-86c} are different.  The
record contains some more metainformation: $S'/I'$ is zero dimensional
and has degree 8.

To operate on RDF data the databases (called \emph{RDF Graphs}\/) have
to be uploaded into an \emph{RDF triple store}.  {\SD} uses the
Virtuoso triple store \cite{Virtuoso} that provides also a SPARQL
endpoint \cite{sdsparql} to query the data, see \cite{cicm-14} or our
wiki \cite{sdwiki} for more background information.

\Ueberschrift{Modelling Polynomial Systems}{MPS}

RDF provides language tools to express metainformation in an interoperably
searchable way that have to be applied in a \emph{domain-specific way} to model
topics from CA subcommunities.  We explain such aspects of {\SD} modelling
again on the topic of Polynomial Systems.  Similar considerations are required
to model any other part of the {\SD} database and the {\SD} wiki \cite{sdwiki}
gives details about modelling other data (Free Algebras, G-Algebras, Geometry
Proof Schemes etc.).

Polynomial Systems XML resources store lists of polynomials in 
distributive normal form with integer coefficients together with a 
complete list of variables (and, for modular systems, the modular base 
domain $GF(p)$).  Hence even modular polynomial systems can be 
semantically considered as set $F = \{f_1,\dots,f_s\}$ of polynomials in 
$S = \mathbb{Z}[x_1,\dots,x_n]$ in the indeterminates $x_1,\dots,x_n$ 
listed in the record.

Polynomial Systems Solving considers polynomial systems in different 
contexts.  We have already shown how to construct the \emph{Czapor-86c} 
ideal from the \emph{Czapor-86c.Flat} resource.  This is an example 
for a standard kind of interpretation where we divide indeterminates 
$x_1,\dots,x_n$ into disjoint subsets $u_1,\dots,u_k$ and $z_1,\dots,z_m$ 
and consider the ideal
$$
I'\subseteq S'=R(u_1,\dots,u_k)[z_1,\dots,z_m]
$$
generated by the images of $f_1,\dots,f_s$ in $S'$ over the base 
coefficient field $R$.  Here $u_1,\dots,u_k$ are considered as 
parameters and $z_1,\dots,z_m$ as variables.  $R$ is usually the field 
$\mathbb{Q}$ of rationals or a modular field $GF(p)$ (other settings are 
possible).  $S$ has the universal property that the canonical map on the 
indeterminates extends to a ring homomorphism $S\to S'$ in a unique way.  
We call such an interpretation of a Polynomial System resource as ideal 
generators in a polynomial rings $S'$ \emph{(ideal) configuration}.  
(Such configurations can be derived not only from Polynomial System 
resources but from other configurations as well.)

While {\SD} version 2 would store each new configuration as a new 
resource, version 3 foresees (polynomial-time) transformations to express 
their relation to the basic resource.

For example, the \texttt{sd:homogenize} transformation derives a new
configuration by homogenization (with respect to standard grading).  Given the
configuration $F$ in $S'$ and a new variable $h$ we generate the homogenized
polynomials $F^h = \{f_1^h,\dots,f_s^h\}$ in
$$
S'' = R(u_1,\dots,u_k)[z_1,\dots,z_m,h]
$$
and the ideal $I''$ generated by $F^h$ in $S''$.  There is a natural 
ring homomorphism $\phi:S''\to S'$ mapping $h\to 1$ and the polynomials 
$f_1^h,\dots,f_s^h$ are called the \emph{pull-back polynomials} of 
$f_1,\dots,f_s$ with respect to $\phi$.  Note that the pull-back ideal 
$\phi^{-1}(I')$ contains the pull-back polynomials but is not 
necessarily generated by them.

 The transformations \texttt{sd:flatten}, \texttt{sd:parame\-terize} and
 \texttt{sd:substitute} are defined by similar concepts.  Compared to former {\SD}
 versions the only restriction is that semantic-aware tools (that `know' what
 a polynomial is) are required to generate configurations from given basic
 ones.  We believe this is not a real restriction since for serious
 computations on Polynomial Systems semantic-aware tools are required in any
 case.  Such tools also have to provide a Polynomial Systems parser to input
 the basic XML examples.  With your favorite CA software being aware of
 polynomial semantics it should be easy to implement the transformation modes
 required to obtain the different configurations. As a proof-of-concept,
 A.~Nareike compiled the \emph{sdsage package} \cite{sdsage} to be integrated
 with the Sagemath system \cite{Sagemath}.

\Ueberschrift{Navigation within the\\ Polynomial Systems Data}{navigation}

Special semantic knowledge is also required for navigation and 
identification of data.  This topic is particularly important for 
intercommunity communication since one cannot expect researchers to be 
familiar with common practices of the different subcommunity.
For a case-in-point let us again turn to Polynomial Systems Solving.

It is one of the challenges to check whether a Polynomial System 
configuration obtained from an external source is contained in the 
database, since the `same' configuration may be given by polynomials 
with different variable sets and in different term orders.  Thus for 
navigational purposes \emph{fingerprints} of Polynomial System 
configurations are required that are independent of variable names and 
term orders.  For a polynomial $0\neq f\in 
S'=R(u_1,\dots,u_k)[z_1,\dots,z_m]$, invariants may be derived from the 
set $T(f)$ of terms.  Every such polynomial has a distributive normal 
representation
$$
f = \sum_{\alpha\in\mathbb{N}^m}{c_\alpha\cdot z^\alpha},
\quad\begin{aligned}
& c_\alpha\in R(u_1,\dots,u_k), \\
& z^\alpha = z_1^{\alpha_1}\cdot\ldots\cdot z_m^{\alpha_m}\,,
\end{aligned}
$$
and $T(f)=\{z^\alpha\,:\,c_\alpha\neq 0\}$ is independent of the term 
order (but not of the variable names).  There are two invariants that 
are well-defined for $f$ regardless of variable names and orders -- 
the number $|T(f)|$ of terms (the \emph{length} of the polynomial $f$) 
and the pattern of the total degrees $(\deg(z^\alpha)\,:\,c_\alpha\neq 
0)$ of the terms in $T(f)$.  In particular, for $0\neq f$ the maximum 
degree $\deg(f)=\max(\deg(z^\alpha)\,:\,c_\alpha\neq 0)$ is 
well-defined.

We use ordered lists of polynomial lengths and of maximum degrees as
fingerprints of configurations and provide them precompiled as part of
the metadata for a given configuration.  Such a fingerprint can easily
be computed by almost all semantic-aware tools.  While configurations
with different fingerprints are definitely distinct, there can be
different examples with the same fingerprint.  Although such
fingerprints could be refined there was no need so far, since the
examples with equal fingerprints are rare and can easily be inspected
by hand.

\Ueberschrift{Knowledge Frames and Social Framing}{Social Framing}

The {\SD} project can be seen as part of the worldwide efforts towards
a \emph{World Digital Mathematical Library} (WDML). The most viable
contributions on that way are the EuDML project \cite{EuDML}, funded
by the European Union during 2010--2013, and the WDML project
\cite{WDML}, triggered by the US National Research Council and heavily
supported by the IMU -- the International Mathematical Union -- and
the Alfred P. Sloan Foundation.  There are other activities on the way
in that direction, in particular by Wolfram Research Inc., developing
\emph{Wolfram Alpha} as one of the most mature online resources of
structured mathematical knowledge.

One of the big challenges within these projects is the question ``How
to structure and represent mathematical knowledge?'' Such a question
is not only (and probably not even in the first plan) a question about
mathematics but also about mathematicians, their common efforts,
social relations, and the way how they organise common work.  Minsky's
well established concept of \emph{knowledge frames} as ``artificial
intelligence data structure used to divide knowledge into
substructures by representing {`stereotyped situations'}''
\cite{frame} suggests that such ``stereotyped situations'', in our
case different research contexts with complex social structures and
interrelations, play an important role in the way how knowledge
structures emerge and evolve.  A. and M. Kohlhase \cite{kohlhase}
emphasized the importance of such practices for mathematicians and
formalized them in the (slightly different) notions of \emph{theory
  graphs} and \emph{framing} (the latter considered as `reframing' in
a Minsky inspired terminology \cite{reframing}).

Computer Algebra at the border of mathematics and computer science
played an important and special role during any computer triggered
technological breakthrough in mathematics in the last decades. It is
a first class challenge to the CA community to organize the changes
towards a GDML for their own community and to contribute to the
changes at large.

The setting of (not only) the {\SD} Project to address the needs of
the CA community and their severe subcommunities -- considered as a
``major social frame'' with many intimately related ``smaller social
frames'' -- is a first class playground
\begin{itemize}
\item to study and explore own cooperate contexts and practices under
  the special aspects and formalization requirements of modern
  semantic web technologies,
\item to develop and communicate best practices along the IMU
  recommendations \cite{PitmannLynch},
\item and to advance towards a better understanding of their
  requirements and consequences,
\end{itemize}
since we have researchers well educated both in mathematics and computer
science and the scientific CA community is small enough to allow for social
relations in a less institutionalized way.

\Ueberschrift{Towards CASN --- a Computer Algebra Social Network}{CASN}

Such a challenge meets other visions of the {\SD} Project, in particluar to
collect not only benchmark and testing data but also valuable background
information about the records in the database, \eg information about papers,
people, history, systems, concerned with the examples in our collection.  RDF
is particularly suited for this, for it provides both the concept of typeless
URIs within a typed world to point to resources of different types in a
uniform way and it allows linking to foreign URIs in other databases to build
up a semantic network with many nodes where the node at
\texttt{symbolicdata.org} is only one in the multitude of nodes of such a
(distributed) Computer Algebra Social Network.

There are plenty of activities towards such an `e-science world', in particular
by the \emph{Association of German Libraries} \cite{GND}, by the
\emph{Zentralblatt Mathematik} \cite{postagging}, by the MKM community
\cite{MKM} and others.

As explained in the introduction it is a great challenge to small scientific
communities to adopt such developments for its own scientific communication
processes and to join forces with other scientific communities to get own
requirements publicly recognised. A first step in such a direction is a more
detailed description of ongoing scientific processes using standard RDF
terminology. 

{\SD} started such efforts within the CA scientific community, collecting and
presenting
\begin{itemize}
\item information about scientific activities of people -- as of Sept. 2014 the
  {\SD} People database contains \texttt{foaf:Person} entries of 708 people
  from the CA scientific community that can be explored via the {\SD} SPARQL
  endpoint \cite{sdsparql} -- it is mirrored at \cite{casnsparql},
\item information about upcoming conferences -- the information is extracted
  via SPARQL query to \cite{casnsparql} from
  \url{http://symbolicdata.org/casn/UpcomingConferences/} and displayed in the
  Wordpress based site of the German Fachgruppe \cite{cafg},
\item information about past conferences and conference reports with references
  to the {\SD} People database about speakers and organizers, 
\item information about German CA working groups and the SPP 1489 projects with
  references to the {\SD} People database,
\item keyword enriched information about scientific publications in the
  CA-Rundbrief of the German Fachgruppe using the \texttt{dcterms} ontology --
  the information is displayed in the Wordpress based site of the German
  Fachgruppe \cite{cafg}, 
\item and semantic annotations to news in the blog of the German Fachgruppe as
  instances of RDF type \texttt{sioc:BlogPost} and \texttt{bibo:Document}
  attached to the blog post URL. 
\end{itemize}
A particularly interesting project was started in cooperation with
\emph{Zentralblatt Mathematik} (ZBMath) towards a better author
disambiguation.  During the last years ZBMath spent much effort for better
author disambiguation primarily using language processing technology to
retrieve data tracks of people from their stock of abstracts, see
\cite{postagging}.  The {\SD} references within our People database offer
additional insight into people activities (and -- if properly recognised by
the community -- allows for active support and influence on a process that
touches both vivid personal and scientific interests of the community), and we
started to align {\SD} People URIs with ZBMath URIs in the \emph{ZBMath Person
  Matches} table at \url{http://symbolicdata.org/Data/ZBMathPeople/}.

Despite the importance of the ongoing 'Big Data' harvesting processes for the
moment this and other activities suffer from little attention of a broader CA
audience.  In our talk at the CICM-14 conference \cite{cicm-14} we asked: ``How
turn passive users (listed in the {\SD} People database) into active ones?'' A
first hurdle is authentication. Nowadays there are three different ways to
solve that problem:
\begin{enumerate}
\item The `classical one': Set up a local database with user/password; users
  have to administer plenty of such application/user/password records (of
  course best using RDF technology).
\item The `Google one': Use the (OAuth based) authentication service of one of
  the 'big players'.  Your advantage: one password fits all, but who knows
  what the guys are tracking {\ldots}
\item The WebID approach: Use your browser certificate and a FOAF profile
  managed by your own to provide access.  

  You register with {\SD} your FOAF profile, we will take the challenge stored
  there and offer it mixed up with our challenge to your browser. If the
  browser (with your certificate, imported into the browser from your source)
  returns the correct answer we know that's you sitting in front of it.
\end{enumerate}
We started to set up such an infrastructure along the FOAF standards
as 'proof of concept':
\begin{itemize}
\item The {\SD} People database contains 'lightweight' FOAF profiles.
\item For any person interested to join actively the CASN we generate
  a \texttt{foaf:Personal\-Profile\-Document} that relates this
  'lightweight' FOAF profile with your own FOAF profile as
  \texttt{foaf:primaryTopic}.
\item For a limited number of people from the German Fachgruppe (yet
  as passive users) we created such profiles and use them to display
  information about the different boards of the German Fachgruppe at
  their website \cite{cafg}.
\item You can take over your own profile at any convenient web place
  under your control and extend it to meet the authentication
  requirements and tell us about that place.

  We change your PersonalProfileDocument and you are ready actively to
  join the CASN process.
\end{itemize}
The vision of a Computer Algebra Social Network goes far beyond that: 
\begin{itemize}
\item Maintain at your local site up-to-date information about your working
  group and its people in a consistent RDF format as \eg the AKSW team
  \cite{AKSW} does at \url{http://aksw.org/Team.html}.
\item Maintain your FOAF profile at a personal page containing all important
  public up-to-date information about your activities in a consistent RDF
  format as \eg the AKSW team member Natanael Arndt does at
  \url{http://aksw.org/NatanaelArndt.html}.
\item Organize a regular harvesting process within the scientific community for
  such information to feed common pages at sites as \eg
  \url{http://www.computeralgebra.de}, and to substitute part of the
  information centrally stored at {\SD} today by decentrally managed one.
\item Set up and run within the scientific community a semantic-aware
  Facebook-like Social Network and contribute to it about all topics around
  Computer Algebra using tools that express your contributions in an RDF-based
  syntax.
\end{itemize}
The last point sounds quite visionary but it is in no way utopic.  The
AKSW team provides a first prototype of a tool that realizes the
challenging concept of a \emph{Distributed Semantic Social Network}
\cite{dssn} to be running, in contrast to Facebook, on a multitude of
independent nodes all over the world.  Since the concept works also
with a single node and can be extended later on, we set up such a node
at \texttt{symbolicdata.org} for testing, see our CASN wiki page
\cite{sdwiki} for more information.  Even if all this is very
pre-alpha yet, the future is already on the way.  Don't miss the
train.

%%%%%%%%%%%%%%%%%%%%%%%%%%%%%%%%%%%%%%%%%%%%%%%%%%%%%%%%%%%%%%%%%%%%%%%%%%%%%%%%

% Literaturverzeichnis
\begin{thebibliography}{1}
\itemsep=0cm plus 0pt minus 0pt

\bibitem{AKSW} The Agile Knowledge Engineering and Semantic Web Group
  at Leipzig University.  \url{http://aksw.org/About.html}
  [2014-02-19]

\bibitem{GDML} Developing a 21st Century Global Library for
  Mathematics Research.  Report of the Committee on Planning a Global
  Library of the Mathematical Sciences.  The National Academies Press
  2014.  \url{http://www.nap.edu/catalog.php?record_id=18619}
  [2014-09-24]

\bibitem{EuDML} EuDML -- the European Digital Mathematical Library.
  \url{https://eudml.org/} [2014-09-23]

\bibitem{frame} Frame at Wikipedia.
  \url{http://en.wikipedia.org/wiki/Frame_(artificial_intelligence)}
      [2014-09-23]

\bibitem{GND} Gemeinsame Normdatei (GND). Informationsseite der
  Deutschen Nationalbibliothek.
  \url{http://www.dnb.de/DE/Standardisierung/GND/gnd_node.html}
      [2014-09-13]

\bibitem{cicm-14} H.-G. Gr\"abe, A. Nareike, S. Johanning. The
  SymbolicData Project --- Towards a Computer Algebra Social Network.
  In: Workshop and Work in Progress Papers at CICM 2014. CEUR-WS.org
  vol. 1186.  \url{http://ceur-ws.org/Vol-1186/#paper-21}
  [2014-09-13]

\bibitem{reframing} S. Kaufman, M. Elliott, D. Shmueli: Frames,
  Framing and Reframing.  In: G. and H. Burgess (eds.): Beyond
  Intractability. Conflict Information Consortium, University of
  Colorado, Boulder. Posted: September 2003. 
  \url{http://www.beyondintractability.org/essay/framing} [2014-09-23]

\bibitem{kohlhase} A. and M. Kohlhase: Spreadsheet Interaction with
  Frames: Exploring a Mathematical Practice. In: Intelligent Computer
  Mathematics, Lecture Notes in Computer Science Volume 5625, 2009, pp
  341--356.

\bibitem{MKM} Mathematical Knowledge Management. The MKM interest
  group.  \url{http://www.mkm-ig.org/} [2014-09-13]

\bibitem{sdsage} A.\ Nareike. The {\SD} sdsage package.
  \url{http://symbolicdata.org/wiki/PolynomialSystems.Sage}
      [2014-02-28]

\bibitem{PitmannLynch} J. Pitman, C. Lynch: Planning a 21st Century
  Global Library for Mathematics Research.  Notices of the AMS, August
  2014.

\bibitem{Sagemath} Sage -- a free open-source mathematics software
  system.  \url{http://www.sagemath.org} [2014-02-19]

\bibitem{postagging} U. Sch\"oneberg, W. Sperber.  POS Tagging and its
  Applications for Mathematics.  In: Intelligent Computer Mathematics.
  Lecture Notes in Computer Science, Volume 8543, 2014, pp 213--223.

\bibitem{sdsparql} The {\SD} SPARQL Endpoint.
   \url{http://symbolicdata.org:8890/sparql} [2014-02-19]

\bibitem{casnsparql} The CASN SPARQL Endpoint.
   \url{http://symbolicdata.org:8891/sparql} [2014-09-13]

\bibitem{sdwiki} The {\SD} Project Wiki.
   \url{http://wiki.symbolicdata.org} [2014-09-13]

\bibitem{dssn} S.\ Tramp et al.: An Architecture of a Distributed Semantic
  Social Network. In: Semantic Web Journal 2012, Special Issue on The Personal and
  Social Semantic Web.%
  \url{http://www.semantic-web-journal.net/sites/default/files/swj201_4.pdf}
      [2014-09-23]

\bibitem{Virtuoso} Virtuoso Open-Source Edition. 
  \url{http://virtuoso.openlinksw.com/} [2014-02-19]

\bibitem{WDML} World Digital Mathematics Library (WDML). 
  \url{http://www.mathunion.org/ceic/wdml/} [2014-09-23] 

\bibitem{cafg} Website of Fachgruppe Computeralgebra  
  \url{http://www.fachgruppe-computeralgebra.de/} [2014-03-06]

\end{thebibliography}

%%%%%%%%%%%%%%%%%%%%%%%%%%%%%%%%%%%%%%%%%%%%%%%%%%%%%%%%%%%%%%%%%%%%%%%%%%%%%%%%

\end{multicols}

\end{document}

