\documentclass{llncs}
\usepackage{url}

\newcommand{\SD}{{\sc Symbo\-lic\-Data}}
\def\pw{{\char94}}
\def\dq{{\char34}}

\title{The {\SD} Project in a Computer Algebra Social Network Perspective.\\
  Some Architectural Considerations}
\author{Hans-Gert Gr\"abe}
\institute{Leipzig University, Leipzig, Germany\\
\email{graebe@informatik.uni-leipzig.de}}
\date{2016-03-19}
\begin{document}
\maketitle

\begin{abstract}  
  On March 1, 2016, version 3.1 of the {\SD} data\-base was released. With the
  new release the {\SD} Project offers new, recompiled and extended data and
  introduces an adjusted git repo structure. The \emph{main goal} of the new
  release was directed towards an architectural redesign of the CASN
  subproject with the following main tasks:
\begin{itemize}
\item Enlarge the {\SD} People database both in the number of instances and
  with valuable additional information for author disambiguation -- one of the
  great challenges of all catalogue systems.
\item Strengthen the notion of a local CASN node maintained by a CA
  substructure as basis of an upcoming federated network of such nodes.
\item Reorganize the CASN data collected so far according to these adjusted
  conceptional basis using established semantic web best practices.
\end{itemize}
In this paper we explain the conceptional background of such a redesign in
more detail and put it in the context of some architectural considerations. 
\end{abstract}

\section{Introduction}
The allocation of resources for a sustainably available research
infrastructure seems to be a great challenge in particular to smaller
scientific communities. The {\SD} Project witnesses the peaks and troughs of
such efforts. It grew up from the Special Session on Benchmarking at the 1998
ISSAC conference in a situation where the research infrastructure built up
within the PoSSo \cite{PoSSo} and FRISCO \cite{FRISCO} projects -- the
Polynomial Systems Database -- was going to break down. After the end of the
projects' fundings there was neither a commonly accepted process nor dedicated
resources to keep the data in a reliable, concise, sustainably and digitally
accessible way. Even within the ISSAC Special Session on Benchmarking the
community could not agree upon a further roadmap to advance that matter.

At those times almost 20 years ago most of the nowadays well established
concepts and standards for storage and representation of research data did not
yet exist -- even the first version of XML as a generic markup standard had to
be accepted by the W3C. It was Olaf Bachmann and me who developed during
1999--2002 with strong support by the Singular group concepts, tools and data
structures for a structured representation and storage of this data and
prepared about 500 instances from \emph{Polynomial Systems Solving} and
\emph{Geometry Theorem Proving} to be available within this research
infrastructure.

The main conceptional goal was a nontechnical one -- to develop a research
infrastructure that is independent of (permanent) project funding but operates
based on overheads of its users. This approach was inspired by the rich
experience of the Open Culture movement ``business models'' to run
infrastructures. It was an early attempt to emphasize the advantage of an
explicitly elaborated concept of a community-based solution to the ``tragedy
of the commons'' \cite{hardin} within the CA community and to apply such a
concept to run a part of its research infrastructure.  Even 15 years later it
remains difficult to keep the {\SD} Project running on such a base.
\medskip

During the last ten years with Open Access, Open Data and the emerging
semantic web the general understanding of the importance of such
community-based efforts to develop common research infrastructures matured.
This development was accompanied with conceptual, technological and
architectural standardization processes that had also impact on the
development of concepts and data structures within the {\SD} Project.  In 2009
we started to refactor the data along standard Semantic Web concepts based on
the Resource Description Framework (RDF).  With {\SD} version~3 released in
September 2013 we completed a redesign of the data along RDF based semantic
technologies, set up a Virtuoso based RDF triple store and an SPARQL endpoint
as Open Data services along Linked Data standards, and started both conceptual
and practical work towards a semantic-aware Computer Algebra Social Network.

Since then we continued that development.  On March 1, 2016, version 3.1 of
the {\SD} tools and data was released. The new release contains
\begin{itemize}
\item new resource descriptions (``fingerprints'' in the notion of
  \cite{cicm-14}) of remotely available data on transitive groups
  (\emph{Database for Number Fields} of Gunter Malle and J\"urgen Kl\"uners
  \cite{MalleKlueners}) and polytopes (databases of Andreas Paffenholz
  \cite{Paffenholz} within the \emph{polymake} project \cite{polymake}),
\item a recompiled and extended version of test sets from integer programming
  -- work by Tim R\"omer (\emph{normaliz} group \cite{normaliz}) --, 
\item an extended version of the \emph{SDEval benchmarking environment} -- work
  by Albert Heinle \cite{heinle-15} -- and
\item a partial integration ({\SD} People database, databases of upcoming and
  past conferences) of data from the CASN -- the Computer Algebra Social
  Network subproject.
\end{itemize}
Moreover, the github account \url{https://github.com/symbolicdata} was
transformed into an organizational account and the git repo structure was
redesigned better to reflect the special life-cycle requirements of the
different parts and activities within {\SD}. We provide the following repos
\begin{itemize}
\item \emph{data} -- the data repo with a single master branch mainly to backup
  recent versions of the data,
\item \emph{code} -- the code directory with master and develop branches,
\item \emph{maintenance} -- code chunks from different tasks and demos as best
  practice examples how to work with RDF based data,
\item \emph{publications} -- a backup store of the {\LaTeX} sources of {\SD}
  publications, 
\item \emph{web} -- an extended backup store of the {\SD} web site that
  provides useful code to learn how RDF based data can be presented.
\end{itemize}
The old repo \emph{symbolicdata} is deprecated and was removed from the github
account, so please adjust your local repo structure. The main development is
coordinated within the \emph{{\SD} Core Team} (Hans-Gert Gr\"abe, Ralf
Hemmecke, Albert Heinle) with direct access to the organizational account.  We
refer to the {\SD} Wiki \cite{sdwiki} for more details about the new release. 
\medskip

All changes reported so far are mentionable advances of the {\SD} Project.
Nevertheless the \emph{main goal} of the new release was directed towards an
architectural redesign of the CASN subproject with the following main tasks:
\begin{itemize}
\item Enlarge the {\SD} People database both in the number of instances and
  with valuable additional information for author disambiguation -- one of the
  great challenges of all catalogue systems, see, e.g., VIAF \cite{viaf}.
\item Strengthen the notion of a local CASN node maintained by a CA
  substructure as basis of an upcoming federated network of such nodes -- in a
  first step such a node exposes valuable information in RDF as files for
  download and further exploration in a local RDF store.
\item Reorganize the CASN data collected so far according to this adjusted
  conceptual basis using established semantic web best practices.
\end{itemize}
In this paper we explain the conceptual background of such a redesign in more
detail and put it in the context of some architectural considerations.

\section{Semantic Web as Web of Data}

At about 2000 a crucial redesign of the web started to increase its potential
for deeper and more intense social cooperation. Notions as Web 2.0, Web 3.0,
Web of Data or Semantic Web were coined to reflect about such developments from
an -- in the first place -- technical point of view.  Shortly it turned out
that many of the effects of the ``Web 2.0'' were based on the new digital
concepts and tools but in fact were triggered by new \emph{opportunities of
  social interaction} and could not be reflected properly by a technologically
centered approach. The competing notion of \emph{Social Web} emphasized such
aspects and the proponents of a more infrastructural-technical approach coined
notions as \emph{Web 3.0} or -- within the W3C since the end of 2013 --
\emph{Web of Data}.

The notion of \emph{Semantic Web} addressing mainly standardization processes
is located in the middle between both perspectives. The English Wikipedia
\cite{Wikipedia:SemanticWeb} puts it as follows:
\begin{quote}
The Semantic Web is an extension of the Web through standards by the World Wide
Web Consortium (W3C). The standards promote common data formats and exchange
protocols on the Web, most fundamentally the Resource Description Framework
(RDF).

According to the W3C, ``The Semantic Web provides a common framework that
allows data to be shared and reused across application, enterprise, and
community boundaries''. The term was coined by Tim Berners-Lee for a web of
data that can be processed by machines. {\ldots} In 2006, Berners-Lee and
colleagues stated that: ``This simple idea {\ldots} remains largely
unrealized''.
\end{quote}
Semantic Web remains to be a complex socio-technical field, and in Berner-Lee's
statement optimism and pessimism are close together -- the optimistic
perspective concerns the potential of the ongoing development and the
pessimistic one the time horizon of that development to mature.

The core of the (technical) vision of a ``semantic web as web of data'' is an
architectural change of the web itself. Web 1.0 started and matured as a web of
interlinked presentations, in the first time as hyperlinked static HTML pages
connected via the HTTP hypertext protocol. Modern web engines deliver
dynamically generated HTML code (in many cases with additional Javascript
driven visualization code and dynamic effects) and provide interlinking of
specially prepared information pieces between different presentation layers --
the ``content''.  Such web applications are typically designed along a 3-tier
architecture with data layer, business or model layer and presentation layer.
All modern web frameworks and web design pattern as, e.g., the
Model-View-Controller (MVC) or the Presentation-Abstraction-Control (PAC)
pattern, support and presuppose such a layered architecture.  The main drawback
of such an architecture is its tight coupling within a local web server and its
direction towards information supply \emph{ready for use}, i.e., provision of
\emph{interpreted data}, mainly controlled by the needs and world perception of
the information \emph{providers}.

Unfortunately, information reveals usefulness often enough in new, unexpected
contexts, not foreseen and even not predictable by the information
provi\-ders.  This is the starting point of the vision of the semantic web --
use existing protocols, in particular HTTP, to connect the data layers
directly and to combine ``uninterpreted'' data from different sources in
machine readable form under the control (and interpretation) of the \emph{data
  user}, not the data provider.

\section{Data Architects and the Semantic Web as Social Web}

A deeper analysis of the problem reveals that there is no (interesting)
``uninterpreted'' data. A bit stream of data exchanged between two computers
can, e.g., represent a picture, if the computer recognizes the image format
(analyzing the file name extension or detecting an appropriate pattern prefix
in the bit stream -- a standard to be agreed upon \emph{before} exchanging the
first picture, and even before writing such a program), and the computer can
``interpret'' the bit stream starting a special algorithm to render the
picture.  The picture itself is ``reinterpreted'' once more by the user who
called that picture with a certain goal in mind. Such multilevel
interpretational processes are ubiquitous in the web and it is one of the
difficult problems to harmonize such interpretational frames within social
communication rooms.

Thus one of the central challenges of digital communication is the coordination
of conceptual and notational conventions on a level of detail that can be
algorithmically processed by digital machines.  Nowadays the most successful
semantic web projects address domains with sophisticated taxonomies and
systematics well established already in a predigital era that have ``only'' to
be fully formalized for digital use.  All other tasks of semantic web
standardization turn out to be much more complicated and require social virtues
as open mind, open culture, empathy and readyness for cooperation compared to
well established virtues as rivalness and competitiveness.

Such requirements led to a completely new job profile within computer science
-- in addition to \emph{software architects}, who are responsible for a
reasonable program architecture within a single application or application
system, nowadays we have \emph{data architects} to design and (socially)
implement powerful comprehensive data models (``ontologies'') for
interpretational frames that facilitate cooperation between applications from
different domains or, more precisely, between the users of those applications.

The systematic development of such ontologies can be supported by tools but is
in its core a socially triggered process.  Thus from the perspective of a data
architect the web is more a web of people driven by different interests and
goals using and producing data as a web of data. The ``web of data'' metaphor
masks not only the users of this data and their goals but also the coordination
processes required to use the ``web of data'' in a sound way.  It is a great
challenge for data architects to design not only technical and notational
architectures (and models) but also processual and social architectures to get
the semantic web mature. 

RDF as language concept and framework offers its strength in such a domain --
due to its generic concept RDF is appropriate to be used for modelling
processes at both the data and the metadata level and can be used to describe
not only data (resources in the RDF terminology) but also data structures
(resource descriptions in the RDF terminology), metadata structures (languages
or meta-meta models in other terminologies) and descriptions at even more
elaborated abstract levels to support processual and even social cooperation.

\section{OpenDreamKit and the \texttt{MathHub.info} Project}

Semantic web concepts largely influenced also the research infrastructure of
science.  Within the domain of mathematics there are big projects on the way as
WDML \cite{PitmannLynch,WDML} or EuDML \cite{EuDML} and also smaller ones as
the semantification of the MSC2010 index \cite{MSC2010} or the swMATH project
\cite{swmath}.

Beyond such domain-specific efforts there is not only \emph{Google Scholar} but
much more combined e-science projects with global scope are on the way to
support and restructure scholarly communication using semantic technologies as,
e.g., VIVO\footnote{\url{http://vivoweb.org}. ``{\ldots} VIVO supports
  recording, editing, searching, browsing and visualizing scholarly
  activity. VIVO encourages research discovery, expert finding, network
  analysis and assessment of research impact.  {\ldots}''. \cite{vivo}},
VIAF\footnote{``The Virtual International Authority File (VIAF) is an
  international service designed to provide convenient access to the world's
  major name authority files.  {\ldots}''. \cite{viaf}} or OCLC\footnote{``A
  global library cooperative that provides shared technology services, original
  research and community programs for its membership and the library community
  at large''.  \url{http://oclc.org}}.

During the last years also the funding agencies increasingly noticed the
importance of the digital extension of research infrastructures, see, e.g., the
ESFRI Roadmap 2016 \cite{esfri}. In particular, on 17 April 2015 the EU funding
agency published a call ``European Research Infrastructure'' \cite{h2020}
within the Horizon2020 Work Programme 2014--2015 with focus on
\begin{itemize}
\item e-infrastructure for Open Access
\item Research Data Alliance
\item High Performance Computing
\item Research and Education Networking
\item Virtual Research Environments
\item Support measures to innovation, human resources etc. 
\end{itemize}
First of all notice, that also the promotion strategy of the EU funding agency
is orthogonal to the ``web of data'' debates, strongly aims at \emph{social
  practices} that already proved to be successful and focuses on proposals to
reinforce such \emph{practices} and not the infrastructure itself.  Further
infrastructural development and thus the ``web of data'' is considered rather
as an aspect of continuation and institutionalization of successful social
practices thus performing a funding practice that is oriented at intrinsically
motivated processes of academic self organization compared to a system of
externally triggered patronage.  For primarily politically motivated funding
practices this is a quite remarkable development. 

Similar aspects can be observed for the \emph{OpenDreamKit Project} \cite{odk}
that was successful within the above mentioned subcall on Virtual Research
Environments.  It was awarded not for promising to develop such an environment
but since \emph{successful cooperative practices} in the area of mathematical
software as ``toolkit for the advancement of mathematics'' were already
developed during the last years, notably around the SageMath Project
\cite{sagemath} despite adverse funding conditions.  The proposed development
direction towards a (also socially highly distributed and interconnected, i.e.,
\emph{organismically} structured) ``sustainable ecosystem of
community-developed open software, databases, and services'' is emphasized in
the project's abstract \cite{odk}:
\begin{quote}
  OpenDreamKit will be built out of a sustainable ecosystem of
  community-developed open software, databases, and services, including popular
  tools such as LinBox, MPIR, Sage (sagemath.org), GAP, PariGP, LMFDB, and
  Singular. We will extend the Jupyter Notebook environment to provide a
  flexible UI. By improving and unifying existing building blocks, OpenDreamKit
  will maximize both sustainability and impact, with beneficiaries extending to
  scientific computing, physics, chemistry, biology and more and including
  researchers, teachers, and industrial practitioners.
\end{quote}
Such an emphasis on \emph{social interaction} is set explicitly also on the
project's reflection focus: ``Our architecture will be informed by recent
research into the sociology of mathematical collaboration, so as to properly
support actual research practice'' \cite{odk}.
\medskip

Since public funding policies are driven mainly by territorially oriented
political structures and decision processes it is a great challenge to smaller
academic communities that are usually structured thematically and not
territorially to adopt such developments for their own scientific communication
processes and to join forces with other scientific communities to get own
requirements publicly recognized.  The OpenDreamKit Project addresses the
research infrastructure of mathematics as a whole, hence its target is a large
enough academic community to generate funding on an EU level. The focus of the
{\SD} CASN subproject is directed towards the much smaller academic community
of Computer Algebra researchers.  Hence the problem to organize its own
research infrastructure is different. I'll come back to this question in the
next section.  
\medskip

In the remainder of this section I will discuss on the example of
\texttt{MathHub.info} as subproject of OpenDreamKit in more detail the
interplay between semantic technologies and their embedding into a funding
structure for research infra\-structures that is mainly driven by successful
social practices.

Michael Kohlhase reported in two contributions \cite{planetary,mathhub} to the
session \emph{Projects and Surveys} at CICM 2012 and CICM 2014 about efforts to
``develop a general framework -- the Planetary system -- for social semantic
portals that support users in interacting with STEM\footnote{STEM is a shortcut
  for ``Science, Technology, Engineering, Mathematics''.} documents {\ldots}''
\cite{planetary} and started with \texttt{MathHub.info} to build up such a
research infrastructure.  It is a very interesting approach to enrich
established STEM technologies (in particular {\LaTeX} and arXiv) semantically
and seems to be the only OpenDreamKit subproject that addresses semantic web
concepts and tools explicitly.

In many cases such a situation runs the risk to be caught between the devil and
the deep blue sea. Kohlhase described the design goal of the system in
\cite{mathhub} as follows:
\begin{quote}
  \texttt{MathHub.info} must satisfy two conflicting goals: On the one hand, it
  must be so generic that it is open to all logics and implementations; on the
  other hand, it must be aware of the semantics of the formalized content so
  that it can offer meaningful services. 
\end{quote}
Elaborated goals require elaborated tools and skilled users, even if
``meaningful services'' could be offered. The experience within the {\SD}
Project indicates that such a design goal is very ambitious and requires a
strategy of \emph{social} communication on a long run to get its results
to be accepted and tools to be used by the target community.

Running a research infrastructure and providing reliable access to it is a
cross cutting concern \cite{ccc} orthogonal to the core research concerns of
any special interest groups. It is ``nice to have'' and ``hard to get'' even
if it has plenty of advantages that Kohlhase described in \cite{mathhub} in
such a way:
\begin{quote}
  We claim that \texttt{MathHub.info} will resolve two major bottlenecks in the
  current state of the art. It will provide a permanent archiving solution that
  not all systems and user communities can afford to maintain separately. And
  it will establish a standardized and open library format that serves as a
  catalyst for comparison and thus evolution of systems.
\end{quote}
This statement is a statement \emph{before} the OpenDreamKit project matured.
To promote his own core goal Kohlhase offers additional benefits that have not
much to do with semantic technologies but address \emph{social needs} and
proposes \emph{social practices} for the target community.  Within OpenDreamKit
this proposal encounters \emph{established practices} of archiving, versioning,
and evolution of the other subprojects. It remains to be seen to what extend
Kohlhase succeeds to tie semantic concepts and technologies into such a context
with established practices that previously were mainly unaware of such semantic
concepts.  In the pre-OpenDreamKit time the main problem was to develop
sustainable social practices around semantic concepts, the new challenge is to
integrate semantic concepts into established (social) practices. 
\medskip

Let me close this section with another point. From an architectural point of
view \texttt{MathHub.info} tries to integrate elements of a local working place
environment and access elements to a global infrastructure that has yet to
emerge. The main focus, see fig.~1 in \cite{mathhub}, lies on the functionality
of a local working place environment.  The structural aspects of the formation
of a global data infra\-structure remain hidden in the boxes ``MMT'' and
``GitLab'' in that figure. Thus the design focuses on \emph{software
  architectural} aspects and does not address the (social) requirements of
\emph{data architecture} building processes.

Ten years of semantic web experience indicate that data architectural aspects
are in the core of the building processes of the ``web of data'' and require
social organization in such a way that the (yet emerging) global data structure
fits with a large number of \emph{different} software frameworks and
architecture models nowadays in use at local sites.

\section{Extending the {\SD} Data Store}

During the last years the {\SD} Project adjusted its focus to address also
more general technical and social aspects of a semantically enriched research
infrastructure within the domain of Computer Algebra based on RDF for
representation of intercommunity and relational information.

Such a change of the focus had its impact on several earlier design decisions
of the data store itself. Enlarging the database of {\SD} we gained the
following experience:
\begin{itemize}
\item The CA community consists of several subcommunities with own concepts,
  notational conventions, semantic-aware tools and established communication
  structures.  

  There is no need to duplicate such structures but to support the
  subcommunities to enrich semantically these communication processes.
\item We provide structural metadata (``fingerprints'' in the notion of
  \cite{cicm-14}) of the different data sets at our central RDF store
  \cite{sdstore} but not necessarily duplicate the data itself. 

  Thus we rely on sustainably available research infrastructures of CA
  subcommunities and restrict our activities to a central search and filter
  service on the metadata level to find and identify data. This service is
  based on a generic semantic web concept, the SPARQL query language, and can
  be accessed via our SPARQL endpoint \cite{sdsparql}.

  We applied this principle to the newly integrated data sets on polytopes and
  on transitive groups and also within the recompiled version of test sets from
  integer programming.  Data are hosted by the \emph{polymake} group
  \cite{polymake}, within the \emph{Database for Number Fields}
  \cite{MalleKlueners} and by the \emph{normaliz} group \cite{normaliz}.
\item RDF is a useful and meanwhile well established standard for metadata and
  relational information, but there is no need and one cannot expect from CA
  subcommunities to give up established notational conventions in favour of
  RDF or XML markup. 

  Semantic-aware tools of the subcommunities are well tuned for the
  established notational conventions, and representation of data in a different
  format requires additional transformation effort to use it.

  Moreover, one can use MathML or OpenMath standards and tools for the casual
  exchange of data. Note nevertheless, that the notational conventions of a
  subcommunity use many shortcuts that are valid only in a special
  interpretational frame (the ``general nonsense'' of the field, well known to
  the specialists) that is hard and probably unnecessary to formalize, since
  practical use of data from a special field requires a minimum of
  semantic-awareness of the user itself.
\end{itemize}

\section{About the CASN Architecture}

The CASN subproject tries to embed aspects of the maintenance of the {\SD}
data store into a more general process of formation of a semantically enriched
social network of academic communication within the CA community in the sense
of a (social) ``web of people'' mentioned above.

As first question we had to decide about the range of such a social network.
This is a difficult question from a ``web of data'' perspective but easily
solved from a ``web of people'' perspective: To the network belong all
individuals who count themselves as computer algebraists or are mentioned as
such in communication processes already covered by the CASN.  For the moment
this input comes mainly from three mailing lists -- the [spp1489-gen] general
mailing list of the SPP 1489 within the German Fachgruppe, the [SIGSAM-MEMBERS]
mailing list and the [Om] mailing list of the OpenMath project.  We installed
our own mailing list [sd-announce] -- for the moment mainly to forward selected
mails from the [spp1489-gen] mailing list for archival purposes.  Open list
archives are required for referencing purposes in an (upcoming) semantically
enriched CASN news channel.

A first roadmap towards such a CASN and our experimental setting was described
in \cite{cicm-14} and developed further during the last years.  We try not to
``reinvent the wheel'' but to address the already existing ``CA memory'' -- a
huge number of very loosely related web pages about conferences, meetings,
working groups, projects, private and public repositories, private and public
mailing lists etc. Hence the main focus towards CASN is to develop a framework
based on modern semantic technologies for a decentralized network that
increases the awareness of the different parts of that already existing ``CA
network''.

We realized that this network itself is an ``overlay network'' that connects a
greater number of research networks of individuals around special topics with
own lightweight research infrastructures.  It is an interesting challenge for
semantic concepts to support the requirements of intercommunity communication
to exchange semantic content on different levels and different levels of
detail.
\medskip

As a coarse architectural concept to establish such a network we propose
\begin{itemize}
\item to operate a central RDF store with SPARQL endpoint providing the full
  bandwidth of Linked Open Data services and 
\item to convert nodes of the ``CA memory'' into CASN nodes providing part of
  their data in structured RDF format for easy access and exchange.
\end{itemize}
{\SD} version 3.1 is a first step in that direction since
\begin{itemize}
\item several data from the formerly separate CASN RDF store are now
  integrated with the {\SD} main RDF store \cite{sdstore} and
\item the experimental setting of the semantic support of the website of the
  German Fachgruppe \cite{cafg} was reorganized as a first CASN node.
\end{itemize}

\subsection*{CASN Integration into the {\SD} RDF Store}

The CASN Integration into the {\SD} RDF Store covers the following topics:
\begin{itemize}
\item The RDF store provides information about scientific activities of people
  mainly extracted from conference announcements.  The personal information is
  stored as instances of the RDF type \texttt{foaf:Person} with (as subset of)
  keys \texttt{foaf:name}, \texttt{foaf:homepage} and \texttt{sd:affiliation}
  (a literal). Due to privacy reasons we do not provide \texttt{foaf:mbox}
  (email) values.

  This list is steadily enlarged and used as URI reference for reports about
  different activities (invited speakers, conference organizers etc.).  As of
  March 2016 the {\SD} People database contains 1036 \texttt{foaf:Person}
  entries from the CA scientific community that can be explored via the {\SD}
  SPARQL endpoint \cite{sdsparql} and also using the \emph{CA People Finder}
  at the {\SD} info page \cite{sdinfo}.

  In August 2014 we compiled a first alignment of the {\SD} People database
  with the ZBMath author database to evalute the potential of a
  community-based author disambiguation and could resolve 348 matchings out of
  678 persons.  The result is available as \texttt{ZBMathPeople} RDF graph in
  our database.

\item The RDF store provides information about upcoming CA conferences from
  several mailing lists, usually up to 20 entries with references to the {\SD}
  People database.

  The information is extracted via SPARQL query and displayed both in the
  Wordpress based site of the German Fachgruppe \cite{cafg} and at the {\SD}
  info page \cite{sdinfo}.
\item The RDF store provides information about past CA conferences with
  references to the {\SD} People database about speakers and organizers (as
  far as available).

  Most of the entries were moved from the upcoming CA conferences list to that
  list for archiving purposes and aligned according to their archival status.
  As of March 2016 there are 139 records of past CA conferences. 

  A short set of information is extracted via SPARQL query and displayed at
  the {\SD} info page \cite{sdinfo}.
\item As another feature we started to provide semantic annotations to a
  subset of news (beyond conference announcements) posted on several mailing
  lists as instances of RDF type \texttt{sioc:BlogPost}.  We operate a special
  mailing list \texttt{sd-announce} with archive and forward interesting news
  to that archive for URI reference if the original mailing list is not
  archived.

  Such an annotation contains an excerpt of that message in a standardized way
  that can be explored at the {\SD} info page \cite{sdinfo}. The concept
  can easily be extended to the concept of an CASN news channel. 
\end{itemize}

\subsection*{The CASN node of the German Fachgruppe}

The CASN node of the German Fachgruppe contains  
\begin{itemize}
\item a list of (extended) FOAF-Profiles used to render, e.g., the page
  \begin{center}
    \url{http://www.fachgruppe-computeralgebra.de/fachgruppenleitung/},
  \end{center}
\item lists of current and former members of the board of the German
  Fachgruppe,
\item (not up to date) information about German CA working groups, 
\item standardized information about the SPP 1489 projects with references to
  the {\SD} People database,
\item keyword enriched information about scientific publications in the
  CA-Rund\-brief of the German Fachgruppe using the \texttt{dcterms} ontology,
\item a survey on successfully defended CA dissertations in the scope of the
  Fachgruppe (a joint effort with the CA-Rundbrief) and
\item a (also not up to date) list of CA books.
\end{itemize}
The data is available in RDF format for direct download from a web directory
\begin{center}
  \url{http://www.fachgruppe-computeralgebra.de/rdf}
\end{center}
as part of an upcoming global RDF data structure and can be harvested and
processed within a local RDF store. The data is used in different PHP-based
presentations that are summarized at
\begin{center}
  \url{http://www.fachgruppe-computeralgebra.de/symbolicdata/}
\end{center}
Best practice code how to embed such information into a Wordpress based
website using the \emph{EasyRDF} PHP library and also the code to operate the
{\SD} info website \cite{sdinfo} is available from our git repos
\emph{maintenance} and \emph{web}.

\subsection*{The {\SD} CASN Node}

We also operate a rudimentary {\SD} CASN node at the publicly accessible
directory
\begin{center}
  \url{http://symbolicdata.org/rdf/}.
\end{center}
In the subdirectory \texttt{Conferences} we provide detailed information about
five CA conferences (the SPP annual meeting in Bad Boll 2014 and the CICM
conferences 2012--2015) as proof of concept of standardized detailed
conference reports using the \emph{Semantic Web Conference Ontology}
\cite{swc}.  The records provide information about the general venue,
programme committes, tracks and talks of the conferences.  

For the CICM conferences we exploited and transformed the publicly available
XML-based representation developed by Serge Autexier to render the conference
pages at \url{http://cicm-conference.org} as proof of concept.  The main
addition in our standardized detailed conference reports is the
(semi-automatic) people disambiguation that easily allows to relate conference
activities with other activities issuing a simple request to our SPARQL
endpoint \cite{sdsparql}.  

\raggedright
\begin{thebibliography}{xxx}
\bibitem{normaliz} Bruns, W., Ichim, B., R\"omer. T., Sieg, R., S\"oger, C.:
  Normaliz. Algorithms for Rational Cones and Affine Monoids.
  \url{https://www.normaliz.uni-osnabrueck.de}. [2016-03-08]
\bibitem{ccc} Cross Cutting Concern. From Wikipedia, the Free Encyclopedia.
  \url{https://en.wikipedia.org/wiki/Cross-cutting_concern}. [2016-03-08]
\bibitem{EuDML} EuDML. The European Digital Mathematical Library.
  \url{https://eudml.org/}. [2015-09-23]
\bibitem{esfri} Strategy Report on Research Infrastructures.  Roadmap 2016.
  Published by the European Strategy Forum for Research Infrastructures
  (ESFRI), Br\"ussel (2016).  \url{http://www.esfri.eu/roadmap-2016}.
  [2016-03-16]
\bibitem{FRISCO} FRISCO -- A Framework for Integrated Symbolic/Numeric
  Computation. (1996--1999).  \url{http://www.nag.co.uk/projects/FRISCO.html}.
  [2016-02-19]
\bibitem{polymake} Gawrilow, E., Joswig, M.: Polymake: a Framework for
  Analyzing Convex Polytopes. In: Kalai, G., Ziegler, G.M. (eds.), Polytopes --
  Combinatorics and Computation (Oberwolfach, 1997), pp. 43--73, DMV Sem., 29,
  Birkh\"auser, Basel (2000). 
\bibitem{cicm-14} Gr\"abe, H.-G., Johanning, S., Nareike, A.: The {\SD} Project
  -- Towards a Computer Algebra Social Network. In: Workshop and Work in
  Progress Papers at CICM 2014, CEUR-WS.org, vol. 1186 (2014).
\bibitem{hardin} Hardin, G.: The Tragedy of the Commons. Science 162 (3859),
  pp. 1243--1248 (1968). \url{doi:10.1126/science.162.3859.1243}. 
\bibitem{heinle-15} Heinle, A., Levandovskyy, V.: The SDEval Benchmarking
  Toolkit. ACM Communications in Computer Algebra, vol. 49.1, pp. 1--10 (2015).
\bibitem{h2020} Research Infrastructures, including e-Infrastructures.
  \url{http://ec.europa.eu/programmes/horizon2020/en/h2020-section/research-infrastructures-including-e-infrastructures}. [2016-03-16]
\bibitem{mathhub} Iancu, M., Jucovschi, C., Kohlhase, M., Wiesing, T.: System
  Description: MathHub.info. In: Watt, S.M., Davenport, J.H., Sexton, A.P.,
  Sojka, P., Urban, J. (eds.), Intelligent Computer Mathematics. LNCS vol.
  8543, pp. 431--434 (2014).
\bibitem{MalleKlueners} Kl\"uners, J., Malle, G.: A Database for Number Fields.
  \url{http://galoisdb.math.uni-paderborn.de/}. [2016-03-08]
\bibitem{planetary} Kohlhase, M.: The Planetary Project: Towards eMath3.0. In:
  Jeuring, J., Campbell, J.A., Carette, J., Dos Reis, G., Sojka, P., Wenzel,
  M., Sorge, V. (eds.), Intelligent Computer Mathematics.  LNCS vol. 7362,
  pp. 448--452 (2012).
\bibitem{MSC2010} Lange, C., Ion, P., Dimou, A., Bratsas, C., Corneli, J.,
  Sperber, W., Kohlhase, M., Antoniou, I.: Reimplementing the Mathematics
  Subject Classification (MSC) as a Linked Open Dataset.  In: Jeuring, J.,
  Campbell, J.A., Carette, J., Dos Reis, G., Sojka, P., Wenzel, M., Sorge,
  V. (eds.), Intelligent Computer Mathematics.  LNCS vol. 7362, pp. 458-462
  (2012).
\bibitem{odk} OpenDreamKit: Open Digital Research Environment Toolkit for the
  Advancement of Mathematics. \url{http://opendreamkit.org/},
  \url{http://cordis.europa.eu/project/rcn/198334_en.html}. [2016-03-16]
\bibitem{Paffenholz} Paffenholz, A.: Polytope Database.
  \url{http://www.mathematik.tu-darmstadt.de/~paffenholz/data/}.  [2016-03-08] 
\bibitem{PitmannLynch} Pitman, J., Lynch, C.: Planning a 21st Century
  Global Library for Mathematics Research.  Notices of the AMS, August
  2014.
\bibitem{PoSSo} The PoSSo Project. Polynomial Systems Solving -- ESPRIT III BRA
  6846.  (1992--1995).
  \url{http://research.cs.ncl.ac.uk/cabernet/www.laas.research.ec.org/esp-syn/text/6846.html}. 
      [2016-03-16]
\bibitem{sagemath} The SageMath Project.  \url{http://www.sagemath.org/}.
  [2016-03-16]
\bibitem{Wikipedia:SemanticWeb} Semantic Web. From Wikipedia, the Free
  Encyclopedia. \url{https://en.wikipedia.org/wiki/Semantic_Web}. [2016-03-07]
\bibitem{swc} The Semantic Web Conference Ontology.
  \url{http://data.semanticweb.org/ns/swc/swc_2009-05-09.html}.  [2016-03-07]
\bibitem{swmath} swMATH -- a new Information Service for Mathematical Software.
  \url{http://www.swmath.org/}. [2016-03-07]
\bibitem{sdinfo} The {\SD} Info Page.  \url{http://symbolicdata.org/info}.
  [2016-03-08]
\bibitem{sdstore} The {\SD} RDF Data Store.
  \url{http://symbolicdata.org/Data}.  [2016-03-15]
\bibitem{sdsparql} The {\SD} SPARQL Endpoint.
  \url{http://symbolicdata.org:8890/sparql}. [2016-02-19]
\bibitem{sdwiki} The {\SD} Project Wiki.
   \url{http://wiki.symbolicdata.org}. [2016-03-13]
\bibitem{viaf} VIAF -- the Virtual International Authority File.
  \url{http://viaf.org/}. [2016-03-07]
\bibitem{vivo} VIVO -- an Open Source Software and Ontology for Representing
  Scholarship.  \url{https://wiki.duraspace.org/display/VIVO/VIVO}.
  [2016-03-07]
\bibitem{cafg} Website of the German Fachgruppe Computeralgebra.   
  \url{http://www.fachgruppe-computeralgebra.de/}. [2016-03-06]
\bibitem{WDML} World Digital Mathematics Library (WDML).
  \url{http://www.mathunion.org/ceic/wdml/}. [2015-09-23]

\end{thebibliography}

\end{document}
